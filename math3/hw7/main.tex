\documentclass{article}
\usepackage{amsmath, amsthm, amssymb}
\usepackage{tikz}
\usepackage{array}
\usepackage{mathtools}
\usepackage{graphicx}
\usepackage{hyperref}

\begin{document}
\section*{\huge Homework Sheet 7}
\begin{flushright}
   \textbf{Author: Abdullah Oğuz Topçuoğlu}
\end{flushright}

% Exercise 25 (4 Points)
% A fair dice (with six faces) is rolled three times. Choose a suitable probability space (Ω, F, P) to
% model this random experiment. Describe the events A1, A2, A1 ∩ A3 and (A1 ∪ A2) \ A3 as subsets
% of Ω and determine their probabilities, where A1, A2, A3 are described verbally as follows:
% A1 : “The sum of the number of pips of the first and the second throw is smaller than five.”
% A2 : “The product of the number of pips of the second and third throw is six.”
% A3 : “The number of pips of the second throw is two.”
% Exercise 26 (4 Points)
% Tutorials A, B, and C of a computer science course are attended by 7, 12, and 11 students, respectively. All 30 students work independently and hand in their solutions to the teaching assistant. The
% submitted solutions are gathered in random order on a desk. The teaching assistant draws three
% of these solutions at random. Choose a suitable probability space (Ω, F, P) to model this random
% experiment, describe the following events as subsets of Ω and determine their probabilities:
% A1 : The first solution drawn is from tutorial group B.
% A2 : All solutions drawn are from different tutorial groups.
% Exercise 27 (4 Points)
% In a raffle, 200 tickets are sold, of which 160 are blank, 38 are non-cash prizes, and two are top
% prizes. Alice buys the first 10 tickets and draws them randomly from the raffle drum. Choose a
% suitable probability space (Ω, F, P) to model this random experiment, describe the following events
% as subsets of Ω and determine their probabilities:
% A1 : Among the 10 tickets, there are one top prize, and at least two non-cash prizes.
% A2 : Less than 6 of the 10 tickets are blank tickets.
% Hint: In this setting, it is not necessary to take care of the order in which the tickets are drawn.
% Exercise 28 (4 Points)
% Let N := {1, 2, . . .}. Let p1, p2 : {1, . . . , 6} → R and pq : N → R, q ∈ (0, 1), be defined as follows:
% p1(k) :=
% 
% 
% 
% −
% 1
% 3
% , k ∈ {1, 3}
% −
% 1
% 6
% , k ∈ {2, 5}
% 1, k ∈ {4, 6}
% , p2(k) :=
% 
% 
% 
% 1
% 3
% , k ∈ {1, 3}
% 1
% 6
% , k ∈ {2, 5}
% 1
% 2
% , k ∈ {4, 6}
% , pq(k) := q
% k
% .
% (i) Are p1, p2 probability counting densities? Justify your answer.
% (ii) Find a linear combination of p1 and p2, by determining α, β ∈ R, such that αp1 + βp2 is a
% probability counting density.
% (iii) For which q ∈ (0, 1) is pq a probability counting density?

\section*{Exercise 25}
We choose the probability space
\begin{align*}
   \Omega &= \{1, 2, 3, 4, 5, 6\}^3, \\
   \mathcal{F} &= \text{Power set of } \Omega, \\
   P(A) &= U_\Omega
\end{align*}

The events are
\begin{align*}
   A_1 &= \{(a,b,c) \in \Omega \mid a + b < 5\}, \\
   A_2 &= \{(a,b,c) \in \Omega \mid b \cdot c = 6\}, \\
   A_3 &= \{(a,b,c) \in \Omega \mid b = 2\} \\
   A_1 \cap A_3 &= \{(a,2,c) \in \Omega \mid a < 3\}, \\
   (A_1 \cup A_2) \setminus A_3 &= \{(a,b,c) \in \Omega \mid (a + b < 5) \text{ or } (b \cdot c = 6) \text{ and } b \neq 2\}
\end{align*}

Since \(P\) is uniform we count the number of elements in each event.

Calculating the probabilities:
\begin{align*}
   \intertext{6 many (a,b) pairs times 6 choices for c}
   P(A_1) &= \frac{6 \cdot 6}{216} = \frac{36}{216} = \frac{1}{6}, \\
   \intertext{4 many (b,c) pairs times 6 choices for a}
   P(A_2) &= \frac{6 \cdot 4}{216} = \frac{24}{216} = \frac{1}{9}, \\
   \intertext{6 choices for a times 1 choice for b times 6 choices for c}
   P(A_3) &= \frac{36}{216} = \frac{1}{6}, \\
   \intertext{2 choices for a times 1 choice for b times 6 choices for c}
   P(A_1 \cap A_3) &= \frac{2 \cdot 6}{216} = \frac{1}{18}, \\
\end{align*}

For the last one we need to calculate \(|A_1 \cup A_2|\) first:
\begin{align*}
   |A_1 \cup A_2| &= |A_1| + |A_2| - |A_1 \cap A_2|, \\
   |A_1| &= 36, \\
   |A_2| &= 24, \\
\end{align*}

Calculating \(|A_1 \cap A_2|\):
\begin{align*}
  &\text{if } b = 1,\ c = 6:\ a + 1 < 5 \Rightarrow a < 4 \Rightarrow 3 \text{ choices} \\
  &\text{if } b = 2,\ c = 3:\ a + 2 < 5 \Rightarrow a < 3 \Rightarrow 2 \text{ choices} \\
  &\text{if } b = 3,\ c = 2:\ a + 3 < 5 \Rightarrow a < 2 \Rightarrow 1 \text{ choice} \\
  &\text{if } b = 6,\ c = 1:\ a + 6 < 5 \Rightarrow \text{no choices}.
\end{align*}
So \(|A_1 \cap A_2| = 3 + 2 + 1 + 0 = 6\).
\begin{align*}
   |A_1 \cup A_2| &= 36 + 24 - 6 = 54, \\
   P(A_1 \cup A_2) &= \frac{54}{216} = \frac{1}{4}, \\
   P((A_1 \cup A_2) \setminus A_3) &= P(A_1 \cup A_2) - P(A_3) + P(A_1 \cap A_3) = \frac{1}{4} - \frac{1}{6} + \frac{1}{18} = \frac{5}{36}.
\end{align*}

\section*{Exercise 26}
Lets define a set for students
\begin{align*}
   S := \{ s \in \mathbb{N} \; | \; s < 30 \}
\end{align*}

Each number represents a student. And note that \(|S| = 30\). And the first 7 students belong to tutorial A
following 12 students belong to tutorial B and the last 11 students belong to tutorial C. \\
We choose the probability space
\begin{align*}
   \Omega &= \{ (a,b,c) \; | \; a,b,c \in S \text{ and } a \neq b \neq c \}, \\
   \mathcal{F} &= \text{Power set of } \Omega, \\
   P(A) &= U_\Omega
\end{align*}
We can choose uniform distribution because each student has equal chance of being selected. \\
The events are:
\begin{align*}
   A_1 &= \{ (a,b,c) \in \Omega \; | \; a \in \{7,\ldots,18\} \}, \\
   A_2 &= \{ (a,b,c) \in \Omega \; | \; \big|\{ p \in \{a,b,c\} \; | \; p \in \{ 0, \ldots,6 \} \}\big| = 1 \quad \land \\
   &\big|\{ p \in \{a,b,c\} \; | \; p \in \{ 7, \ldots,18 \} \}\big| = 1 \quad \land \\
   &\big|\{ p \in \{a,b,c\} \; | \; p \in \{ 19, \ldots,29 \} \}\big| = 1  \}
\end{align*}

Here note that \(|\Omega| = 30 \cdot 29 \cdot 28\)

Calculating the probabilities:
\begin{align*}
   \intertext{12 choices for the first paper, 29 choices for the second paper and 28 choices for the third paper}
   P(A_1) &= \frac{12 \cdot 29 \cdot 28}{|\Omega|} = \frac{12 \cdot 29 \cdot 28}{30 \cdot 29 \cdot 28} = \frac{12}{30} = \frac{2}{5}, \\
   \intertext{choose a student from tutorial A (7 choices), tutorial B (12 choices) and tutorial C (11 choices). Then permute them (3! = 6) since order doesnt matter}
   P(A_2) &= \frac{7 \cdot 12 \cdot 11 \cdot 6}{|\Omega|} = \frac{7 \cdot 12 \cdot 11 \cdot 6}{30 \cdot 29 \cdot 28} = \frac{5544}{24360} = \frac{231}{1015}.
\end{align*}

\end{document}