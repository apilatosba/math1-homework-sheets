\documentclass{article}
\usepackage{amsmath, amsthm, amssymb}
\usepackage{tikz}
\usepackage{array}
\usepackage{mathtools}
\usepackage{graphicx}

\begin{document}
\section*{\huge Homework Sheet 3}
\begin{flushright}
   \textbf{Author: Abdullah Oğuz Topçuoğlu}
\end{flushright}

% Exercise 9 (4 Points)
% Let f : R
% 2 → R be a function defined by
% f(x1, x2) := ( x
% 2
% 1
% x2
% x
% 2
% 1+x
% 2
% 2
% , (x1, x2) ̸= (0, 0)
% 0 , (x1, x2) = (0, 0)
% .
% (i) Show that f is continuous at the point (0, 0).
% (ii) Show that f is not totally differentiable at the point (0, 0). Hint: Use 1.7.2 and 1.7.3.
% Exercise 10 (4 Points)
% Let f : R
% 3 → R
% 2 be a function defined by
% f(x1, x2, x3) := 
% x
% 4
% 1
% ln(3 + 2x
% 2
% 2
% )
% x1 sin(x2x3)e
% x1
% 
% .
% (i) Show that f is a C
% 1
% function.
% (ii) Is f totally differentiable? Justify your answer.
% Exercise 11 (4 Points)
% Let f1 : R → R
% 2
% , f2 : R
% 2 → R
% 3
% , f3 : R
% 3 → R be functions defined by
% f1(x) := [x
% 2
% , 1]⊤, f2(y1, y2) := [sin(y1), cos(y2),sin(y1) + cos(y2)]⊤, f3(z1, z2, z3) := e
% z1+z2+z3
% .
% Solve the following two problems using the chain rule (in the form of Theorem 1.7.8):
% (i) Explain why the map f3 ◦ f2 ◦ f1 is totally differentiable.
% (ii) Determine the total derivative of f3 ◦ f2 ◦ f1 at any point x ∈ R.
% Exercise 12 (4 Points)
% Let f : R
% 2 → R be a function defined by
% f(x1, x2) := (x
% 2
% 1 + 2x2) ln(1 + x
% 2
% 2
% )
% (i) Show that f is a C
% 1
% function.
% (ii) Determine the directional derivative of f at any point x = (x1, x2) ∈ R
% 2
% in direction v for
% v = v1, v2, v3, where
% v1 := 
% 0
% 1
% 
% , v2 := 
% − cos(α)
% sin(α)
% 
% , v3 :=
% 1
% √
% e
% 2 + π
% 2
% 
% −e
% π
% 
% for some fixed α ∈ (0, 2π).

\section*{Exercise 9}
We are given the function:
\begin{align*}
   f(x_1, x_2) :=
   \begin{cases}
      \frac{x_1^2 x_2}{x_1^2 + x_2^2}, & (x_1, x_2) \neq (0, 0) \\
      0,                               & (x_1, x_2) = (0, 0)
   \end{cases}
\end{align*}

\subsection*{(i)}
To show that \( f \) is continuous at the point \( (0, 0) \), we need to verify that every sequence \( (x_1^{(n)}, x_2^{(n)}) \) converging to \( (0, 0) \), \(f(x_1^{(n)}, x_2^{(n)})\)
also converges to \( f(0, 0) = (0, 0) \). \\
Fix a sequence \( (x_1^{(n)}, x_2^{(n)}) \) such that \( (x_1^{(n)}, x_2^{(n)}) \rightarrow (0, 0) \). \\
Then, we have: \\
\textbf{If \( (x_1^{(n)}, x_2^{(n)}) \neq (0, 0)\)}
\begin{align*}
   |f(x_1^{(n)}, x_2^{(n)}) - f(0, 0)| &= \left| \frac{(x_1^{(n)})^2 x_2^{(n)}}{(x_1^{(n)})^2 + (x_2^{(n)})^2} - 0 \right| \\
   &= \left| \frac{(x_1^{(n)})^2 x_2^{(n)}}{(x_1^{(n)})^2 + (x_2^{(n)})^2} \right| \\
   &\leq \left| \frac{(x_1^{(n)})^2 x_2^{(n)}}{(x_1^{(n)})^2} \right| \quad \text{(since } (x_1^{(n)})^2 + (x_2^{(n)})^2 \geq (x_1^{(n)})^2 \text{)} \\
   &= |x_2^{(n)}|
\end{align*}
\( |x_2^{(n)}| \to 0 \) since \( (x_1^{(n)}, x_2^{(n)}) \to (0, 0) \). Therefore \( f(x_1^{(n)}, x_2^{(n)}) \to (0, 0) \). \\
\\
\textbf{If \( (x_1^{(n)}, x_2^{(n)}) = (0, 0)\)}
\begin{align*}
   |f(x_1^{(n)}, x_2^{(n)}) - f(0, 0)| &= |0 - 0| = 0
\end{align*}
which also converges to \( 0 \). \\
Thats what we wanted to show.

\subsection*{(ii)}
We need to show that the following limit doesnt converge to zero
\begin{align*}
   \lim_{x \to 0} \frac{f(x) - f(0) - Jf(0)x}{\|x\|}
\end{align*}
The J is the Jacobian matrix. \\
\begin{align*}
   J = \begin{bmatrix}
      \frac{\partial f}{\partial x_1} & \frac{\partial f}{\partial x_2}
   \end{bmatrix}
\end{align*}

Lets compute the partials at (0, 0) \\
\begin{align*}
   \frac{\partial f}{\partial x_1}(0, 0) &= \lim_{h \to 0} \frac{f(h, 0) - f(0, 0)}{h} = \lim_{h \to 0} \frac{0 - 0}{h} = 0 \\
   \frac{\partial f}{\partial x_2}(0, 0) &= \lim_{h \to 0} \frac{f(0, h) - f(0, 0)}{h} = \lim_{h \to 0} \frac{0 - 0}{h} = 0
\end{align*}

Thus, the Jacobian matrix at (0, 0) is:
\begin{align*}
   Jf(0, 0) = \begin{bmatrix}
      0 & 0
   \end{bmatrix}
\end{align*}

Substituting this into our limit we get
\begin{align*}
   \lim_{x \to 0} \frac{f(x) - f(0)}{\|x\|}
\end{align*}

We also know that \( f(0) = 0 \), so we can simplify this to:
\begin{align*}
   \lim_{x \to 0} \frac{f(x)}{\|x\|}
\end{align*}
So it is enough if i can show a vector \(x\) where this limit is not zero. \\
Lets choose \( x = (x_1, x_1) \). Then we have:
\begin{align*}
   \lim_{x_1 \to 0} \frac{f(x_1, x_1)}{\|(x_1, x_1)\|} &= \lim_{x_1 \to 0} \frac{\frac{x_1^2 x_1}{x_1^2 + x_1^2}}{\sqrt{x_1^2 + x_1^2}} \\
   &= \lim_{x_1 \to 0} \frac{\frac{x_1^3}{2x_1^2}}{\sqrt{2} |x_1|} \\
   &= \lim_{x_1 \to 0} \frac{x_1}{2\sqrt{2} |x_1|} \\
   &= \lim_{x_1 \to 0} \frac{1}{2\sqrt{2}} \cdot \frac{x_1}{|x_1|}
\end{align*}
This limit doesnt even exist so it cant be equal to zero. \\
Thats what we wanted to show.

\end{document}