\documentclass{article}
\usepackage{amsmath, amsthm, amssymb}
\usepackage{tikz}
\usepackage{array}
\usepackage{mathtools}
\usepackage{graphicx}

\begin{document}
\section*{\huge Homework Sheet 2}
\begin{flushright}
   \textbf{Author: Abdullah Oğuz Topçuoğlu}
\end{flushright}

% Mathematics for Computer Scientists III
% Exercise Sheet 2
% Exercise 5 (4 Points)
% Let the function f : R2 → R be defined by
% f(x1, x2) := |x1| + |x2|.
% At which points (x0
% 1, x0
% 2) ∈ R2 is f partially differentiable? Determine the partial derivatives of f
% at the points (x0
% 1, x0
% 2) ∈ R2 at which f is partially differentiable.
% Exercise 6 (4 Points)
% Let D := {(x1, x2) ∈ R2 : x1 ∕= x2
% 2, x2 > 0} and f : D → R a function defined by
% f(x1, x2) := x1 ln(x2)
% (x1 − x2
% 2)x2
% .
% (i) Explain why D is open.
% (ii) Show that f is partially differentiable and determine the partial derivatives at any point
% (x0
% 1, x0
% 2) ∈ D.
% (iii) Determine the gradient of f at any point (x0
% 1, x0
% 2) ∈ D.
% Exercise 7 (4 Points)
% Let f : R2 → R be a function defined by
% f(x1, x2) :=
% 󰀫
% (x2
% 1 + x2
% 2)sin
% 󰀓 1
% x2
% 1+x2
% 2
% 󰀔
% , (x1, x2) ∕= (0, 0)
% 0 , (x1, x2) = (0, 0) .
% (i) Show that f is partially differentiable.
% (ii) Show that f is totally differentiable at the point (0, 0). Hint: Use 1.7.2 and 1.7.3.
% (iii) Show that f is not continuously differentiable at the point (0, 0).
% Exercise 8 (4 Points)
% Let f : R2 → R be a function defined by
% f(x1, x2) :=
% 󰀫 x1x3
% 2−x3
% 1x2
% x2
% 1+x2
% 2
% , (x1, x2) ∕= (0, 0)
% 0 , (x1, x2) = (0, 0) .
% (i) Show that f is twice continuously differentiable at any point x = (x1, x2) ∈ R2 \ {(0, 0)} and
% determine the Hessian matrix of f at x.
% (ii) Show that f is twice partially differentiable at point (0, 0) and determine the Hessian matrix
% of f at (0, 0). Why does the form of the Hessian matrix not contradict Schwarz’s theorem
% (Theorem 1.5.12)?

\section*{Problem 5}

We are given the function
\[
   f(x_1, x_2) := |x_1| + |x_2|.
\]

We are looking for points \(x^0 \in \mathbb{R}^2\) where \(\frac{\partial f}{\partial x^0_1}\) and \(\frac{\partial f}{\partial x^0_2}\) exist.
First things first that we realize the function is symmetric in both variables, so we can just focus on one of them and the other will follow the same logic. \\
Let's consider the partial derivative with respect to \(x^0_1\):
\begin{align*}
   \frac{\partial f}{\partial x^0_1} &= \lim_{h \to 0} \frac{f(x^0_1 + h, x^0_2) - f(x^0_1, x^0_2)}{h}. \\
   &= \lim_{h \to 0} \frac{|x^0_1 + h| + |x^0_2| - (|x^0_1| + |x^0_2|)}{h}. \\
   &= \lim_{h \to 0} \frac{|x^0_1 + h| - |x^0_1|}{h}.
\end{align*}
Now we have to consider different cases for \(x^0_1\):
\begin{itemize}
   \item If \(x^0_1 > 0\):
   \begin{align*}
      \frac{\partial f}{\partial x^0_1} &= \lim_{h \to 0} \frac{(x^0_1 + h) - x^0_1}{h} = \lim_{h \to 0} \frac{h}{h} = 1.
   \end{align*}
   \item If \(x^0_1 < 0\):
   \begin{align*}
      \frac{\partial f}{\partial x^0_1} &= \lim_{h \to 0} \frac{-(x^0_1 + h) + x^0_1}{h} = \lim_{h \to 0} \frac{-h}{h} = -1.
   \end{align*}
   \item If \(x^0_1 = 0\):
   \begin{align*}
      \frac{\partial f}{\partial x^0_1} &= \lim_{h \to 0^+} \frac{|h| - 0}{h} = \lim_{h \to 0^+} \frac{h}{h} = 1, \\
      \frac{\partial f}{\partial x^0_1} &= \lim_{h \to 0^-} \frac{|-h| - 0}{h} = \lim_{h \to 0^-} \frac{-h}{h} = -1.
   \end{align*}
   Since the left-hand limit and right-hand limit are not equal, the partial derivative does not exist at this point.
\end{itemize}

So partial derivates exist for all points where \(x^0_1 \neq 0\). By symmetry, the same applies for \(x^0_2\). \\
\textbf{Partial Derivaties:}
\begin{align*}
   \frac{\partial f}{\partial x^0_1} =
   \begin{cases}
      1, & x^0_1 > 0 \\
      -1, & x^0_1 < 0
   \end{cases}, \\
   \frac{\partial f}{\partial x^0_2} =
   \begin{cases}
      1, & x^0_2 > 0 \\
      -1, & x^0_2 < 0
   \end{cases}
\end{align*}

\section*{Problem 6}
We are given the set
\begin{align*}
   D := \{(x_1, x_2) \in \mathbb{R}^2 : x_1 \neq x_2^2, x_2 > 0\}
\end{align*}

and the function
\begin{align*}
   f : D \to \mathbb{R}, \quad f(x_1, x_2) := \frac{x_1 \ln(x_2)}{(x_1 - x_2^2)x_2}.
\end{align*}

\subsection*{(i)}
We need to show that for any point \(x \in D\) we can find an open ball around it. \\

\begin{figure}[h!]
  \centering
  \includegraphics[width=\linewidth]{Screenshot From 2025-10-22 12-07-40.png}
  \caption{The set D}
\end{figure}

So if we pick a poitn in \(D\) we can get the distance to \(x_1\) axis and distance to the parabola in the image and if we choose the radius of the open ball to be
the smaller of both then every point in that open ball will be in \(D\) because we are guaranteed to not touch the \(x_1\) axis and the parabola. \\
Thats why \(D\) is open.

\subsection*{(ii)}
We want to show that for any point in the set \(D\) the partial derivatives exist. \\
Lets start with \(\frac{\partial f}{\partial x_1}\): \\
Define \(f_{x_2}(x_1) = f(x_1, x_2)\). Now we want to show that \(\frac{df_{x_2}}{dx_1}\) exists. \\
And yes it does because \(f_{x_2}(x_1)\) is a rational function where the numerator and denominator are polynomials in \(x_1\) (with \(x_2\) being constant)
and demominator is never zero. \\
\\
Now we want to show that \(\frac{\partial f}{\partial x_2}\) exists: \\
Define \(f_{x_1}(x_2) = f(x_1, x_2)\). Now we want to show that \(\frac{df_{x_1}}{dx_2}\) exists. \\
And yes it does again because the numerator is differentiable and the denomainator is also differentiable and never zero. \\
\\
\textbf{Partial derivaties:}\\
\begin{align*}
   \frac{\partial f}{\partial x_1} &= \frac{df_{x_2}}{dx_1} \intertext{apply quotient rule}\\
   &= \frac{ln(x_2)((x_1 - x_2^2) x_2) - x_1 ln(x_2) x_2}{(x_1 - x_2^2)^2 x_2^2}, \\
   &= \frac{ln(x_2)x_2 (x_1 - x_2^2 - x_1)}{(x_1 - x_2^2)^2 x_2^2}, \\
   &= \frac{ln(x_2)x_2 (-x_2^2)}{(x_1 - x_2^2)^2 x_2^2}, \\
   &= \frac{-ln(x_2)x_2^3}{(x_1 - x_2^2)^2 x_2^2}, \\
   &= \frac{-ln(x_2)x_2}{(x_1 - x_2^2)^2}, \\
\end{align*}

\begin{align*}
   \frac{\partial f}{\partial x_2} &= \frac{df_{x_1}}{dx_2} \intertext{apply quotient rule}\\
   &= \frac{x_1/x_2 ((x_1 - x_2^2) x_2) - (x_1 (ln(x_2) (x_1 - 3x_2^2)))}{(x_1 - x_2^2)^2 x_2^2}\\
   &= \frac{x_1 (x_1 - x_2^2) - (x_1 ln(x_2) (x_1 - 3x_2^2))}{(x_1 - x_2^2)^2 x_2^2}\\
   &= \frac{x_1 (x_1 - x_2^2 - ln(x_2) (x_1 - 3x_2^2))}{(x_1 - x_2^2)^2 x_2^2}\\
\end{align*}
This is the furthest i could simplify :)

\subsection*{(iii)}
The fradient is simply the vector of partial derivatives which we just calculated in (ii):
\begin{align*}
   \nabla f(x_1, x_2) &= \begin{pmatrix}
                           \frac{\partial f}{\partial x_1} \\
                           \\
                           \frac{\partial f}{\partial x_2}
                        \end{pmatrix} \\
    &= \begin{pmatrix}
      \frac{-ln(x_2)x_2}{(x_1 - x_2^2)^2} \\
      \\
      \frac{x_1 (x_1 - x_2^2 - ln(x_2) (x_1 - 3x_2^2))}{(x_1 - x_2^2)^2 x_2^2}
   \end{pmatrix}
\end{align*}

\section*{Problem 7}
We are given the function
\begin{align*}
   f : \mathbb{R}^2 \to \mathbb{R}, \quad f(x_1, x_2) :=
   \begin{cases}
      (x_1^2 + x_2^2) \sin\left(\frac{1}{x_1^2 + x_2^2}\right), & (x_1, x_2) \neq (0, 0) \\
      0, & (x_1, x_2) = (0, 0)
   \end{cases}
\end{align*}

\subsection*{(i)}
We want to show that \(f\) is partially differentiable. \\
Lets start with \(\frac{\partial f}{\partial x_1}\):
\begin{align*}
   \frac{\partial f}{\partial x_1} &= \frac{df_{x_2}}{dx_1} \\
   \intertext{where}
   f_{x_2}(x_1) &= f(x_1, x_2) = (x_1^2 + x_2^2) \sin\left(\frac{1}{x_1^2 + x_2^2}\right)
\end{align*}
And \(f_{x_2}\) is differentiable for all \(x_1 \in \mathbb{R}\) because it is a product of differentiable functions. \\
\\
Now we want to show that \(\frac{\partial f}{\partial x_2}\):
\begin{align*}
   \frac{\partial f}{\partial x_2} &= \frac{df_{x_1}}{dx_2} \\
   \intertext{where}
   f_{x_1}(x_2) &= f(x_1, x_2) = (x_1^2 + x_2^2) \sin\left(\frac{1}{x_1^2 + x_2^2}\right)
\end{align*}
And the same thing as above applies here because of the symmetry of the function. \\
Now we need to check the point \((0,0)\):
\begin{align*}
   \frac{\partial f}{\partial x_1} &= \lim_{h \to 0} \frac{f(0 + h, 0) - f(0, 0)}{h} \\
   &= \lim_{h \to 0} \frac{h^2 \sin\left(\frac{1}{h^2}\right) - 0}{h} \\
   &= \lim_{h \to 0} h \sin\left(\frac{1}{h^2}\right) \\
   &= 0 \\
\end{align*}
\begin{align*}
   \frac{\partial f}{\partial x_2} &= \lim_{h \to 0} \frac{f(0, 0 + h) - f(0, 0)}{h} \\
   &= \lim_{h \to 0} \frac{h^2 \sin\left(\frac{1}{h^2}\right) - 0}{h} \\
   &= \lim_{h \to 0} h \sin\left(\frac{1}{h^2}\right) \\
   &= 0 \\
\end{align*}

So we have shown that \(f\) is partially differentiable everywhere.

\subsection*{(ii)}
We want to show that \(f\) is totally differentiable at point \((0,0)\). \\
So we are going to show that jacobian matrix exists at point \((0,0)\) and the limit condition holds. \\
Limit condition is:
\begin{align*}
   \lim_{(h_1, h_2) \to (0,0)} \frac{f(0 + h_1, 0 + h_2) - f(0,0) - J_f(0,0) \cdot (h_1, h_2)^T}{\|(h_1, h_2)\|} = 0
\end{align*}

Jacobian matrix is:
\begin{align*}
   J_f = \begin{pmatrix}
      \frac{\partial f}{\partial x_1} & \frac{\partial f}{\partial x_2}
   \end{pmatrix}
\end{align*}

we are interested in \(J_f(0,0)\):
\begin{align*}
   J_f(0,0) = \begin{pmatrix}
      0 & 0
   \end{pmatrix}
\end{align*}
Values we calculated in (i). \\
Now we can calculate the limit: \\
Case: \((h_1, h_2) \neq (0,0)\)
\begin{align*}
   &\lim_{(h_1, h_2) \to (0,0)} \frac{f(0 + h_1, 0 + h_2) - f(0,0) - J_f(0,0) \cdot (h_1, h_2)^T}{\|(h_1, h_2)\|} \\
   &= \lim_{(h_1, h_2) \to (0,0)} \frac{(h_1^2 + h_2^2) \sin\left(\frac{1}{h_1^2 + h_2^2}\right) - 0 - 0}{\sqrt{h_1^2 + h_2^2}} \\
   &= \lim_{(h_1, h_2) \to (0,0)} \sqrt{h_1^2 + h_2^2} \sin\left(\frac{1}{h_1^2 + h_2^2}\right) \\
   &= 0
\end{align*}

Case: \((h_1, h_2) = (0,0)\)
\begin{align*}
   &\lim_{(h_1, h_2) \to (0,0)} \frac{f(0 + h_1, 0 + h_2) - f(0,0) - J_f(0,0) \cdot (h_1, h_2)^T}{\|(h_1, h_2)\|} \\
   &= \lim_{(h_1, h_2) \to (0,0)} \frac{0 - 0 - 0}{0} \\
   &= 0
\end{align*}
So the limit condition holds and \(f\) is totally differentiable at point \((0,0)\). Thats what we wanted to show.

\subsection*{(iii)}
We want to show that \(f\) is not continuously differentiable at point \((0,0)\). \\
It suffices to find one entry in the jacobian matrix that is not continuous at point \((0,0)\). \\
In order to to do that I am gonna find a sequence that converges to \((0,0)\) but the value of the partial derivative at that sequence does not converge to the value of the partial derivative at \((0,0)\). \\
\\
Lets consider \(\frac{\partial f}{\partial x_1}\):\\
We first need to find a formula for the partial derivative:
\begin{align*}
   \frac{\partial f}{\partial x_1} &= \frac{df_{x_2}}{dx_1} \intertext{apply product rule}\\
   &= 2x_1 \sin\left(\frac{1}{x_1^2 + x_2^2}\right) + (x_1^2 + x_2^2) \cos\left(\frac{1}{x_1^2 + x_2^2}\right) \left(-\frac{1}{(x_1^2 + x_2^2)^2} 2x_1\right) \\
   &= 2x_1 \sin\left(\frac{1}{x_1^2 + x_2^2}\right) - \frac{2x_1}{x_1^2 + x_2^2} \cos\left(\frac{1}{x_1^2 + x_2^2}\right)
\end{align*}
Now we can consider the sequence \((x_1^{(n)}, x_2^{(n)}) = \left(\sqrt{\frac{1}{2\pi n}}, 0\right)\) which converges to \((0,0)\) as \(n\) approaches infinity. \\
Now we can calculate the value of the partial derivative at that sequence:
\begin{align*}
   \frac{\partial f}{\partial x_1}(x_1^{(n)}, x_2^{(n)}) &= 2\sqrt{\frac{1}{2\pi n}} \sin(2\pi n) - \frac{2\sqrt{\frac{1}{2\pi n}}}{\frac{1}{2\pi n}} \cos(2\pi n) \\
   &= 0 - 2\sqrt{2\pi n} \cdot 1 \\
   &= -2\sqrt{2\pi n}
\end{align*}
As \(n\) approaches infinity this value approaches \(-\infty\) which is not equal to \(\frac{\partial f}{\partial x_1}(0,0) = 0\). \\
So we have found a sequence that converges to \((0,0)\) but the value of the partial derivative at that sequence does not converge to the value of the partial derivative at \((0,0)\). \\
Thus \(\frac{\partial f}{\partial x_1}\) is not continuous at point \((0,0)\) and therefore \(f\) is not continuously differentiable at point \((0,0)\).

\section*{Problem 8}

We are given the function
\begin{align*}
   f : \mathbb{R}^2 \to \mathbb{R}, \quad f(x_1, x_2) :=
   \begin{cases}
      \frac{x_1 x_2^3 - x_1^3 x_2}{x_1^2 + x_2^2}, & (x_1, x_2) \neq (0, 0) \\
      0, & (x_1, x_2) = (0, 0)
   \end{cases}
\end{align*}

\subsection*{(i)}
We want to show that \(f\) is twice continuously differentiable at any point \(x = (x_1, x_2) \in \mathbb{R}^2 \setminus \{(0, 0)\}\) and determine the Hessian matrix of \(f\) at \(x\). \\
Since \(f\) is a rational function where the numerator and denominator are polynomials and the denominator is never zero for any point \(x \in \mathbb{R}^2 \setminus \{(0, 0)\}\)
we can take the derivate of a a polynomial function infinitely many times and the result will still be a polynomial function. And also the combination of polynomial functions (addition, subtraction, multiplication, division) are also polynomial functions. \\
Thus \(f\) is twice continuously differentiable at any point \(x = (x_1, x_2) \in \mathbb{R}^2 \setminus \{(0, 0)\}\). \\

\subsection*{(ii)}
We need to show that the PD condition holds for \(f\). \(PD_{(0,0) (i,j)}\) for \(i,j \in \{1,2\}\). \\
Lets start with \(PD_{(0,0) (1,1)}\):
\begin{itemize}
   \item f is partially differentiable at \((0,0)\) with respect to \(x_1\):\\
   \begin{align*}
      \frac{\partial f}{\partial x_1} &= \lim_{h \to 0} \frac{f(0 + h, 0) - f(0, 0)}{h} \\
      &= \lim_{h \to 0} \frac{0 - 0}{h} = 0
   \end{align*}

   \item \(\frac{\partial f}{\partial x_1}\) is partially differentiable at \((0,0)\) with respect to \(x_1\):\\
   \begin{align*}
      \frac{\partial f}{\partial x_1} &= \frac{df_{x_2}}{dx_1} \shortintertext{apply quotient rule} \\
      &= \frac{x_2^3 - 3x_1^2 x_2}{x_1^2 + x_2^2} - \frac{(x_1 x_2^3 - x_1^3 x_2) 2x_1}{(x_1^2 + x_2^2)^2} \\
   \end{align*}
   \begin{align*}
      \frac{\partial^2 f}{\partial x_1^2} &= \lim_{h \to 0} \frac{\frac{\partial f}{\partial x_1}(0 + h, 0) - \frac{\partial f}{\partial x_1}(0, 0)}{h} \\
      &= \lim_{h \to 0} \frac{0 - 0}{h} = 0
   \end{align*}
\end{itemize}
Now lets do \(PD_{(0,0) (1,2)}\):
\begin{itemize}
   \item f is partially differentiable at \((0,0)\) with respect to \(x_1\):\\
   (Already shown above)
   \item \(\frac{\partial f}{\partial x_1}\) is partially differentiable at \((0,0)\) with respect to \(x_2\):\\
   \begin{align*}
      \frac{\partial^2 f}{\partial x_2 \partial x_1} &= \lim_{h \to 0} \frac{\frac{\partial f}{\partial x_1}(0, 0 + h) - \frac{\partial f}{\partial x_1}(0, 0)}{h} \\
      &= \lim_{h \to 0} \frac{h - 0}{h} = 1
   \end{align*}
\end{itemize}
Now lets do \(PD_{(0,0) (2,1)}\):
\begin{itemize}
   \item f is partially differentiable at \((0,0)\) with respect to \(x_2\):\\
   \begin{align*}
      \frac{\partial f}{\partial x_2} &= \lim_{h \to 0} \frac{f(0, 0 + h) - f(0, 0)}{h} \\
      &= \lim_{h \to 0} \frac{0 - 0}{h} = 0
   \end{align*}
   \item \(\frac{\partial f}{\partial x_2}\) is partially differentiable at \((0,0)\) with respect to \(x_1\):\\
   \begin{align*}
      \frac{\partial f}{\partial x_2} &= \frac{df_{x_1}}{dx_2} \shortintertext{apply quotient rule} \\
      &= \frac{3x_1 x_2^2 - x_1^3}{x_1^2 + x_2^2} - \frac{(x_1 x_2^3 - x_1^3 x_2) 2x_2}{(x_1^2 + x_2^2)^2} \\
   \end{align*}
   \begin{align*}
      \frac{\partial^2 f}{\partial x_1 \partial x_2} &= \lim_{h \to 0} \frac{\frac{\partial f}{\partial x_2}(0 + h, 0) - \frac{\partial f}{\partial x_2}(0, 0)}{h} \\
      &= \lim_{h \to 0} \frac{-h - 0}{h} = -1
   \end{align*}
\end{itemize}
Now lets do \(PD_{(0,0) (2,2)}\):
\begin{itemize}
   \item f is partially differentiable at \((0,0)\) with respect to \(x_2\):\\
   (Already shown above)
   \item \(\frac{\partial f}{\partial x_2}\) is partially differentiable at \((0,0)\) with respect to \(x_2\):\\
   \begin{align*}
      \frac{\partial^2 f}{\partial x_2^2} &= \lim_{h \to 0} \frac{\frac{\partial f}{\partial x_2}(0, 0 + h) - \frac{\partial f}{\partial x_2}(0, 0)}{h} \\
      &= \lim_{h \to 0} \frac{0 - 0}{h} = 0
   \end{align*}
\end{itemize}

So we have shown that \(f\) is twice partially differentiable at point \((0,0)\) and the Hessian matrix at this point is:
\begin{align*}
   H_f(0,0) = \begin{pmatrix}
      \frac{\partial^2 f}{\partial x_1^2} & \frac{\partial^2 f}{\partial x_2 \partial x_1} \\
      \frac{\partial^2 f}{\partial x_1 \partial x_2} & \frac{\partial^2 f}{\partial x_2^2}
   \end{pmatrix} = \begin{pmatrix}
      0 & 1 \\
      -1 & 0
   \end{pmatrix}
\end{align*}
\\
Schwarz theorem reqiures the function to be twice continuously partially differentiable at a point but our function \(f\) is only twice partially differentiable.

\end{document}