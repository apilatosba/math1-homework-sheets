\documentclass{article}
\usepackage{amsmath, amsthm, amssymb}
\usepackage{tikz}
\usepackage{array}
\usepackage{mathtools}

\begin{document}
\section*{\huge Homework Sheet 1}
\begin{flushright}
   \textbf{Author: Abdullah Oğuz Topçuoğlu}
\end{flushright}

% Exercise 1 (3 Points)
% Let f : R2 → R be a function defined by
% f(x) := x2
% 1 + 2x2
% 2.
% Show that f is continuous by showing that the ε-δ criterion in Definition 1.3.1 is satisfied for every
% point (x0
% 1, x0
% 2) ∈ R2.
% Exercise 2 (4 Points)
% Let f : R2 → R be a function defined by
% f(x1, x2) :=
% 󰀫 x2
% 1 sin(x1+x2)
% √x4
% 1+x4
% 2
% , (x1, x2) ∕= (0, 0)
% 0 , (x1, x2) = (0, 0) .
% Show that f is continuous.
% Exercise 3 (5 Points)
% Let R+ := [0, ∞). Let f : R+ × R → R be a function defined by
% f(x1, x2) :=
% 󰀫 √x1 x2
% 2+x1x2
% 2
% x1+2x2
% 2
% , (x1, x2) ∕= (0, 0)
% 0 , (x1, x2) = (0, 0) .
% (i) Let x0 = (x0
% 1, x0
% 2) ∈ R2 be arbitrary, but fixed. Let the functions fx0
% 1 : R → R and fx0
% 2 : R+ → R
% be defined by fx0
% 2
% (x1) := f(x1, x0
% 2) and fx0
% 1
% (x2) := f(x0
% 1, x2). Show that both fx0
% 1 and fx0
% 2 are
% continuous.
% (ii) Show that f is not continuous at the point (0, 0).
% Exercise 4∗ (4 Bonus Points)
% Let f : R2 → R be a function defined by
% f(x1, x2) :=
% 󰀫 1
% x2
% 1+x2
% 2
% , (x1, x2) ∈ A
% x1x2 , (x1, x2) ∈/ A ,
% where A := 󰀋󰀃 1
% x cos( 2π
% x ), 1
% x sin( 2π
% x )
% 󰀄
% : x ∈ (0, ∞)
% 󰀌 ⊆ R2.
% (i) Plot (or sketch) the set A.
% (ii) Show that f is not continuous at the point (0, 0).
% (iii) Show that f(hN v) → f(0, 0) (as N → ∞) for every v ∈ R2 and every decreasing sequence
% (hN )N∈N in R with hN → 0.
% Hint for (iii): Consider how “often” the sequence (hN v)N∈N can hit the set A. To do this, look at
% the plot (or sketch) of the set A.

\section*{Problem 1}
We are given the function:
\[
   f: \mathbb{R}^2 \to \mathbb{R} \quad f(x_1,x_2)=x_1^2+2x_2^2
\]
We want to show that \(f\) is continuous on \(\mathbb{R}^2\). We want to find a delta that satisfies:
\[
   \forall \varepsilon>0, \exists \delta>0: \forall x \in \mathbb{R}^2 \|x-x^0\| < \delta \implies |f(x)-f(x^0)|<\varepsilon
\]
for an arbitrary but fixed \(x^0\). \\
Fix \(x^0=(x_1^0,x_2^0)\). Fix \(\varepsilon>0\). We want to find a delta that makes this inequality satisfied always:
\begin{align*}
   |f(x)-f(x^0)| &< \varepsilon \\
\end{align*}

We are gonna find a relation between delta and \(|f(x)-f(x^0)|\) and from there we are look for values of delta where it is
less than epsilon.

\begin{align*}
   |f(x)-f(x^0)| &= |x_1^2 + 2x_2^2 - (x_1^0)^2 - 2(x_2^0)^2| \\
   &= |(x_1^2 - (x_1^0)^2) + 2(x_2^2 - (x_2^0)^2)| \\
   &\leq |x_1^2 - (x_1^0)^2| + 2|x_2^2 - (x_2^0)^2| \\
   &= |(x_1 - x_1^0)(x_1 + x_1^0)| + 2|(x_2 - x_2^0)(x_2 + x_2^0)| \\
   &\leq |x_1 - x_1^0||x_1 + x_1^0| + 2|x_2 - x_2^0||x_2 + x_2^0|  \quad \shortintertext{(replace invidual components of the vector with the vector itself. that makes the overall value larger)} \\
   &\leq \|x - x^0\|\|x + x^0\| + 2\|x - x^0\|\|x + x^0\| \\
   &= 3\|x - x^0\|\|x + x^0\|\\
   &\leq 3 \delta \|x + x^0\| \\
\end{align*}

and now assume that \(\delta \leq 1\), then we can say that:
\begin{align*}
   \|x\| \leq \|x^0\| + 1
\end{align*}

\begin{align*}
   |f(x)-f(x^0)| &\leq 3 \delta \|x + x^0\| \\
   &\leq 3 \delta (\|x\| + \|x^0\|) \\
   &\leq 3 \delta (\|x^0\| + 1 + \|x^0\|) \\
   &= 3 \delta (2\|x^0\| + 1)
\end{align*}

We are almost done. Now if we chose a delta that satisfies:
\begin{align*}
   3 \delta (2\|x^0\| + 1) < \varepsilon \\
   \delta < \frac{\varepsilon}{3(2\|x^0\| + 1)}
\end{align*}

And at the top we assumed that \(\delta \leq 1\), so we can chose:
\[
   \delta = \min\left(1, \frac{\varepsilon}{3(2\|x^0\| + 1)}\right)
\]

So found a delta meaning that \(f\) is continuous at \(x^0\). Since \(x^0\) was arbitrary, \(f\) is continuous on \(\mathbb{R}^2\).


\end{document}