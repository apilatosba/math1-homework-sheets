\documentclass{article}
\usepackage{amsmath, amsthm, amssymb}
\usepackage{tikz}
\usepackage{array}
\usepackage{mathtools}
\usepackage{graphicx}
\usepackage{hyperref}

\begin{document}
\section*{\huge Homework Sheet 6}
\begin{flushright}
   \textbf{Author: Abdullah Oğuz Topçuoğlu}
\end{flushright}

% Exercise 21 (4 Points)
% Let the function f : R
% 2 → R
% 2 be defined by
% f(x1, x2) := 
% log(1 + x
% 2
% 1 + 2x
% 2
% 2
% )
% x1 e
% 2x2
% 
% .
% (i) Determine the set A of all points x ∈ R
% 2 at which the Jacobian matrix Jf(x) is invertible.
% (ii) Justify that f is locally invertible at each point x ∈ A.
% (iii) Determine for each x ∈ A the Jacobian matrix of the local inverse of f at f(x).
% Exercise 22 (4 Points)
% Justify the existence of the following R-integrals, and then calculate them:
% (i) R
% A
% 1
% x
% 3
% 1
% +
% 2
% x
% 2
% 2
% 
% d(x1, x2) for A := [1, 3] × [2, 4].
% (ii) R
% A
% cos(x1)x
% 3
% 2
% d(x1, x2) for A := [−
% pπ
% 2
% ,
% pπ
% 2
% ] × [−1, 1].
% Exercise 23 (4 Points)
% Let a, b ∈ R+ with 0 < a < b and set
% Aa,b := 
% (x1, x2) ∈ R
% 2
% : x2 ≤ 0, a2 ≤ x
% 2
% 1 + x
% 2
% 2 ≤ b
% 2
% .
% Justify the existence of the following R-integral, and then calculate it using the transformation
% theorem 1.15.15:
% Z
% Aa,b
% q
% x
% 2
% 1 + x
% 2
% 2
% d(x1, x2).
% Exercise 24 (4 Points)
% Let (c1, c2) ∈ R
% 2 and h, ϱ ∈ R++, and set
% A(c1,c2),ϱ := 
% (x1, x2) ∈ R
% 2
% : (x1 − c1)
% 2 + (x2 − c2)
% 2 ≤ ϱ
% .
% Justify the existence of the following R-integral, and then calculate it using the transformation
% theorem 1.15.15, taking into account the last part of Remark 1.15.17:
% Z
% A(c1,c2),ϱ
% h d(x1, x2).
% Here, the integrand is given by the function on R
% 2
% , which is equal to h everywhere

\section*{Exercise 21}
We are given the function
\begin{align*}
   f(x_1, x_2) := \begin{pmatrix}
      \log(1 + x_1^2 + 2x_2^2) \\
      x_1 e^{2x_2}
   \end{pmatrix}.
\end{align*}

\subsection*{(i)}
The jacobian matrix is invertible at the points where the determinant is not zero. \\
The jacobian matrix:
\begin{align*}
   J_f(x_1, x_2) = \begin{pmatrix}
      \frac{2x_1}{1 + x_1^2 + 2x_2^2} & \frac{4x_2}{1 + x_1^2 + 2x_2^2} \\
      \\
      e^{2x_2} & 2x_1 e^{2x_2}
   \end{pmatrix}.
\end{align*}
The determinant of the jacobian matrix:
\begin{align*}
   \text{det}(J_f(x_1, x_2)) &= \frac{2x_1 \cdot 2x_1 e^{2x_2}}{1 + x_1^2 + 2x_2^2} - \frac{4x_2 \cdot e^{2x_2}}{1 + x_1^2 + 2x_2^2} \\
   &= \frac{4x_1^2 e^{2x_2} - 4x_2 e^{2x_2}}{1 + x_1^2 + 2x_2^2} \\
   &= \frac{4e^{2x_2}(x_1^2 - x_2)}{1 + x_1^2 + 2x_2^2}.
\end{align*}
The determinant is zero when
\begin{align*}
   4e^{2x_2}(x_1^2 - x_2) = 0 \implies x_1^2 - x_2 = 0 \implies x_2 = x_1^2.
\end{align*}
The set \(A\) would be
\begin{align*}
   A = \{(x_1, x_2) \in \mathbb{R}^2 : x_2 \neq x_1^2\}.
\end{align*}

\subsection*{(ii)}
\(f\) is locally invertible at a point if the jacobian matrix is invertible at that point and we know that the jacobian matrix is invertible at each point in \(A\). \\
So \(f\) is locally invertible at each point \(x \in A\).

\subsection*{(iii)}
To find the jacobian matrix of the local inverse of \(f\) at \(f(x)\), we can use the formula:
\begin{align*}
   J_{f^{-1}}(f(x)) = (J_f(x))^{-1}.
\end{align*}
We already have \(J_f(x)\) so we need to find its inverse. \\
The jacobian matrix:
\begin{align*}
   J_f(x_1, x_2) = \begin{pmatrix}
      \frac{2x_1}{1 + x_1^2 + 2x_2^2} & \frac{4x_2}{1 + x_1^2 + 2x_2^2} \\
      \\
      e^{2x_2} & 2x_1 e^{2x_2}
   \end{pmatrix}.
\end{align*}
And its determinant:
\begin{align*}
   \text{det}(J_f(x_1, x_2)) = \frac{4e^{2x_2}(x_1^2 - x_2)}{1 + x_1^2 + 2x_2^2}.
\end{align*}
We calculated those in the previous parts. \\
The inverse of jacobian matrix:
\begin{align*}
   (J_f(x_1, x_2))^{-1} &= \frac{1}{\text{det}(J_f(x_1, x_2))} \begin{pmatrix}
      2x_1 e^{2x_2} & -\frac{4x_2}{1 + x_1^2 + 2x_2^2} \\
      \\
      -e^{2x_2} & \frac{2x_1}{1 + x_1^2 + 2x_2^2}
   \end{pmatrix} \\
   &= \frac{1 + x_1^2 + 2x_2^2}{4e^{2x_2}(x_1^2 - x_2)} \begin{pmatrix}
      2x_1 e^{2x_2} & -\frac{4x_2}{1 + x_1^2 + 2x_2^2} \\
      \\
      -e^{2x_2} & \frac{2x_1}{1 + x_1^2 + 2x_2^2}
   \end{pmatrix} \\
\end{align*}

\end{document}