\documentclass{article}
\usepackage{amsmath, amsthm, amssymb}
\usepackage{tikz}
\usepackage{array}
\usepackage{mathtools}
\usepackage{graphicx}
\usepackage{hyperref}

\begin{document}
\section*{\huge Homework Sheet 8}
\begin{flushright}
   \textbf{Author: Abdullah Oğuz Topçuoğlu}
\end{flushright}

% Exercise 29 (6 Points)
% A fair dice (with six faces) and a fair decahedron are rolled. The decahedron has ten (congruent)
% faces, which are numbered from 1 to 10. Let the random variable X specify the maximum of the
% number of pips (on the top face) of the dice and the number (on the top face) of the decahedron
% (after they have been rolled).
% (i) Find a suitable probability space (Ω, F, P) with which to model the underlying random experiment, define X as a mapping from Ω to N, and determine the range of X.
% (ii) Determine the distribution of X.
% (iii) Determine the probability that X realises a value between two and five, i.e. 2 ≤ X ≤ 5.
% Exercise 30 (4 Points)
% Given two urns, one with seven marbles numbered 1, . . . , 7, and another with four marbles numbered
% 1, . . . , 4. Two marbles are pulled, one from each urn. Let the random variables D7 and D4 specify
% the number written on the marbles, and let the random variables X and Y be defined as follows:
% X := max{|D7 − D4|, 0}, Y := min{|D7 − D4|, 5}.
% (i) Find a suitable probability space (Ω, F, P) with which to model the underlying random experiments, define D7 and D4 as mappings from Ω to N, and determine their ranges.
% (ii) Determine the ranges and the distributions of X and Y .
% Exercise 31 (2 Points)
% A wholesaler buys light bulbs from three manufacturers. It is known that 30% of the bulbs purchased are from manufacturer 1 and 50% of the bulbs purchased are from manufacturer 2. Despite
% the manufacturers’ quality controls, 5% of the bulbs from manufacturer 1, 2% of the bulbs from
% manufacturer 2 and 8% of the bulbs from manufacturer 3 are defective. What is the probability
% that a randomly selected bulb from the wholesaler is defective?
% Exercise 32 (4 Points)
% According to an expert from TU¨V Saarland, 8% of all inspected cars are not roadworthy due to
% serious defects and therefore do not receive a TU¨V sticker. It is also known that 70% of all inspected
% cars, that do not receive a TU¨V sticker, are older than ten years. Furthermore, 25% of all inspected
% cars are older than ten years and receive a TU¨V sticker. What is the probability that
% (i) an inspected car is older than ten years?
% (ii) an inspected car, that is older than ten years, does not receive a TU¨V sticker?
% (iii) an inspected car, that has received a TU¨V sticker, is older than ten years?

\section*{Exercise 29}

Let \(N_k\) denote the set of natural numbers from 1 to k.
\begin{align*}
   N_k = \{i \in \mathbb{N} \; \big| \; 0 < i \leq k\}
\end{align*}
So that \(N_k\) has k elements.

\subsection*{(i)}
The probability space would be
\begin{align*}
   \Omega &= N_6 \times N_{10} \\
   \mathcal{F} &= \text{Power set of } \Omega \\
   P &= U_\Omega
\end{align*}
I chose the uniform distribution because each outcome is equally likely when rolling fair dice. \\
The random variable \(X\) would be
\begin{align*}
   X: \Omega &\to \mathbb{N} \\
   (d, h) &\mapsto \max(d, h)
\end{align*}
The range of \(X\)
\begin{align*}
   X(\Omega) &= \{\max(d, h) \; | \; d \in N_6, h \in N_{10}\} \\
   &= N_{10}
\end{align*}

\subsection*{(ii)}
We want to calculate \(P[\{X = k\}]\) for \(k \in N_{10}\). \\
The random variable \(X\) takes the value \(k\) only in one of the following three cases:
\begin{itemize}
   \item The decahedron shows \(k\) and the dice shows a number less than \(k\).
   \item The dice shows \(k\) and the decahedron shows a number less than \(k\).
   \item Both the dice and the decahedron show \(k\).
\end{itemize}
Here note that the cases are disjoint. Lets count the number of possibilities for each case and add them up.
We need to consider those cases when k is less than or equal to 6 and when k is greater than 6 separately.
\begin{itemize}
   \item For \(k \in N_6\):
   \begin{itemize}
      \item Case 1: There are \(k - 1\) choices for the dice (1 to \(k - 1\)) and 1 choice for the decahedron (\(k\)). So there are \(k - 1\) possibilities.
      \item Case 2: There are \(k - 1\) choices for the decahedron (1 to \(k - 1\)) and 1 choice for the dice (\(k\)). So there are \(k - 1\) possibilities.
      \item Case 3: There is only 1 possibility where both show \(k\).
   \end{itemize}
   So in total there are \(2(k - 1) + 1 = 2k - 1\) possibilities.
   \item For \(k \in \{7, 8, 9, 10\}\):
   \begin{itemize}
      \item Case 1: There are 6 choices for the dice (1 to 6) and 1 choice for the decahedron (\(k\)). So there are 6 possibilities.
      \item Case 2: Cant happen since the dice only goes up to 6.
      \item Case 3: Cant happen since the dice only goes up to 6.
   \end{itemize}
   So in total there are \(6 + 0 + 0 = 6\) possibilities.
\end{itemize}

\begin{align*}
   P[\{X = k\}] &=
   \begin{cases}
      \frac{2k - 1}{60}, & k \in N_6 \\
      \frac{6}{60}, & k \in \{7, 8, 9, 10\}
   \end{cases}
   \\
   60 &= |\Omega| = |N_6| \times |N_{10}| = 6 \times 10
\end{align*}

\subsection*{(iii)}
We want to calculate \(P[\{2 \leq X \leq 5\}]\).
\begin{align*}
   P[\{2 \leq X \leq 5\}] &= \sum_{k=2}^{5} P[\{X = k\}] \\
   &= \sum_{k=2}^{5} \frac{2k - 1}{60} \\
   &= \frac{1}{60} \sum_{k=2}^{5} (2k - 1) \\
   &= \frac{1}{60} (3 + 5 + 7 + 9) \\
   &= \frac{24}{60} \\
   &= \frac{2}{5}
\end{align*}

\end{document}