\documentclass{article}
\usepackage{amsmath, amsthm, amssymb}
\usepackage{tikz}
\usepackage{array}
\usepackage{mathtools}
\usepackage{graphicx}

\begin{document}
\section*{\huge Homework Sheet 4}
\begin{flushright}
   \textbf{Author: Abdullah Oğuz Topçuoğlu}
\end{flushright}

% Exercise 13 (3 Points)
% Let f : R
% 2 → R be a function defined by
% f(x1, x2) := |x1 − x2| ,
% and consider the two points a := (0, 1) and b := (1, 2) in R
% 2
% .
% (i) Verify that f, a and b satisfy the assumptions of the mean value theorem (1.9.1) for a suitable
% choice of G.
% (ii) Determine θ ∈ (0, 1) such that, for ξ := a + θ(b − a), we have
% f(b) − f(a) = ⟨∇f(ξ), b − a⟩.
% (iii) Is it possible to find a θ as in part (ii) for the two points ea := (−1, −1) and eb := (1, 1)? Justify
% your answer.
% Exercise 14 (4 Points)
% Let f : R
% 2 → R be a function defined by
% f(x1, x2) := e
% 1+x1 + x1 sin(πx2).
% Determine the Taylor polynomial of order 1, and order 2 of f at x
% 0 = (x
% 0
% 1
% , x0
% 2
% ) := (0,
% 1
% 2
% ).
% Exercise 15 (5 Points)
% Let fi
% : R
% 2 → R, i = 1, 2, 3, be functions defined by
% f1(x1, x2) := −x
% 4
% 1 + x
% 2
% 2
% , f2(x1, x2) := −x
% 2
% 2 + cos(x1) and f3(x1, x2) := 1 − x
% 2
% 1
% .
% (i) Check whether (0, 0) is a local extremum point of f1.
% (ii) Show that (0, 0) is an isolated maximum point of f2.
% (iii) Show that (0, 0) is a maximum point of f3, but not an isolated maximum point.
% Exercise 16 (4 Points)
% Let f : R
% 2 → R be a function defined by
% f(x1, x2) := x
% 2
% 1 + 4x2
% e
% x
% 2
% 1+x
% 2
% 2
% .
% Assume that it has already been shown that f is a C
% 2
% function. Specify and classify all the stationary
% points of f, i.e. check for each such point whether it is a local (isolated) minimum point, a local
% (isolated) maximum point or a saddle point. Also check for each local extremum point of f whether
% it is a (“global”) extremum point.

\section*{Exercise 13}
We are given the function:
\begin{align*}
   f : \mathbb{R}^2 \to \mathbb{R}, \quad f(x_1, x_2) := |x_1 - x_2|
\end{align*}

and the points:
\begin{align*}
   a := (0, 1), \quad b := (1, 2)
\end{align*}

\subsection*{(i)}
Choose the \(G\):
\begin{align*}
   G := \{(x_1, x_2) \in \mathbb{R}^2 \mid x_1 - x_2 < 0\}
\end{align*}

It is obvious that \(a, b \in G\).\\
Now we need to show that the line segment connecting \(a\) and \(b\) lies in \(G\).\\
\begin{align*}
   \text{line segment} &= \{a + t(b - a) \mid t \in [0, 1]\} \\
   &= \{(0, 1) + t(1 - 0, 2 - 1) \mid t \in [0, 1]\} \\
   &= \{(t, 1 + t) \mid t \in [0, 1]\}
\end{align*}
which obviously is in \(G\).

\subsection*{(ii)}
When we consider the function \(f\) in the domain \(G\), \(f\) is equal to
\begin{align*}
   f(x_1, x_2) = x_2 - x_1
\end{align*}
Lets calculate the gradient of \(f\)
\begin{align*}
   \nabla f(x_1, x_2) = \begin{pmatrix}
      -1 \\
      1
   \end{pmatrix}
\end{align*}
Now we need to find \(\theta \in (0, 1)\) such that
\begin{align*}
   f(b) - f(a) = \langle \nabla f(\xi), b - a \rangle
\end{align*}
where \(\xi := a + \theta(b - a)\).\\
Calculating the left hand side:
\begin{align*}
   f(b) - f(a) &= f(1, 2) - f(0, 1) \\
   &= 1 - 1 = 0
\end{align*}
Calculating the right hand side:
\begin{align*}
   \langle \nabla f(\xi), b - a \rangle &= \langle \nabla f(a + \theta(b - a)), b - a \rangle \\
   &= \langle \nabla f((0, 1) + \theta(1, 1)), (1, 1) \rangle \\
      &= \langle \nabla f(\theta, 1 + \theta), (1, 1) \rangle \\
   &= \langle \begin{pmatrix}
      -1 \\
      1
   \end{pmatrix}, (1, 1) \rangle \\
   &= 0
\end{align*}
We get \(0 = 0\) which is true always and doesnt depend on what \(\theta\) is. So any \(\theta \in (0, 1)\) satisfies the equation.

\subsection*{(iii)}
We are given the points:
\begin{align*}
   \tilde{a} := (-1, -1), \quad \tilde{b} := (1, 1)
\end{align*}

No, we cant directly use the part (ii) here because simply the points \(\tilde{a}\) and \(\tilde{b}\) are not in \(G\) we chose. And actually there is no
\(G\) that contains a point in the diagonal line (where \(x_1 = x_2\)) because \(f\) is not differentiable on that line.

\section*{Exercise 14}
We are given the function
\begin{align*}
   f : \mathbb{R}^2 \to \mathbb{R}, \quad f(x_1, x_2) := e^{1 + x_1} + x_1 \sin(\pi x_2)
\end{align*}

Taylor polynomial of order 1 and 2 is given by
\begin{align*}
   T_1^f(x^0; y) &= f(x^0) + \nabla f(x^0) \cdot (y - x^0) \\
   T_2^f(x^0; y) &= T_1(x^0;y) + \frac{1}{2}(y - x^0)^T H_f(x^0)(y - x^0)
\end{align*}

First and second order ferivaties of \(f\) exist and given by
\begin{align*}
   \frac{\partial f}{\partial x_1} &= e^{1 + x_1} + \sin(\pi x_2) \\
   \frac{\partial f}{\partial x_2} &= \pi x_1 \cos(\pi x_2) \\
   \frac{\partial^2 f}{\partial x_1^2} &= e^{1 + x_1} \\
   \frac{\partial^2 f}{\partial x_2^2} &= -\pi^2 x_1 \sin(\pi x_2) \\
   \frac{\partial^2 f}{\partial x_2 \partial x_1} &= \frac{\partial^2 f}{\partial x_1 \partial x_2} = \pi \cos(\pi x_2) \\
\end{align*}

Since all the partial derivaites are addition and multiplication of other continuous functions, they are continous as well.
\\
\\
\(T_1^f(x^0; y)\) at \((0, \frac{1}{2})\):
\begin{align*}
   f(0, \tfrac{1}{2}) &= e^{1 + 0} + 0 \cdot \sin(\pi \cdot \tfrac{1}{2}) = e \\
   \nabla f(0, \tfrac{1}{2}) &= \begin{pmatrix}
      e^{1 + 0} + \sin(\pi \cdot \tfrac{1}{2}) \\
      \pi \cdot 0 \cdot \cos(\pi \cdot \tfrac{1}{2})
   \end{pmatrix} = \begin{pmatrix}
      e + 1 \\
      0
   \end{pmatrix} \\
   T_1^f((0, \tfrac{1}{2}); (y_1, y_2)) &= e + \begin{pmatrix}
      e + 1 \\
      0
   \end{pmatrix} \cdot \begin{pmatrix}
      y_1 - 0 \\
      y_2 - \tfrac{1}{2}
   \end{pmatrix} = e + (e + 1)y_1
\end{align*}
\\
\(T_2^f(x^0; y)\) at \((0, \frac{1}{2})\):
\begin{align*}
   H_f(0, \tfrac{1}{2}) &= \begin{pmatrix}
      e^{1 + 0} & \pi \cos(\pi \cdot \tfrac{1}{2}) \\
      \pi \cos(\pi \cdot \tfrac{1}{2}) & -\pi^2 \cdot 0 \cdot \sin(\pi \cdot \tfrac{1}{2})
   \end{pmatrix} = \begin{pmatrix}
      e & 0 \\
      0 & 0
   \end{pmatrix} \\
   T_2^f((0, \tfrac{1}{2}); (y_1, y_2)) &= T_1^f((0, \tfrac{1}{2}); (y_1, y_2)) + \frac{1}{2} \begin{pmatrix}
      y_1 - 0 & y_2 - \tfrac{1}{2}
   \end{pmatrix} \begin{pmatrix}
      e & 0 \\
      0 & 0
   \end{pmatrix} \begin{pmatrix}
      y_1 - 0 \\
      y_2 - \tfrac{1}{2}
   \end{pmatrix} \\
   &= e + (e + 1)y_1 + \frac{1}{2} e (y_1)^2
\end{align*}


\end{document}