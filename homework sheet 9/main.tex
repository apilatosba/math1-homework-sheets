\documentclass{article}
\usepackage{amsmath, amsthm, amssymb}
\usepackage{array}

\begin{document}
\section*{\huge Mathematics Homework Sheet 9}
\begin{flushright}
   \textbf{Author: Abdullah Oguz Topcuoglu \& Yousef Farag}
\end{flushright}

\section*{Problem 1}
\section*{Problem 1(a)}

\[
   f: [0,\infty) \rightarrow R, \quad x \mapsto \sqrt{x}
\]
We want to prove that f is continous on \((0,\infty)\). \\
So we want to show that \(\forall x_0 \in (0,\infty), \; \lim_{x \rightarrow x_0} f(x) = f(x_0)\) \\
So we want to show
\[
   \lim_{x \rightarrow x_0} \sqrt{x} = \sqrt{x_0}, \quad \forall x_0 \in (0,\infty)
\]
Let \(x_0 \in (0,\infty)\) be given. \\
Let \(\epsilon > 0\) be given. \\
We want to find \(\delta > 0\) such that
\[
   0 < |x - x_0| < \delta \implies |\sqrt{x} - \sqrt{x_0}| < \epsilon
\]
\begin{align*}
   |\sqrt{x} - \sqrt{x_0}| &= \frac{|x - x_0|}{|\sqrt{x} + \sqrt{x_0}|} < \epsilon \\
\end{align*}
And since \(|\sqrt{x} - \sqrt{x_0}| < \epsilon\) the minimum value of \(\sqrt{x}\) can be \(\sqrt{x_0} - \epsilon\).
Inserting this information into the above inequality we get
\begin{align*}
   \frac{|x - x_0|}{|\sqrt{x} + \sqrt{x_0}|} &< \frac{|x - x_0|}{|2 \sqrt{x_0} - \epsilon|} < \epsilon \\
   |x - x_0| &< |2 \sqrt{x_0} - \epsilon| \epsilon \\
\end{align*}
So, if I choose \(\delta = |2 \sqrt{x_0} - \epsilon| \epsilon\), then the condition is satisfied.

\section*{Problem 1(b)}
Yes, the function f is also continous at the point 0. Because the the proof at 1(a) is also valid for \(x_0 = 0\).
That is
\[
   \lim_{x \rightarrow 0} \sqrt{x} = \sqrt{0}
\]

\end{document}