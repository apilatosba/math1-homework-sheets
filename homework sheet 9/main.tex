\documentclass{article}
\usepackage{amsmath, amsthm, amssymb}
\usepackage{array}

\begin{document}
\section*{\huge Mathematics Homework Sheet 9}
\begin{flushright}
   \textbf{Author: Abdullah Oguz Topcuoglu \& Yousef Farag}
\end{flushright}

\section*{Problem 1}
\section*{Problem 1(a)}

\[
   f: [0,\infty) \rightarrow R, \quad x \mapsto \sqrt{x}
\]
We want to prove that f is continous on \((0,\infty)\). \\
So we want to show that \(\forall x_0 \in (0,\infty), \; \lim_{x \rightarrow x_0} f(x) = f(x_0)\) \\
So we want to show
\[
   \lim_{x \rightarrow x_0} \sqrt{x} = \sqrt{x_0}, \quad \forall x_0 \in (0,\infty)
\]
Let \(x_0 \in (0,\infty)\) be given. \\
Let \(\epsilon > 0\) be given. \\
We want to find \(\delta > 0\) such that
\[
   0 < |x - x_0| < \delta \implies |\sqrt{x} - \sqrt{x_0}| < \epsilon
\]
\begin{align*}
   |\sqrt{x} - \sqrt{x_0}| &= \frac{|x - x_0|}{|\sqrt{x} + \sqrt{x_0}|} < \epsilon \\
\end{align*}
And since \(|\sqrt{x} - \sqrt{x_0}| < \epsilon\) the minimum value of \(\sqrt{x}\) can be \(\sqrt{x_0} - \epsilon\).
Inserting this information into the above inequality we get
\begin{align*}
   \frac{|x - x_0|}{|\sqrt{x} + \sqrt{x_0}|} &< \frac{|x - x_0|}{|2 \sqrt{x_0} - \epsilon|} < \epsilon \\
   |x - x_0| &< |2 \sqrt{x_0} - \epsilon| \epsilon \\
\end{align*}
So, if I choose \(\delta = |2 \sqrt{x_0} - \epsilon| \epsilon\), then the condition is satisfied.

\section*{Problem 1(b)}
Yes, the function f is also continous at the point 0. Because the the proof at 1(a) is also valid for \(x_0 = 0\).
That is
\[
   \lim_{x \rightarrow 0} \sqrt{x} = \sqrt{0}
\]

\section*{Problem 2}
We are given that f is continous at the point 0 and f(3x) = f(x) for all x. \\
We want to show that f is constant on R. \\
Let's call the the limit and the value of the function at 0 as \(L\). \\
\[
   \lim_{x \rightarrow 0} f(x) = f(0) = L
\]
So from the definiton of the continuity at 0 we have
\[
   \forall \epsilon > 0, \; \exists \delta > 0, \; 0 < |x| < \delta \implies |f(x) - L| < \epsilon
\]
So let's focus on \(x_0 = \delta / 2\). From the continuity at 0 we know that \(|f(x_0) - L| < \epsilon\).
And from the definiton of f, the above inequality is satisfied for every \(x = 3^k x_0\), where \(k \in Z\).
This means that the above inequality is satisfied for every \(x = 3^k \delta / 2\). In this case I chose \(x_0\) to be \(\delta / 2\)
but in fact, \(x_0\) can be chosen to be any number in the interval \((0,\delta)\). And if we multiply this number by \(3^k\) we can get any real number.
This tells us that every real number in the codomain of f stays in the \(\epsilon\) neighborhood of L.
% NOTE(oguz, 16/12/25) we cant conclude the below statement directly from the above paragraph. needs more explanation but I am too lazy to do it.
So every real number in the codomain of f is equal to L. Therefore, f is constant on R.

\end{document}