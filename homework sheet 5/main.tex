\documentclass{article}
\usepackage{amsmath, amsthm, amssymb}
\usepackage{array}

\begin{document}
\section*{\huge Mathematics Homework Sheet 5}
\begin{flushright}
   \textbf{Author: Abdullah Oguz Topcuoglu \& Yousef Farag}
\end{flushright}

\section*{Problem 1}

\section*{Problem 1(b)}
\section*{Problem 1(b)(i)}
We want to prove \(\bigcap_{i \in I} U_i\) is closed.\\
We are given (\(\forall i \in I \quad U_i \subseteq R\)) is closed. \\
A set being closed means that its complement is open. So we want to prove that \(\bigcup_{i \in I} U_i^c\) is open.\\
Since each \(U_i\) is closed, we know that \(U_i^c\) is open.\\
And from the lecture we know that union or intersection of open sets is open. Thus \(\bigcup_{i \in I} U_i^c\) is open. Which means that \(\bigcap_{i \in I} U_i\) is closed.\\
And this completes the proof.\\

\section*{Problem 1(b)(ii)}
We want to prove \(\bigcup_{i = 1}^n U_i\) is closed.\\
We are given (\(U_1,...,U_n \subseteq R\)) are closed. \\
A set being closed means that its complement is open. So we want to prove that \(\bigcap_{i = 1}^n U_i^c\) is open.\\
Since each \(U_i\) is closed, we know that \(U_i^c\) is open.\\
And from the lecture we know that union or intersection of open sets is open. Thus \(\bigcap_{i = 1}^n U_i^c\) is open. Which means that \(\bigcup_{i = 1}^n U_i\) is closed.\\
And this completes the proof.\\

\section*{Problem 3}

\section*{Problem 3(a)}
\[
   a_n := (-1)^n
\]
\(a_n\) is not convergent. Because, for example, if we take \(\epsilon = 1/10\) then there is no N that satisfies
\[
   \forall n \geq N \quad |a_n - a| < 1/10
\]
\(a_n\) alternates between -1 and 1. So we can't find a value \(a\) that stays in the neighborhood of both -1 and 1.
For example, when \(\epsilon = 1/10\) there is no \(a \in R\) that satisfies
\[
   |1 - a| < 1/10 \quad \text{and} \quad |-1 - a| < 1/10
\]
No matter what you choose N to be you will always get for some \(j \in N\) \(a_j = 1\) and \(a_j = -1\). \\
Thus \(a_n\) is divergent.\\

\section*{Problem 3(b)}
\[
   b_n := \frac{(-1)^n}{n}
\]
\(b_n\) is convergent.\\
Because, we can find an N that satisfies
\[
   \forall \epsilon > 0 \quad \forall n \geq N \quad |b_n - 0| < \epsilon
\]
We are trying to find an N such that this inequality holds for any choice of \(\epsilon\).\\
\[
   |\frac{(-1)^n}{n}| < \epsilon \qquad \text{when} \; n \geq N
\]
When \(n\) is even, we have
\begin{align*}
   & |\frac{1}{n}| < \epsilon \qquad \text{when} \; n \geq N \\
   & \frac{1}{n} < \epsilon \qquad \text{when} \; n \geq N \\
   & n > \frac{1}{\epsilon} \qquad \text{when} \; n \geq N
\end{align*}
So if we choose N to be the any integer greater than \(\frac{1}{\epsilon}\) then the inequality holds for even \(n\). So, such N exist when \(n\) is even. \\
When \(n\) is odd, we have
\begin{align*}
   & |\frac{-1}{n}| < \epsilon \qquad \text{when} \; n \geq N \\
   & \frac{1}{n} < \epsilon \qquad \text{when} \; n \geq N \\
   & n > \frac{1}{\epsilon} \qquad \text{when} \; n \geq N
\end{align*}
Basically, we have the same thing for odd \(n\). So, such N exist when \(n\) is odd too. \\
Thus, we can find an N that satisfies the inequality for any choice of \(\epsilon\) if we choose \(a\) to be 0.\\
Since \(a = 0\), the limit is zero.\\
\[
   \lim_{n \to \infty} b_n = \frac{(-1)^n}{n} = 0
\]

\end{document}