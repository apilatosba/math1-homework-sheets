\documentclass{article}
\usepackage{amsmath, amsthm, amssymb}
\usepackage{tikz}
\usepackage{array}
\usepackage{graphicx} % Added for including the PDF file as an image
\usepackage{float}
\usepackage{mathpazo}
\usetikzlibrary{angles, quotes}

\begin{document}
\section*{\huge Mathematics Homework Sheet 5}
\begin{flushright}
   \textbf{Authors: Abdullah Oguz Topcuoglu \& Ahmed Waleed Ahmed Badawy Shora}
\end{flushright}

% 1. Prove that the following rules of arithmetic hold in a commutative ring (R, +, ·).
% 1. −(x + y) = (−x) + (−y),
% 2. −(x − y) = (−x) + y,
% 3. x · 0 = 0 · x = 0,
% 4. (−x) · y = −(x · y),
% 5. x · (−y) = −(x · y),
% 6. (−x) · (−y) = x · y,
% 7. x + y = z if and only if x = z − y.
% 2. Demonstrate the rule (x + y) + z = x + (y + z) for all vectors x, y, z in R
% 2 and R
% 3
% geometrically by depicting them as arrows.
% 3. Show that
% T =
% 
% 
% 
% 
% 
% 1
% i
% 1 + i
% 
%  ,
% 
% 
% 0
% 1
% i
% 
% 
% 
% 
% 
% is a linearly independent subset of C
% 3 and
% S =
% 
% 
% 
% 
% 
% i
% 0
% 0
% 
%  ,
% 
% 
% 0
% 1
% 0
% 
%  ,
% 
% 
% 0
% i
% 1
% 
% 
% 
% 
% 
% is a spanning set for C
% 3
% . Use the algorithm in the Steinitz exchange theorem to replace two
% elements of S with elements of T.
% 4. Let V be be a vector space and v1, v2, . . . , vn ∈ V . Prove the following assertions.
% (a) If ⟨v1, . . . , vn⟩ = V , then ⟨v1 − v2, v2 − v3, . . . , vn−1 − vn, vn⟩ = V .
% (b) If v1, v2, . . . , vn are linearly independent, then v1 − v2, v2 − v3, . . . , vn−1 − vn, vn are
% also linearly independent.
% 5. (a) Show that the set C
% 2
% is a complex vector space with respect to the vector addition
% and scalar multiplication defined by
% 
% x1
% x2
% 
% +
% 
% y1
% y2
% 
% := 
% x1 + y1 + 1
% x2 + y2 + 1
% , α 
% x1
% x2
% 
% := 
% αx1 + α − 1
% αx2 + α − 1
% 
% .
% Are the vectors 
% 0
% 2
% 
% und 
% 2
% 8
% 
% linearly independent in this vector space?
% (b) Let X be an arbitrary set. Show that the power set P(X) is a vector space over the
% trivial field {0, 1} with respect to the vector addition and scalar multiplication defined by
% Y1 + Y2 := Y1∆Y2
% and
% 0Y := ∅, 1Y := Y.
% [Note: The power set P(X) is the set of all subsets of X. The symmetric difference of two
% sets A and B is A∆B := (A ∪ B) \ (A ∩ B).]

\section*{Problem 1}
\begin{enumerate}
    \item $-(x + y) = (-x) + (-y)$: \\
    By the definition of the additive inverse, we have:
    \[
    -(x + y) + (x + y) = 0
    \]
   Use distributivity property:
   \begin{align*}
      (-x) + (-y) + (x + y) &= 0 \\
      ((-x) + (-y)) + (x + y) &= 0 \\
   \end{align*}
   So, since adding \(((-x) + (-y))\) to \((x + y)\) gives \(0\), we can conclude that it is the additive inverse of \((x + y)\).
   And that's what we are trying to prove.

    \item $-(x - y) = (-x) + y$: \\
    Apply the rule above.
    \begin{align*}
      -(x + (-y)) &= (-x) + (-(-y)) \\
      &= (-x) + y \\
    \end{align*}

    \item $x \cdot 0 = 0 \cdot x = 0$: \\
    \[
      x \cdot 0 + x\cdot 0 = x \cdot (0 + 0) = x \cdot 0
    \]
    \begin{align*}
      x \cdot 0 + x\cdot 0 &= x \cdot 0 \qquad \text{(add additive inverse of \(x\cdot 0\))} \\
      x \cdot 0 + (x\cdot 0 + -(x \cdot 0)) &= x \cdot 0 + -(x \cdot 0) \\
      x \cdot 0 + 0 &= 0 \\
      x \cdot 0 &= 0 \\
   \end{align*}

   And by commutativity we have \(0 \cdot x = 0\).

    \item $(-x) \cdot y = -(x \cdot y)$: \\
   %  (x⋅y)+((−x)⋅y)=(x+(−x))⋅y=0⋅y=0,
   \[
      (x \cdot y) + ((-x) \cdot y) = (x + (-x)) \cdot y = 0 \cdot y = 0
   \]
   So, \((-x) \cdot y\) is additive inverse of \((x \cdot y)\).

    \item $x \cdot (-y) = -(x \cdot y)$: \\
    \begin{align*}
      x \cdot (-y) &= x \cdot (-y) \qquad \text{(commutativity)} \\
      &= (-y) \cdot x  \qquad \text{(insert this into original equation)}\\
      (-y) \cdot x &= -(x.y) \qquad \text{(true by the previous rule)} \\
    \end{align*}

    \item $(-x) \cdot (-y) = x \cdot y$: \\
    Use rule (4) to get:
    \[
      (-x) \cdot (-y) = -(x \cdot (-y))
    \]
    Now use rule (5) to get:
    \[
      -(x \cdot (-y)) = -(-(x \cdot y))
    \]
      And by the definition of additive inverse, we have:
      \[
         -(-(x \cdot y)) = x \cdot y
      \]
    \item $x + y = z$ if and only if $x = z - y$: \\
    By the definition of addition, we have:
    \[
    x + y = z \implies x = z - y
    \]
    and
    \[
    x = z - y \implies x + y = z
    \]
\end{enumerate}

\section*{Problem 3}

We want to show that only solution to the equation
\[
   a_1 \begin{pmatrix}
      1 \\
      i \\
      1 + i
   \end{pmatrix} + a_2 \begin{pmatrix}
      0 \\
      1 \\
      i
   \end{pmatrix} = 0
\]
is \(a_1 = 0\) and \(a_2 = 0\) where \(a_1, a_2 \in \mathbb{C}\).
We can write this as a system of equations:
\begin{align*}
   a_1 + 0 \cdot a_2 &= 0 \\
   a_1 i + a_2 &= 0 \\
   a_1 (1 + i) + a_2 i &= 0
\end{align*}
The first equation gives us \(a_1 = 0\). Substituting this into the second equation gives us \(a_2 = 0\). Thus, the only solution is \(a_1 = 0\) and \(a_2 = 0\).

Now we want to show that the set
\[
   S = \left\{
      \begin{pmatrix}
         i \\
         0 \\
         0
      \end{pmatrix},
      \begin{pmatrix}
         0 \\
         1 \\
         0
      \end{pmatrix},
      \begin{pmatrix}
         0 \\
         i \\
         1
      \end{pmatrix}
   \right\}
\]
is a spanning set for \(\mathbb{C}^3\). This means that any vector in \(\mathbb{C}^3\) can be written as a linear combination of the vectors in \(S\). We can write this as:
\[
   \begin{pmatrix}
      x_1 \\
      x_2 \\
      x_3
   \end{pmatrix} = a_1 \begin{pmatrix}
      i \\
      0 \\
      0
   \end{pmatrix} + a_2 \begin{pmatrix}
      0 \\
      1 \\
      0
   \end{pmatrix} + a_3 \begin{pmatrix}
      0 \\
      i \\
      1
   \end{pmatrix}
\]
This gives us the system of equations:
\begin{align*}
   x_1 &= a_1 i \\
   x_2 &= a_2 + a_3 i \\
   x_3 &= a_3
\end{align*}
We can solve this system of equations for \(a_1, a_2, a_3\) in terms of \(x_1, x_2, x_3\):
\begin{align*}
   a_1 &= \frac{x_1}{i} \\
   a_3 &= x_3 \\
   a_2 &= x_2 - a_3 i = x_2 - x_3 i
\end{align*}
Thus, any vector in \(\mathbb{C}^3\) can be written as a linear combination of the vectors in \(S\), so \(S\) is a spanning set for \(\mathbb{C}^3\).

\[
   S = \left\{
      \begin{pmatrix}
         i \\
         0 \\
         0
      \end{pmatrix},
      \begin{pmatrix}
         0 \\
         1 \\
         0
      \end{pmatrix},
      \begin{pmatrix}
         0 \\
         i \\
         1
      \end{pmatrix}
   \right\}
\]

Create a ne set by adding a vector from \(T\) to \(S\):
\[
   S_1 = \left\{
      \begin{pmatrix}
         i \\
         0 \\
         0
      \end{pmatrix},
      \begin{pmatrix}
         0 \\
         1 \\
         0
      \end{pmatrix},
      \begin{pmatrix}
         0 \\
         i \\
         1
      \end{pmatrix},
      \begin{pmatrix}
         1 \\
         i \\
         1 + i
      \end{pmatrix}
   \right\}
\]

We can write the vector \(
   \begin{pmatrix}
      i \\
      0 \\
      0
   \end{pmatrix},
\) as a linear combination of the vectors in \(S_1\):
\[
   \begin{pmatrix}
      i \\
      0 \\
      0
   \end{pmatrix} = a_1 \begin{pmatrix}
      0 \\
      1 \\
      0
   \end{pmatrix} + a_2 \begin{pmatrix}
      0 \\
      i \\
      1
   \end{pmatrix} + a_3 \begin{pmatrix}
      1 \\
      i \\
      1 + i
   \end{pmatrix}
\]
With the coeeficients:
\begin{align*}
   a_1 &= -i \\
   a_2 &= 1-i \\
   a_3 &= i
\end{align*}
Thus we can get rid of the vector \(
   \begin{pmatrix}
      i \\
      0 \\
      0
   \end{pmatrix}
\) and replace it with the vector \(
   \begin{pmatrix}
      1 \\
      i \\
      1 + i
   \end{pmatrix}
\) from \(T\).
\[
   S_2 = \left\{
      \begin{pmatrix}
         1 \\
         i \\
         1 + i
      \end{pmatrix},
      \begin{pmatrix}
         0 \\
         1 \\
         0
      \end{pmatrix},
      \begin{pmatrix}
         0 \\
         i \\
         1
      \end{pmatrix}
   \right\}
\]
Let's add the other vector from \(T\):
\[
   S_3 = \left\{
      \begin{pmatrix}
         1 \\
         i \\
         1 + i
      \end{pmatrix},
      \begin{pmatrix}
         0 \\
         1 \\
         0
      \end{pmatrix},
      \begin{pmatrix}
         0 \\
         i \\
         1
      \end{pmatrix},
      \begin{pmatrix}
         0 \\
         1 \\
         i
      \end{pmatrix}
   \right\}
\]
We can write the vector \(
   \begin{pmatrix}
      0 \\
      1 \\
      0
   \end{pmatrix},
\) as a linear combination of the vectors in \(S_3\):
\[
   \begin{pmatrix}
      0 \\
      1 \\
      0
   \end{pmatrix} = a_1 \begin{pmatrix}
      1 \\
      i \\
      1 + i
   \end{pmatrix} + a_2 \begin{pmatrix}
      0 \\
      i \\
      1
   \end{pmatrix} + a_3 \begin{pmatrix}
      0 \\
      1 \\
      i
   \end{pmatrix}
\]
With the coeeficients:
\begin{align*}
   a_1 &= 0 \\
   a_2 &= -i/2 \\
   a_3 &= 1/2
\end{align*}
Thus we can get rid of the vector \(
   \begin{pmatrix}
      0 \\
      1 \\
      0
   \end{pmatrix}
\) and replace it with the vector \(
   \begin{pmatrix}
      0 \\
      1 \\
      i
   \end{pmatrix}
\) from \(T\).
\[
   S_4 = \left\{
      \begin{pmatrix}
         1 \\
         i \\
         1 + i
      \end{pmatrix},
      \begin{pmatrix}
         0 \\
         1 \\
         i
      \end{pmatrix},
      \begin{pmatrix}
         0 \\
         i \\
         1
      \end{pmatrix}
   \right\}
\]
\(S_4\) still spans the same space as \(S\).

\section*{Problem 4}
\subsection*{(a)}
We can use Steinitz theorem recursively to show that \(\langle v_1, ..., v_n\rangle = \langle v_1-v_2, ..., v_n \rangle\).
We can replace \(v_1\) with \(v_1 - v_2\) and get:
\[
   \langle v_1 - v_2, v_2 - v_3, ..., v_n \rangle = \langle v_1, v_2 -v_3, ..., v_n \rangle
\]
by choosing every coefficient to be 1. \\
We can repeat this for \(v_2\). Replace \(v_2 - v_3\) with \(v_2\) to get:
\[
   \langle v_1, v_2 -v_3, ..., v_n \rangle = \langle v_1, v_2, ..., v_n \rangle
\]
Again by choosing every coefficient to be 1. \\
We can repeat this for every \(v_i\). And at the end we will have:
\[
   \langle v_1 - v_2, v_2 - v_3, ..., v_n \rangle = \langle v_1, v_2, ..., v_n \rangle
\]


\end{document}