\documentclass{article}
\usepackage{amsmath, amsthm, amssymb}
\usepackage{tikz}
\usepackage{array}
\usepackage{graphicx} % Added for including the PDF file as an image
\usepackage{float}
\usepackage{mathpazo}
\usetikzlibrary{angles, quotes}

\begin{document}
\section*{\huge Mathematics Homework Sheet 10}
\begin{flushright}
   \textbf{Authors: Abdullah Oguz Topcuoglu \& Ahmed Waleed Ahmed Badawy Shora}
\end{flushright}

% 1. Compute the eigenvalues and eigenspaces of the complex matrices
% 
% 
% 1 0 −1
% 1 2 1
% 2 2 3
% 
%  ,
% 
% 
% 2 i 1 + 2i
% −i 0 −i
% 1 − 2i i 0
% 
%  ,
% 
% 
% 1 1 1 0
% 1 1 0 1
% 1 0 1 1
% 0 1 1 1
% 
%  .
% 2. Let
% A =
% 
% −5 3
% 6 −2
% 
% .
% (a) Find a real, invertible matrix P such that P
% −1AP is diagonal.
% (b) Find a real, invertible matrix B such that B3 = A.
% 3. Let K be a field. The trace of a matrix in Kn×n
% is the sum of its diagonal entries.
% (a) Prove that tr AB = tr BA for all A, B ∈ Kn×n
% , and that two similar matrices in Kn×n
% have the same trace. How would you define the trace of a linear transformation T : V → V
% for an n-dimensional vector space V over K?
% (b) Let A ∈ Kn×n and c be its characterisic polynomial. Show that
% (i) the coefficient of λ
% n
% in c is (−1)n
% ,
% (ii) the coefficient of λ
% n−1
% in c is (−1)n−1
% tr A,
% (iii) the coefficient of λ
% 0
% in c is det A.
% [Hint: Study the proof that c is a polynomial of degree n.]
% 4. Let a0, . . . , an−1 be distinct real numbers. The matrix
% V (a0, . . . , an−1) =
% 
% 
% 1 a0 a
% 2
% 0
% · · · a
% n−1
% 0
% 1 a1 a
% 2
% 1
% · · · a
% n−1
% 1
% .
% .
% .
% .
% .
% .
% .
% .
% .
% .
% .
% .
% 1 an−2 a
% 2
% n−2
% · · · a
% n−1
% n−2
% 1 an−1 a
% 2
% n−1
% · · · a
% n−1
% n−1
% 
% 
% is called a Vandemonde matrix.
% Show that
% det V (a0, . . . , an−1) = (a1 − a0)(a2 − a0). . .(an−1 − a0) det V (a1, . . . , an−1)
% and deduce that
% det V (a0, . . . , an−1) = Y
% 0≤i<j≤n−1
% (aj − ai), n = 1, 2, . . . .
% 5. (a) Define norms ∥ · ∥1, ∥ · ∥2 and ∥ · ∥∞ on R
% 2 by the formulae
% ∥x∥1 = |x1| + |x2|, ∥x∥2 = (|x1|
% 2 + |x2|
% 2
% )
% 1
% 2 , ∥x∥∞ = max{|x1|, |x2|}.
% Draw a diagram of the set {x : ∥x∥ < 1} in each of these cases.
% (b) Let x and y be non-zero vectors in a normed space. Show that
% ∥x + y∥
% 2 + ∥x − y∥
% 2
% ∥x∥
% 2 + ∥y∥
% 2
% ≤ 4

\section*{Problem 1}

Lets calculate the characteristic function of each matrix.
\begin{align*}
\text{For } A_1 = \begin{pmatrix}
1 & 0 & -1 \\
1 & 2 & 1 \\
2 & 2 & 3
\end{pmatrix}, \quad \text{the characteristic polynomial is } \\
c(\lambda) = \det(A_1 - \lambda I) = \det\begin{pmatrix}
1 - \lambda & 0 & -1 \\
1 & 2 - \lambda & 1 \\
2 & 2 & 3 - \lambda
\end{pmatrix} \\
= (1 - \lambda)\det\begin{pmatrix}
2 - \lambda & 1 \\
2 & 3 - \lambda
\end{pmatrix} - 0 + (-1)\det\begin{pmatrix}
1 & 2 - \lambda \\
2 & 2
\end{pmatrix} \\
= (1 - \lambda)((2 - \lambda)(3 - \lambda) - 2) - (1(2) - 2(2 - \lambda)) \\
= (1 - \lambda)(\lambda^2 - 5\lambda + 4) - (2 - 4 + 2\lambda) \\
= (1 - \lambda)(\lambda^2 - 5\lambda + 4) - (2\lambda - 2) \\
= (1 - \lambda)(\lambda - 4)(\lambda - 1) - 2(\lambda - 1) \\
= (\lambda - 1)(-\lambda^2 + 5\lambda - 4 - 2) \\
= (1 - \lambda)(\lambda^2 - 5\lambda + 6) \\
= (1 - \lambda)(\lambda - 2)(\lambda - 3)
\end{align*}

The eigenvalues are $\lambda_1 = 1$, $\lambda_2 = 2$, and $\lambda_3 = 3$.
The eigenspaces are:
\begin{align*}
E_1 &= \ker\begin{pmatrix}
1 - 1 & 0 & -1 \\
1 & 2 - 1 & 1 \\
2 & 2 & 3 - 1
\end{pmatrix} \\
&= \ker\begin{pmatrix}
0 & 0 & -1 \\
1 & 1 & 1 \\
2 & 2 & 2
\end{pmatrix} \\
&= \ker\begin{pmatrix}
1 & 1 & 1 \\
0 & 0 & -1 \\
0 & 0 & 0
\end{pmatrix} \\
&= \text{span}\left\{\begin{pmatrix}
1 \\
-1 \\
0
\end{pmatrix}\right\}
\end{align*}

\begin{align*}
E_2 &= \ker\begin{pmatrix}
1 - 2 & 0 & -1 \\
1 & 2 - 2 & 1 \\
2 & 2 & 3 - 2
\end{pmatrix} \\
&= \ker\begin{pmatrix}
-1 & 0 & -1 \\
1 & 0 & 1 \\
2 & 2 & 1
\end{pmatrix} \\
&= \ker\begin{pmatrix}
1 & 0 & 1 \\
0 & 2 & -1 \\
0 & 0 & 0
\end{pmatrix} \\
&= \text{span}\left\{\begin{pmatrix}
-1 \\
1/2 \\
1
\end{pmatrix}\right\}
\end{align*}

\begin{align*}
    E_3 &= \ker\begin{pmatrix}
1 - 3 & 0 & -1 \\
1 & 2 - 3 & 1 \\
2 & 2 & 3 - 3
\end{pmatrix} \\
&= \ker\begin{pmatrix}
-2 & 0 & -1 \\
1 & -1 & 1 \\
2 & 2 & 0
\end{pmatrix} \\
&= \ker\begin{pmatrix}
1 & -1 & 1 \\
0 & 2 & -1 \\
0 & 0 & 0
\end{pmatrix} \\
&= \text{span}\left\{\begin{pmatrix}
-1/2 \\
1/2 \\
1
\end{pmatrix}\right\}
\end{align*}

\end{document}