\documentclass{article}
\usepackage{amsmath, amsthm, amssymb}
\usepackage{tikz}
\usepackage{array}
\usepackage{graphicx} % Added for including the PDF file as an image
\usepackage{float}
\usepackage{mathpazo}
\usetikzlibrary{angles, quotes}

\begin{document}
\section*{\huge Mathematics Homework Sheet 6}
\begin{flushright}
   \textbf{Authors: Abdullah Oguz Topcuoglu \& Ahmed Waleed Ahmed Badawy Shora}
\end{flushright}

% 1. Show that (R, +) forms a finite-dimensional vector space over (R, +, .) but an infinitedimensional vector space over (Q, +, .)
% 2. (a) Which of the subsets
% U1 = {p ∈ R[x] : p(0) = 0} ,
% U2 = {p ∈ R[x] : p(0) = 1} ,
% U3 = {p ∈ R[x] : p(1) = 0} ,
% U4 =
% 
% p ∈ R[x] : Z 1
% 0
% p(x) dx = 0
% ,
% U5 = {p ∈ R[x] : p
% 0
% (0) + p
% 00(0) = 0} ,
% U6 = {p ∈ R[x] : p
% 0
% (0)p
% 00(0) = 0}
% of R[x] are subspaces of R[x]?
% (b) Which of the subsets
% S1 =
%  a b
% c d 
% : a = b
% 
% , S2 =
%  a b
% c d 
% : a + b = 1
% , S3 =
%  a b
% c d 
% : a
% 2 = b
% 2
% 
% of R
% 2×2 are subspaces of R
% 2×2?
% 3. Show that {x
% 3 − x
% 2
% , x3 − x} is a basis for the subspace
% W = {p ∈ R3[x] : p(0) = p(1) = 0}
% of R3[x]. Extend this basis to a basis for R3[x] and hence find a complement of W in R3[x].
% 4. Let S3 be the set of all real, symmetric 3 × 3 matrices. Find a basis for the subspace S3
% of R
% 3×3 and hence determine dim S3. Extend this basis to a basis for R
% 3×3 and hence find
% a complement of S3 in R
% 3×3
% .
% [Note: An n×n matrix A = (aij )i,j=1,...,n is called symmetric if aij = aji for all i, j = 1, . . . n.]
% 5. (a) Let U1, U2 be subspaces of a vector space V . Show that U1 ∪ U2 is a subspace of V
% if and only if U1 ⊆ U2 or U2 ⊆ U1.
% (b) Let U1, U2 be subspaces of a vector space V . Show that U1 + U2 is a subspace of V .

\section*{Problem 2}

\subsection*{(a)}

\(U_1\): \\
\(U_1\) is subspace of \(R[x]\). \\
\textbf{Not empty:} \\
\(U_1\) is not empty because \(0 \in U_1\) (zero polynomial). \\
\textbf{Closed under addition:} \\
Let \(p(x), q(x) \in U_1\). Then \(p(0) = 0\) and \(q(0) = 0\). \\
Then, \((p + q)(0) = p(0) + q(0) = 0 + 0 = 0\). \\
\textbf{Closed under scalar multiplication:} \\
Let \(p(x) \in U_1\) and \(c \in R\). Then, \((cp)(0) = c(p(0)) = c(0) = 0\). \\
Thus, \(U_1\) is closed under scalar multiplication. \\
\\
\(U_2\): \\
\(U_2\) is not a subspace of \(R[x]\). Because \(U_2\) doesn't contain the zero polynomial. (every vector space has to contain the zero vector which is the zero polynomial in this case) \\
\\
\(U_3\): \\
\(U_3\) is subspace of \(R[x]\). \\
\textbf{Not empty:} \\
\(U_3\) is not empty because \(0 \in U_3\) (zero polynomial). \\
\textbf{Closed under addition:} \\
Let \(p(x), q(x) \in U_3\). Then \(p(1) = 0\) and \(q(1) = 0\). \\
Then, \((p + q)(1) = p(1) + q(1) = 0 + 0 = 0\). \\
\textbf{Closed under scalar multiplication:} \\
Let \(p(x) \in U_3\) and \(c \in R\). Then, \((cp)(1) = c(p(1)) = c(0) = 0\). \\
Thus, \(U_3\) is closed under scalar multiplication. \\
\\
\(U_4\): \\
\(U_4\) is subspace of \(R[x]\). \\
\textbf{Not empty:} \\
\(U_4\) is not empty because \(0 \in U_4\) (zero polynomial). \\
\textbf{Closed under addition:} \\
Let \(p(x), q(x) \in U_4\). Then \(\int_0^1 p(x) dx = 0\) and \(\int_0^1 q(x) dx = 0\). \\
Then, \(\int_0^1 (p + q)(x) dx = \int_0^1 p(x) dx + \int_0^1 q(x) dx = 0 + 0 = 0\). \\
\textbf{Closed under scalar multiplication:} \\
Let \(p(x) \in U_4\) and \(c \in R\). Then, \(\int_0^1 (cp)(x) dx = c\int_0^1 p(x) dx = c(0) = 0\). \\
Thus, \(U_4\) is closed under scalar multiplication. \\
\\
\(U_5\): \\
\(U_5\) is subspace of \(R[x]\). \\
\textbf{Not empty:} \\
\(U_5\) is not empty because \(0 \in U_5\) (zero polynomial). \\
\textbf{Closed under addition:} \\
Let \(p(x), q(x) \in U_5\). Then \(p'(0) + p''(0) = 0\) and \(q'(0) + q''(0) = 0\). \\
Then, \((p + q)'(0) + (p + q)''(0) = p'(0) + q'(0) + p''(0) + q''(0) = 0 + 0 = 0\). \\
\textbf{Closed under scalar multiplication:} \\
Let \(p(x) \in U_5\) and \(c \in R\). Then, \((cp)'(0) + (cp)''(0) = c(p'(0)) + c(p''(0)) = c(p'(0) + p''(0)) = c(0) = 0\). \\
Thus, \(U_5\) is closed under scalar multiplication. \\
\\
\(U_6\): \\
\(U_6\) is not a subspace of \(R[x]\). Becuase it is not closed under addition \\
Let \(p(x), q(x) \in U_6\). Then \(p'(0)p''(0) = 0\) and \(q'(0)q''(0) = 0\). \\
Then, \((p + q)'(0)(p + q)''(0) = (p'(0) + q'(0)) (p''(0) + q''(0)) = p'(0)p''(0) + p'(0)q''(0) + q'(0)p''(0) + q'(0)q''(0) = p'(0)q''(0) + q'(0)p''(0)\) \\
Which is not necessarily equal to 0. Thus \(U_6\) is not closed under addition. \\

\subsection*{(b)}
\(S_1\): \\
\(S_1\) is a subspace of \(R^{2 \times 2}\). \\
\textbf{Not empty:} \\
\(S_1\) is not empty because \(0 \in S_1\) (2 by 2 zero matrix). \\
\textbf{Closed under addition:} \\
Let \(A, B \in S_1\). Then \(A = \begin{pmatrix} a & b \\ c & d \end{pmatrix}\) and \(B = \begin{pmatrix} e & f \\ g & h \end{pmatrix}\) where \(a = b\) and \(e = f\). \\
Then, \(A + B = \begin{pmatrix} a + e & b + f \\ c + g & d + h \end{pmatrix}\) where \(a + e = b + f\). \\
\textbf{Closed under scalar multiplication:} \\
Let \(A \in S_1\) and \(c \in R\). Then, \(A = \begin{pmatrix} a & b \\ c & d \end{pmatrix}\) where \(a = b\). \\
Then, \(cA = \begin{pmatrix} ca & cb \\ cc & cd \end{pmatrix}\) where \(ca = cb\). \\
\\
\(S_2\): \\
\(S_2\) is not a subspace of \(R^{2 \times 2}\). Because \(S_2\) doesn't contain the zero matrix. (every vector space has to contain the zero vector which is the zero matrix in this case) \\
\\
\(S_3\): \\
\(S_3\) is not a subspace of \(R^{2 \times 2}\). Because \(S_3\) is not closed under addition. \\
Let \(A, B \in S_3\). Then \(A = \begin{pmatrix} a & b \\ c & d \end{pmatrix}\) and \(B = \begin{pmatrix} e & f \\ g & h \end{pmatrix}\) where \(a^2 = b^2\) and \(e^2 = f^2\). \\
Then, \(A + B = \begin{pmatrix} a + e & b + f \\ c + g & d + h \end{pmatrix}\) where \((a + e)^2 = (b + f)^2\) is not necessarily true. \\
\begin{align*}
   (a + e)^2 &= (b + f)^2 \\
   a^2 + 2ae + e^2 &= b^2 + 2bf + f^2  \qquad \text{use \(a^2 = b^2\) and \(e^2 = f^2\)} \\
   2ae &= 2bf \\
   ae &= bf \\
\end{align*}
Which is not always true. Thus \(S_3\) is not closed under addition. \\

\section*{Problem 3}
We need to show two things: vectors are linearly independent and they span the subspace \(W\). \\
\textbf{Vectors are linearly independent:} \\
Let \(c_1(x^3 - x^2) + c_2(x^3 - x) = 0\). \\
Then, \(c_1x^3 - c_1x^2 + c_2x^3 - c_2x = 0\). \\
Then, \((c_1 + c_2)x^3 - c_1x^2 - c_2x = 0\). \\
The coefficients of \(x^3\), \(x^2\) and \(x\) must be equal to 0. \\
Thus, \(c_1\) and \(c_2\) must be equal to 0. \\
\textbf{Vectors span the subspace \(W\):} \\
Let \(p(x) \in W\). Then, \(p(0) = 0\) and \(p(1) = 0\). \\
Then, \(p(x) = a_3x^3 + a_2x^2 + a_1x + a_0\). \\
Then, \(p(0) = a_0 = 0\). \\
Then, \(p(1) = a_3 + a_2 + a_1 = 0\). \\
Thus, \(p(x) = a_3x^3 + a_2x^2 + (-a_3 - a_2)x\). \\
Then, \(p(x) = a_3(x^3 - x^2) + a_2(x^3 - x)\). \\
Thus, \(p(x)\) can be written as a linear combination of the vectors \(x^3 - x^2\) and \(x^3 - x\). \\
\\
\textbf{Extend the basis to a basis for \(R^3[x]\):} \\
We want to be able write any polynomial in \(R^3[x]\) as a linear combination of the basis vectors. We know that \(dim R^3[x] = 4\). So we need 2 more linearly independent vectors. \\
From the basis extension theorem we know that if  we add two more linearly independent vectors to our original set of vectors, we will have a basis for \(R^3[x]\). \\
So let's peak two vectors outside of the subspace \(W\) which are linearly independent. \\
\(q_1(x) = 1\) and \(q_2(x) = x\) are linearly independent and not in the subspace \(W\). They are not in the subspace \(W\) because \(q_1(1) \neq 0\) and \(q_2(1) \neq 0\). \\
Thus, a basis for \(R^3[x]\) is \(\{x^3 - x^2, x^3 - x, 1, x\}\). \\


\section*{Problem 4}
We are interested in the matrices in the folowing form:
\begin{align*}
   A = \begin{pmatrix}
      a & b & c \\
      b & d & e \\
      c & e & f
   \end{pmatrix}
\end{align*}
So, if we determine the upper part (or lower part) of the matrix we can determine the whole matrix. \\
We can write the matrix \(A\) as a linear combination of the following matrices:
\begin{align*}
   A = a \begin{pmatrix}
      1 & 0 & 0 \\
      0 & 0 & 0 \\
      0 & 0 & 0
   \end{pmatrix} + b \begin{pmatrix}
      0 & 1 & 0 \\
      1 & 0 & 0 \\
      0 & 0 & 0
   \end{pmatrix} + c \begin{pmatrix}
      0 & 0 & 1 \\
      0 & 0 & 0 \\
      1 & 0 & 0
   \end{pmatrix} + d \begin{pmatrix}
      0 & 0 & 0 \\
      0 & 1 & 0 \\
      0 & 0 & 0
   \end{pmatrix} + e \begin{pmatrix}
      0 & 0 & 0 \\
      0 & 0 & 1 \\
      0 & 1 & 0
   \end{pmatrix} + f \begin{pmatrix}
      0 & 0 & 0 \\
      0 & 0 & 0 \\
      0 & 0 & 1
   \end{pmatrix}
\end{align*}
These matrices are trivially linearly independent. \\
Since there are 6 linearly independent matrices, the dimension of \(S_3\) is 6. \\
\\
\textbf{Extend this basis to a basis for \(R^{3 \times 3}\):} \\
We want to be able write any matrix in \(R^{3 \times 3}\) as a linear combination of the basis vectors. We know that \(dim R^{3 \times 3} = 9\). So we need 3 more linearly independent vectors. \\
Intiutively we want to be able to control the lower part of the matrix. Following matrices are linearly independent and not in the subspace \(S_3\):
\begin{align*}
   A_1 = \begin{pmatrix}
      0 & 0 & 0 \\
      0 & 0 & 0 \\
      1 & 0 & 0
   \end{pmatrix}, A_2 = \begin{pmatrix}
      0 & 0 & 0 \\
      0 & 0 & 0 \\
      0 & 1 & 0
   \end{pmatrix}, A_3 = \begin{pmatrix}
      0 & 0 & 0 \\
      1 & 0 & 0 \\
      0 & 0 & 0
   \end{pmatrix}
\end{align*}

Thus, a basis for \(R^{3 \times 3}\) is \(\{A_1, A_2, A_3\} \cup S_3\).

\end{document}