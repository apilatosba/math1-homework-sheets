\documentclass{article}
\usepackage{amsmath, amsthm, amssymb}
\usepackage{tikz}
\usepackage{array}

\begin{document}
\section*{\huge Mathematics Homework Sheet 2}
\begin{flushright}
   \textbf{Authors: Abdullah Oguz Topcuoglu \& Ahmed Waleed Ahmed Badawy Shora}
\end{flushright}


% 1. The following tables show the results of the arithmetical operations in Z3 (where ⊕ and
%  denote addition and multiplication modulo 3).
% ⊕ [0] [1] [2]
% [0] [0] [1] [2]
% [1] [1] [2] [0]
% [2] [2] [0] [1]
%  [0] [1] [2]
% [0] [0] [0] [0]
% [1] [0] [1] [2]
% [2] [0] [2] [1]
% (a) Compute the corresponding tables for Z5 and Z7.
% (b) Compute the corresponding tables for Z4 and deduce that (Z4, ⊕, ) is not a field
% (where ⊕ and  denote addition and multiplication modulo 4).
% 2. Let n be a natural number.
% (a) Suppose that [m] ∈ Zn. Show that [i] 7→ [i] + [m] is a bijection Zn 7→ Zn (i.e. a
% permutation of Zn).
% (b) Write the permutations [i] 7→ [i] + [1], [i] 7→ [i] + [2], [i] 7→ [i] + [3] and [i] 7→ [i] + [4]
% von Z8 in cyclic notation.
% 3. Let p be a prime number.
% (a) Suppose that [m] ∈ Zp \ {[0]}. Show that [i] 7→ [m].[i] is a bijection
% Zp \ {[0]} 7→ Zp \ {[0]} (i.e. a permutation of Zp \ {[0]}).
% (b) Write the permutations [i] 7→ [6].[i] and [i] 7→ [2].[i] von Z7 \ {[0]} in cyclic
% notation.
% 4. Two sets M1, M2 have the same cardinality if there is a bijection M1 → M2. Let I be
% an arbitrary interval. Prove that I and R have the same cardinality.
% [Hint: Consider (half-)open and (half-)closed intervals, and likewise finite and infinite
% intervals separately.]
% 5. Define a relation ∼ on N0 × N0 by
% (p, n) ∼ (q, m) ⇔ p + m = q + n.
% (a) Show that ∼ is an equivalence relation on N0 × N0.
% (b) Show that
% (p, n) ∼ (k + p, k + n)
% for all k ∈ N0.
% (c) Denote the equivalence class [(k, 0)] by k and define the ‘sum’ of two equivalence
% classes by the formula
% [(p, n)] + [(q, m)] = [(p + q, n + m)].
% Determine the equivalence class −k with the property that
% −k + k = 0.
% [You may assume that ‘+’ is well defined.]


\section*{Problem 1}

\subsection*{(a)}

\(Z_5\):
\[
   \begin{array}{c|ccccc}
   \oplus & 0 & 1 & 2 & 3 & 4 \\\hline
   0 & 0 & 1 & 2 & 3 & 4 \\
   1 & 1 & 2 & 3 & 4 & 0 \\
   2 & 2 & 3 & 4 & 0 & 1 \\
   3 & 3 & 4 & 0 & 1 & 2 \\
   4 & 4 & 0 & 1 & 2 & 3 \\
   \end{array}
\]

\[
\begin{array}{c|ccccc}
\times & 0 & 1 & 2 & 3 & 4 \\\hline
0 & 0 & 0 & 0 & 0 & 0 \\
1 & 0 & 1 & 2 & 3 & 4 \\
2 & 0 & 2 & 4 & 1 & 3 \\
3 & 0 & 3 & 1 & 4 & 2 \\
4 & 0 & 4 & 3 & 2 & 1 \\
\end{array}
\]

\(Z_7\):
\[
   \begin{array}{c|ccccccc}
   \oplus & 0 & 1 & 2 & 3 & 4 & 5 & 6 \\\hline
   0 & 0 & 1 & 2 & 3 & 4 & 5 & 6 \\
   1 & 1 & 2 & 3 & 4 & 5 & 6 & 0 \\
   2 & 2 & 3 & 4 & 5 & 6 & 0 & 1 \\
   3 & 3 & 4 & 5 & 6 & 0 & 1 & 2 \\
   4 & 4 & 5 & 6 & 0 & 1 & 2 & 3 \\
   5 & 5 & 6 & 0 & 1 & 2 & 3 & 4 \\
   6 & 6 & 0 & 1 & 2 & 3 &4&5\\
   \end{array}
\]

\[
\begin{array}{c|ccccccc}
\times & 0 & 1 & 2 & 3 & 4 & 5 & 6 \\\hline
0 & 0 & 0 & 0 & 0 & 0 & 0 & 0 \\
1 & 0 & 1 & 2 & 3 & 4 & 5 & 6 \\
2 & 0 & 2 & 4 & 6 & 1 & 3 & 5 \\
3 & 0 & 3 & 6 & 2 & 5 & 1 & 4 \\
4 & 0 & 4 & 1 & 5 & 2 & 6 & 3 \\
5 & 0 & 5 & 3 & 1 & 6 & 4 & 2 \\
6 & 0 & 6 & 5 & 4 & 3 & 2 & 1 \\
\end{array}
\]

\subsection*{(b)}
\(Z_4\):
\[
   \begin{array}{c|cccc}
   \oplus & 0 & 1 & 2 & 3 \\\hline
   0 & 0 & 1 & 2 & 3 \\
   1 & 1 & 2 & 3 & 0 \\
   2 & 2 & 3 & 0 & 1 \\
   3 & 3 & 0 & 1 & 2 \\
   \end{array}
\]

\[
\begin{array}{c|cccc}
\times & 0 & 1 & 2 & 3 \\\hline
0 & 0 & 0 & 0 & 0 \\
1 & 0 & 1 & 2 & 3 \\
2 & 0 & 2 & 0 & 2 \\
3 & 0 & 3 & 2 & 1 \\
\end{array}
\]

\((Z_4, \oplus, \times)\) is not a field because \([2]\) does not have a multiplicative inverse. \\

\section*{Problem 2}

% Let n be a natural number.
% (a) Suppose that [m] ∈ Zn. Show that [i] 7→ [i] + [m] is a bijection Zn 7→ Zn (i.e. a
% permutation of Zn).
% (b) Write the permutations [i] 7→ [i] + [1], [i] 7→ [i] + [2], [i] 7→ [i] + [3] and [i] 7→ [i] + [4]
% von Z8 in cyclic notation.

\subsection*{(a)}

Let \(f: Z_n \to Z_n\) be:
\[
   f([i]) = [i] + [m], \qquad [m] \in Z_n
\]

We want to show that \(f\) is a bijection.
\begin{itemize}
   \item \textbf{Injective:} Suppose \(f([i_1]) = f([i_2])\):
   \[
      [i_1] + [m] = [i_2] + [m]
   \]
   Thus,
   \[
      [i_1] = [i_2]
   \]
   Thus, \(f\) is injective.
   \item \textbf{Surjective:} Let \([j] \in Z_n\). We want to show that there exists an \([x]\) such that \(f([x]) = [j]\):
   \[
      f([j - m]) = [j - m] + [m] = [j]
   \]
   Thus, \(f\) is surjective.
\end{itemize}
Thus, \(f\) is a bijection.

\subsection*{(b)}
\([i] \rightarrow [i] + [1]\):
\(
   (01234567)
\)
\\
\([i] \rightarrow [i] + [2]\):
\(
   (0246)(1357)
\)
\\
\([i] \rightarrow [i] + [3]\):
\(
   (03614725)
\)
\\
\([i] \rightarrow [i] + [4]\):
\(
   (04)(15)(26)(37)
\)


\end{document}