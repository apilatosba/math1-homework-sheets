\documentclass{article}
\usepackage{amsmath, amsthm, amssymb}
\usepackage{tikz}
\usepackage{array}
\usepackage{graphicx} % Added for including the PDF file as an image
\usepackage{float}
\usepackage{mathpazo}
\usetikzlibrary{angles, quotes}

\begin{document}
\section*{\huge Mathematics Homework Sheet 4}
\begin{flushright}
   \textbf{Authors: Abdullah Oguz Topcuoglu \& Ahmed Waleed Ahmed Badawy Shora}
\end{flushright}

% 1. Calculate (1552303, 233927) and find integers m and n such that
% (1552303, 233927) = 1552303m + 233927n.
% 2. Compute 1062
% , 1064
% , 1068 and 10611 (mod 143). You should give all the steps in your
% calculations and not write down any numbers with more than 5 digits.
% 3. (a) Compute the solution set of the simultaneous equations
% x ≡ 2 (mod 3),
% x ≡ 5 (mod 7),
% x ≡ 8 (mod 11)
% by applying the Chinese remainder theorem twice.
% (b) What are the last two digits of the number 4919? [Hint: We want to compute
% the number 4919 (mod 100). Note that 100 = 25 × 4.]
% 4. (a) Show using Fermat’s little theorem that 63 and 341 are not prime numbers.
% [Hint: 62 = 6.10 + 2, 340 = 3.113 + 1 and
% 1 ≡ 2
% 6
% (mod 63), 1 ≡ 563
% (mod 341).]
% (b) Show using Fermat’s little theorem that 32769 is not a prime number.
% (c) Let p be a prime number. Show using Fermat’s little theorem that
% (a + b)
% p ≡ (a
% p + b
% p
% ) (mod p).
% (d) Compute
% (37433709 + 742011127)
% 3709 (mod 3709).
% [Hint: 3709 is a prime number.]
% 5. A commutative ring is a nonempty set equipped with two binary operations + (‘addition’)
% and · (‘multiplication’) which satisfy the field axioms with the exception of the existence of
% multiplicative inverses.
% Suppose that (R, +, ·) is a commutative ring. Show that
% Z × R = {(m, r) : m ∈ Z, r ∈ R}
% is a commutative ring with respect to addition and multiplication defined by
% (m1, r1) ⊕ (m2, r2) := (m1 + m2, r1 + r2),
% (m1, r1) ⊙ (m2, r2) := (m1m2, m1r2 + m2r1 + r1r2).
% (Here
% mr = r + · · · + r
% | {z }
% m times
% , nr = −r + · · · + (−r)
% | {z }
% −n times
% for m ≥ 0 and n ≤ 0.)

\section*{Problem 1}

% 1. Calculate (1552303, 233927) and find integers m and n such that
% (1552303, 233927) = 1552303m + 233927n.

Question is asking to find Bezout coefficients.

\begin{align*}
   1552303 &= 6 \cdot 233927 + 148741, & u_2 = u_0 - 6u_1 = 1, \quad & v_2 = v_0 - 6v_1 = -6 \\
   233927 &= 1 \cdot 148741 + 85186, & u_3 = u_1 - u_2 = -1, \quad & v_3 = v_1 - v_2 = 7 \\
   148741 &= 1 \cdot 85186 + 63555, & u_4 = u_2 - u_3 = 2, \quad & v_4 = v_2 - v_3 = -13 \\
   85186 &= 1 \cdot 63555 + 21631, & u_5 = u_3 - u_4 = -3, \quad & v_5 = v_3 - v_4 = 20 \\
   63555 &= 2 \cdot 21631 + 20293, & u_6 = u_4 - 2u_5 = 8, \quad & v_6 = v_4 - 2v_5 = -53 \\
   21631 &= 1 \cdot 20293 + 1338, & u_7 = u_5 - u_6 = -11, \quad & v_7 = v_5 - v_6 = 73 \\
   20293 &= 15 \cdot 1338 + 223, & u_8 = u_6 - 15u_7 = 173, \quad & v_8 = v_6 - 15v_7 = -1148 \\
   1338 &= 6 \cdot 223 + 0, & &
\end{align*}

Which means that \((1552303, 233927) = 223\). And we have the Bezout coefficients \(u_8 = 173, v_8 = -1148\)
\[
   1552303 \cdot 173 + 233927 \cdot (-1148) = 223
\]
Thus \(m = 173, n = -1148\).

\section*{Problem 2}

% 2. Compute 106^2
% , 106^4
% , 106^8 and 106^11 (mod 143). You should give all the steps in your
% calculations and not write down any numbers with more than 5 digits.

% From Euler's theorem we know that \(a^{\phi(n)} \equiv 1 \mod n\) for \((a,n) = 1\)
% \[
%    \phi(143) = \phi(11) \cdot \phi(13) = 10 \cdot 12 = 120
% \]
% We know that (106, 143) = 1, so we can use Euler's theorem.
% \[
%    106^{120} \equiv 1 \mod 143
% \]

Start with the fact that \(106 \equiv 106 \mod 143\).
\begin{align*}
   106 &\equiv 106 \mod 143 \qquad \text{(Square both sides)} \\
   106^2 = 11236 &\equiv 82 \mod 143 \qquad \text{(Square both sides)} \\
   106^4 &\equiv 82^2 \equiv 3 \mod 143 \qquad \text{(Square both sides)} \\
   106^8 &\equiv 3^2 \equiv 9 \mod 143 \\
\end{align*}

And note these:
\begin{align*}
   106^2 &= 11236 = 78 \cdot 143 + 82 \\
   82^2 &= 6724 = 46 \cdot 143 + 3 \\
   3^2 &= 9 = 0 \cdot 143 + 9 \\
\end{align*}

Now we can compute \(106^{11}\):
\begin{align*}
   106^{11} &= 106^8 \cdot 106^2 \cdot 106 \\
   &\equiv 9 \cdot 82 \cdot 106 \mod 143 \\
   &\equiv 738 \cdot 106 \mod 143 \\
   &\equiv (5 \cdot 143 + 23) \cdot 106 \mod 143 \\
   &\equiv 23 \cdot 106 \mod 143 \\
   &\equiv 2428 \mod 143 \\
   &\equiv (16 \cdot 143 + 140) \mod 143 \\
   &\equiv 140 \mod 143 \\
\end{align*}

So, these are the results:
\begin{align*}
   106^2 &\equiv 82 \mod 143 \\
   106^4 &\equiv 3 \mod 143 \\
   106^8 &\equiv 9 \mod 143 \\
   106^{11} &\equiv 140 \mod 143 \\
\end{align*}

\end{document}