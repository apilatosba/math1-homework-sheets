\documentclass{article}
\usepackage{amsmath, amsthm, amssymb}
\usepackage{tikz}
\usepackage{array}
\usepackage{graphicx} % Added for including the PDF file as an image
\usepackage{float}
\usepackage{mathpazo}
\usetikzlibrary{angles, quotes}

\begin{document}
\section*{\huge Mathematics Homework Sheet 9}
\begin{flushright}
   \textbf{Authors: Abdullah Oguz Topcuoglu \& Ahmed Waleed Ahmed Badawy Shora}
\end{flushright}


% 1. Construct a 4 × 4 real matrix D such that
% ker D =
% *
% 
% 
% 0
% 1
% 0
% 0
% 
%  ,
% 
% 
% 0
% 0
% 1
% 0
% 
% 
% +
% , Im D =
% *
% 
% 
% 1
% 1
% 1
% 1
% 
%  ,
% 
% 
% 0
% 1
% 1
% 1
% 
% 
% +
% .
% [Hint: D = (De1|De2|De3|De4) and Im D = ⟨De1, De2, De3, De4⟩.]
% 2. Find all real nontrivial solutions of the equations
% 2x1 − 3x2 − x3 + x4 = 0,
% 3x1 + 4x2 − 4x3 − 3x4 = 0,
% 17x2 − 5x3 − 9x4 = 0.
% Show that one of these solutions satisfies the equations
% x1 + x2 + x3 + x4 + 1 = 0,
% x1 − x2 − x3 − x4 − 3 = 0
% but none can be written as a linear combination of the vectors (0, 1, 2, 3) and (3, 2, 1, 0).
% 3. Show that
% det
% 
% 
% a
% 2 + 1 ab ac
% ab b2 + 1 bc
% ac bc c2 + 1
% 
%  = a
% 2 + b
% 2 + c
% 2 + 1.
% 4. Let K be a field.
% (a) Two matrices A, B = Kn×n are called similar if there is a matrix P ∈ GL(n, K) such
% that B = P
% −1AP. Prove that
% (i) similarity of matrices defines an equivalence relation ∼ on Kn×n
% ,
% (ii) A ∼ B if and only if there is a finite-dimensional vector space V over K with bases
% B, B
% ′ and a linear transformation T : V → V such that A = MB
% B
% (T), B = MB
% ′
% B′(T).
% (b) Prove that two similar matrices have the same determinant.
% (c) Let V be an n-dimensional vector space over K. How would you define the determinant
% of a linear transformation T : V → V ?
% 5. Which of the matrices
% 
% −3 4
% −1 1
% ,
% 
% −3 1
% −2 −1
% 
% ,
% 
% −3 3
% 2 −2
% 
% ,
% 
% 2 −1
% −1 3 
% are diagonalisable over Q, R or C?


\section*{Problem 1}
From the given hint i observe that the columns of the matrix is just the images of the standard basis vectors.
And since \(e_2\) and \(e_3\) are in the kernel, second and third columns of \(D\) are all zeros.
\begin{align*}
D = \begin{pmatrix}
x & 0 & 0 & x \\
x & 0 & 0 & x \\
x & 0 & 0 & x \\
x & 0 & 0 & x
\end{pmatrix}
\end{align*}

And also from the hint i observe that the image is just span of the column vectors of the matrix. So i will just plug the given vectors in the image of \(D\)
into the columns of \(D\)
\begin{align*}
D = \begin{pmatrix}
1 & 0 & 0 & 0 \\
1 & 0 & 0 & 1 \\
1 & 0 & 0 & 1 \\
1 & 0 & 0 & 1
\end{pmatrix}
\end{align*}


\end{document}