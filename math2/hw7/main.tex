\documentclass{article}
\usepackage{amsmath, amsthm, amssymb}
\usepackage{tikz}
\usepackage{array}
\usepackage{graphicx} % Added for including the PDF file as an image
\usepackage{float}
\usepackage{mathpazo}
\usetikzlibrary{angles, quotes}

\begin{document}
\section*{\huge Mathematics Homework Sheet 7}
\begin{flushright}
   \textbf{Authors: Abdullah Oguz Topcuoglu \& Ahmed Waleed Ahmed Badawy Shora}
\end{flushright}

\section*{Problem 1}

% 1. Which of the following transformations are linear?
% (i) R
% 2 → R,
% 
% x
% y
% 
% 7→ x + 2y
% (ii) R
% 2 → R,
% 
% x
% y
% 
% 7→ x + y
% 2
% (iii) R
% 2 → R,
% 
% x
% y
% 
% 7→ xy
% (iv) C → C, z 7→ z
% (v) R
% 2 → R
% 2
% ,
% 
% x
% y
% 
% 7→
% 
% x + 1
% y − 1
% 
% (vi) R
% 2 → R
% 2
% ,
% 
% x
% y
% 
% 7→
% 
% x − y
% x + 2y
% 
% (vii) Rn[x] → R, p(x) 7→ p(1)
% (viii) Rn[x] → Rn+2[x], p(x) 7→ x
% 2p(x)

A transformation T is linear if and only if it satisfies the following two properties for all vectors \( u, v \) and scalar \( c \):
\begin{enumerate}
    \item \( T(u + v) = T(u) + T(v) \)
    \item \( T(cu) = cT(u) \)
\end{enumerate}

\begin{enumerate}
    \item \( T: \mathbb{R}^2 \to \mathbb{R}, \quad \begin{pmatrix} x \\ y \end{pmatrix} \mapsto x + 2y \) is linear.
    \item \( T: \mathbb{R}^2 \to \mathbb{R}, \quad \begin{pmatrix} x \\ y \end{pmatrix} \mapsto x + y^2 \) is not linear. \\
    Because rule (1) is not satisfied.
    \item \( T: \mathbb{R}^2 \to \mathbb{R}, \quad \begin{pmatrix} x \\ y \end{pmatrix} \mapsto xy \) is not linear. \\
    Because rule (2) is not satisfied.
    \item \( T: \mathbb{C} \to \mathbb{C}, z \mapsto \overline{z} \) is linear.
    \item \( T: \mathbb{R}^2 \to \mathbb{R}^2, \quad \begin{pmatrix} x \\ y \end{pmatrix} \mapsto \begin{pmatrix} x + 1 \\ y - 1 \end{pmatrix} \) is not linear. \\
      Because rule (1) is not satisfied.
    \item \( T: \mathbb{R}^2 \to \mathbb{R}^2, \quad \begin{pmatrix} x \\ y \end{pmatrix} \mapsto \begin{pmatrix} x - y \\ x + 2y\end{pmatrix}  )\) is linear.
    \item \( T: \mathbb{R}^n[x] \to \mathbb{R}, p(x) \mapsto p(1) \) is linear.
    \item \( T: \mathbb{R}^n[x] \to \mathbb{R}^{n+2}[x], p(x) \mapsto x^2p(x) \) is linear.
\end{enumerate}

\section*{Problem 2}

% 2. (a) Let T : R
% 3 → R
% 3 be the linear transformation defined by the formula
% T
% 
% 
% x
% y
% z
% 
%  =
% 
% 
% x − y
% x + 2y − z
% 2x + y + z
% 
%  .
% Find the matrix of T with respect to the usual basis for R
% 3
% .
% (b) Let n ∈ N and T : Rn[x] → Rn[x] be the linear transformation defined by
% (T(p))(x) = p(x + 1).
% Find the matrix of T with respect to the usual basis for Rn[x].
% (c) Let T : R
% 2×2 → R
% 2×2 be the linear transformation defined by the formula
% T
% 
% a b
% c d 
% =
% 
% a 2b
% 3c 4d
% 
% .
% Find the matrix of T with respect to the usual basis for R
% 2×2

\subsection*{(a)}

The way we construct the matrix given a linear transformation is that we take the each basis vector and apply the
transformation to it and write the result to corresponding column of the matrix.

\begin{align*}
   T(e_1) &= T(1, 0, 0) = (1 - 0, 1 + 2 \cdot 0 - 0, 2 \cdot 1 + 0 + 0) = (1, 1, 2) \\
   T(e_2) &= T(0, 1, 0) = (0 - 1, 0 + 2 \cdot 1 - 0, 2 \cdot 0 + 1 + 0) = (-1, 2, 1) \\
   T(e_3) &= T(0, 0, 1) = (0 - 0, 0 + 2 \cdot 0 - 1, 2 \cdot 0 + 0 + 1) = (0, -1, 1)
\end{align*}
Put each result as a column in the matrix \( A \):
\[
   A = \begin{pmatrix}
       1 & -1 & 0 \\
       1 & 2 & -1 \\
       2 & 1 & 1
   \end{pmatrix}
\]

\subsection*{(b)}
The standard basis looks like this: \(\{1, x, ..., x^n\}\)
\begin{align*}
   T(1) &= 1 \\
   T(x) &= x + 1 \\
   T(x^2) &= (x + 1)^2 = x^2 + 2x + 1 \\
   &\vdots \\
   T(x^k) &= (x + 1)^k = \sum_{i=0}^{k} \binom{k}{i}x^i \\
   &\vdots \\
   T(x^n) &= (x + 1)^n
\end{align*}

\[
   m_{i,j} =
      \begin{cases}
         \binom{j}{i} & \text{if } 0 \leq i \leq j \leq n \\
         0            & \text{if } i > j
      \end{cases}
\]

Each column of the matrix has the coefficients of a binomial expansion of \( (x + 1)^k \) for \( k = 0, 1, \ldots, n \).
For example the first column looks like this:
\[
   \begin{pmatrix}
      1 \\
      0 \\
      0 \\
      \vdots \\
      0
   \end{pmatrix}
\]
The second column looks like this:
\[
   \begin{pmatrix}
      1 \\
      1 \\
      0 \\
      \vdots \\
      0
   \end{pmatrix}
\]

\subsection*{(c)}

The standard basis of \( \mathbb{R}^{2 \times 2} \) is:
\[
   E_{11} = \begin{pmatrix} 1 & 0 \\ 0 & 0 \end{pmatrix}, \quad
   E_{12} = \begin{pmatrix} 0 & 1 \\ 0 & 0 \end{pmatrix}, \quad
   E_{21} = \begin{pmatrix} 0 & 0 \\ 1 & 0 \end{pmatrix}, \quad
   E_{22} = \begin{pmatrix} 0 & 0 \\ 0 & 1 \end{pmatrix}
\]
We compute \( T(E_{ij}) \) for each basis element:
\begin{align*}
   T(E_{11}) &= T\left(\begin{pmatrix} 1 & 0 \\ 0 & 0 \end{pmatrix}\right) = \begin{pmatrix} 1 & 0 \\ 0 & 0 \end{pmatrix} = E_{11} \\
   T(E_{12}) &= T\left(\begin{pmatrix} 0 & 1 \\ 0 & 0 \end{pmatrix}\right) = \begin{pmatrix} 0 & 2 \\ 0 & 0 \end{pmatrix} = 2E_{12} \\
   T(E_{21}) &= T\left(\begin{pmatrix} 0 & 0 \\ 1 & 0 \end{pmatrix}\right) = \begin{pmatrix} 0 & 0 \\ 3 & 0 \end{pmatrix} = 3E_{21} \\
   T(E_{22}) &= T\left(\begin{pmatrix} 0 & 0 \\ 0 & 1 \end{pmatrix}\right) = \begin{pmatrix} 0 & 0 \\ 0 & 4 \end{pmatrix} = 4E_{22}
\end{align*}
Thus, the matrix of \( T \) with respect to the basis \( \{E_{11}, E_{12}, E_{21}, E_{22}\} \) is:
\[
   A = \begin{pmatrix}
       1 & 0 & 0 & 0 \\
       0 & 2 & 0 & 0 \\
       0 & 0 & 3 & 0 \\
       0 & 0 & 0 & 4
   \end{pmatrix}
\]


\end{document}