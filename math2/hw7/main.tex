\documentclass{article}
\usepackage{amsmath, amsthm, amssymb}
\usepackage{tikz}
\usepackage{array}
\usepackage{graphicx} % Added for including the PDF file as an image
\usepackage{float}
\usepackage{mathpazo}
\usetikzlibrary{angles, quotes}

\begin{document}
\section*{\huge Mathematics Homework Sheet 7}
\begin{flushright}
   \textbf{Authors: Abdullah Oguz Topcuoglu \& Ahmed Waleed Ahmed Badawy Shora}
\end{flushright}

\section*{Problem 1}

% 1. Which of the following transformations are linear?
% (i) R
% 2 → R,
% 
% x
% y
% 
% 7→ x + 2y
% (ii) R
% 2 → R,
% 
% x
% y
% 
% 7→ x + y
% 2
% (iii) R
% 2 → R,
% 
% x
% y
% 
% 7→ xy
% (iv) C → C, z 7→ z
% (v) R
% 2 → R
% 2
% ,
% 
% x
% y
% 
% 7→
% 
% x + 1
% y − 1
% 
% (vi) R
% 2 → R
% 2
% ,
% 
% x
% y
% 
% 7→
% 
% x − y
% x + 2y
% 
% (vii) Rn[x] → R, p(x) 7→ p(1)
% (viii) Rn[x] → Rn+2[x], p(x) 7→ x
% 2p(x)

A transformation T is linear if and only if it satisfies the following two properties for all vectors \( u, v \) and scalar \( c \):
\begin{enumerate}
    \item \( T(u + v) = T(u) + T(v) \)
    \item \( T(cu) = cT(u) \)
\end{enumerate}

\begin{enumerate}
    \item \( T: \mathbb{R}^2 \to \mathbb{R}, \quad \begin{pmatrix} x \\ y \end{pmatrix} \mapsto x + 2y \) is linear.
    \item \( T: \mathbb{R}^2 \to \mathbb{R}, \quad \begin{pmatrix} x \\ y \end{pmatrix} \mapsto x + y^2 \) is not linear. \\
    Because rule (1) is not satisfied.
    \item \( T: \mathbb{R}^2 \to \mathbb{R}, \quad \begin{pmatrix} x \\ y \end{pmatrix} \mapsto xy \) is not linear. \\
    Because rule (2) is not satisfied.
    \item \( T: \mathbb{C} \to \mathbb{C}, z \mapsto \overline{z} \) is linear.
    \item \( T: \mathbb{R}^2 \to \mathbb{R}^2, \quad \begin{pmatrix} x \\ y \end{pmatrix} \mapsto \begin{pmatrix} x + 1 \\ y - 1 \end{pmatrix} \) is not linear. \\
      Because rule (1) is not satisfied.
    \item \( T: \mathbb{R}^2 \to \mathbb{R}^2, \quad \begin{pmatrix} x \\ y \end{pmatrix} \mapsto \begin{pmatrix} x - y \\ x + 2y\end{pmatrix}  )\) is linear.
    \item \( T: \mathbb{R}^n[x] \to \mathbb{R}, p(x) \mapsto p(1) \) is linear.
    \item \( T: \mathbb{R}^n[x] \to \mathbb{R}^{n+2}[x], p(x) \mapsto x^2p(x) \) is linear.
\end{enumerate}

\section*{Problem 2}

% 2. (a) Let T : R
% 3 → R
% 3 be the linear transformation defined by the formula
% T
% 
% 
% x
% y
% z
% 
%  =
% 
% 
% x − y
% x + 2y − z
% 2x + y + z
% 
%  .
% Find the matrix of T with respect to the usual basis for R
% 3
% .
% (b) Let n ∈ N and T : Rn[x] → Rn[x] be the linear transformation defined by
% (T(p))(x) = p(x + 1).
% Find the matrix of T with respect to the usual basis for Rn[x].
% (c) Let T : R
% 2×2 → R
% 2×2 be the linear transformation defined by the formula
% T
% 
% a b
% c d 
% =
% 
% a 2b
% 3c 4d
% 
% .
% Find the matrix of T with respect to the usual basis for R
% 2×2

\subsection*{(a)}

The way we construct the matrix given a linear transformation is that we take the each basis vector and apply the
transformation to it and write the result to corresponding column of the matrix.

\begin{align*}
   T(e_1) &= T(1, 0, 0) = (1 - 0, 1 + 2 \cdot 0 - 0, 2 \cdot 1 + 0 + 0) = (1, 1, 2) \\
   T(e_2) &= T(0, 1, 0) = (0 - 1, 0 + 2 \cdot 1 - 0, 2 \cdot 0 + 1 + 0) = (-1, 2, 1) \\
   T(e_3) &= T(0, 0, 1) = (0 - 0, 0 + 2 \cdot 0 - 1, 2 \cdot 0 + 0 + 1) = (0, -1, 1)
\end{align*}
Put each result as a column in the matrix \( A \):
\[
   A = \begin{pmatrix}
       1 & -1 & 0 \\
       1 & 2 & -1 \\
       2 & 1 & 1
   \end{pmatrix}
\]

\subsection*{(b)}
The standard basis looks like this: \(\{1, x, ..., x^n\}\)
\begin{align*}
   T(1) &= 1 \\
   T(x) &= x + 1 \\
   T(x^2) &= (x + 1)^2 = x^2 + 2x + 1 \\
   &\vdots \\
   T(x^k) &= (x + 1)^k = \sum_{i=0}^{k} \binom{k}{i}x^i \\
   &\vdots \\
   T(x^n) &= (x + 1)^n
\end{align*}

\[
   m_{i,j} =
      \begin{cases}
         \binom{j}{i} & \text{if } 0 \leq i \leq j \leq n \\
         0            & \text{if } i > j
      \end{cases}
\]

Each column of the matrix has the coefficients of a binomial expansion of \( (x + 1)^k \) for \( k = 0, 1, \ldots, n \).
For example the first column looks like this:
\[
   \begin{pmatrix}
      1 \\
      0 \\
      0 \\
      \vdots \\
      0
   \end{pmatrix}
\]
The second column looks like this:
\[
   \begin{pmatrix}
      1 \\
      1 \\
      0 \\
      \vdots \\
      0
   \end{pmatrix}
\]

\subsection*{(c)}

The standard basis of \( \mathbb{R}^{2 \times 2} \) is:
\[
   E_{11} = \begin{pmatrix} 1 & 0 \\ 0 & 0 \end{pmatrix}, \quad
   E_{12} = \begin{pmatrix} 0 & 1 \\ 0 & 0 \end{pmatrix}, \quad
   E_{21} = \begin{pmatrix} 0 & 0 \\ 1 & 0 \end{pmatrix}, \quad
   E_{22} = \begin{pmatrix} 0 & 0 \\ 0 & 1 \end{pmatrix}
\]
We compute \( T(E_{ij}) \) for each basis element:
\begin{align*}
   T(E_{11}) &= T\left(\begin{pmatrix} 1 & 0 \\ 0 & 0 \end{pmatrix}\right) = \begin{pmatrix} 1 & 0 \\ 0 & 0 \end{pmatrix} = E_{11} \\
   T(E_{12}) &= T\left(\begin{pmatrix} 0 & 1 \\ 0 & 0 \end{pmatrix}\right) = \begin{pmatrix} 0 & 2 \\ 0 & 0 \end{pmatrix} = 2E_{12} \\
   T(E_{21}) &= T\left(\begin{pmatrix} 0 & 0 \\ 1 & 0 \end{pmatrix}\right) = \begin{pmatrix} 0 & 0 \\ 3 & 0 \end{pmatrix} = 3E_{21} \\
   T(E_{22}) &= T\left(\begin{pmatrix} 0 & 0 \\ 0 & 1 \end{pmatrix}\right) = \begin{pmatrix} 0 & 0 \\ 0 & 4 \end{pmatrix} = 4E_{22}
\end{align*}
Thus, the matrix of \( T \) with respect to the basis \( \{E_{11}, E_{12}, E_{21}, E_{22}\} \) is:
\[
   A = \begin{pmatrix}
       1 & 0 & 0 & 0 \\
       0 & 2 & 0 & 0 \\
       0 & 0 & 3 & 0 \\
       0 & 0 & 0 & 4
   \end{pmatrix}
\]

\section*{Problem 3}

% We are given:

% 𝑀
% 1
% =
% (
% 1
% 0
% 1
% )
% (
% 3
% ×
% 1
% )
% ,
% 𝑀
% 2
% =
% (
%  
% 2
%   
% −
%  ⁣
% 1
%   
% 3
%  
% )
% (
% 1
% ×
% 3
% )
% ,
% 𝑀
% 3
% =
% (
% 4
% 2
% 1
% 3
% −
% 1
% 1
% )
% (
% 3
% ×
% 2
% )
% ,
% 𝑀
% 4
% =
% (
% 4
% 1
% 2
% 3
% )
% (
% 2
% ×
% 2
% )
% ,
% 𝑀
% 5
% =
% (
% 2
% −
% 1
% 1
% 3
% 1
% −
% 1
% )
% (
% 2
% ×
% 3
% )
% ,
% 𝑀
% 6
% =
% (
% 1
% 2
% 1
% 2
% 3
% 2
% 3
% 4
% 3
% )
% (
% 3
% ×
% 3
% )
% .
% M 
% 1
% ​
 
% M 
% 3
% ​
 
% M 
% 5
% ​
 
% ​
  
% = 
% ​
  
% 1
% 0
% 1
% ​
  
% ​
%  (3×1),
% = 
% ​
  
% 4
% 1
% −1
% ​
  
% 2
% 3
% 1
% ​
  
% ​
%  (3×2),
% =( 
% 2
% 3
% ​
  
% −1
% 1
% ​
  
% 1
% −1
% ​
%  )(2×3),
% ​
  
% M 
% 2
% ​
 
% M 
% 4
% ​
 
% M 
% 6
% ​
 
% ​
  
% =(2−13)(1×3),
% =( 
% 4
% 2
% ​
  
% 1
% 3
% ​
%  )(2×2),
% = 
% ​
  
% 1
% 2
% 3
% ​
  
% 2
% 3
% 4
% ​
  
% 1
% 2
% 3
% ​
  
% ​
%  (3×3).
% ​
 
% We compute 
% 𝑀
% 𝑖
% 𝑀
% 𝑗
% M 
% i
% ​
%  M 
% j
% ​
%   only when the inner dimensions match (i.e.\ 
% (columns of 
% 𝑀
% 𝑖
% )
% =
% (rows of 
% 𝑀
% 𝑗
% )
% (columns of M 
% i
% ​
%  )=(rows of M 
% j
% ​
%  )).

% 𝑀
% 1
% (
% 3
% ×
% 1
% )
% ×
% 𝑀
% 2
% (
% 1
% ×
% 3
% )
%   
% →
%   
% (
% 3
% ×
% 3
% )
% M 
% 1
% ​
%  (3×1)×M 
% 2
% ​
%  (1×3)→(3×3).

% 𝑀
% 1
%   
% =
%   
% (
% 1
% 0
% 1
% )
% ,
% 𝑀
% 2
% =
% (
% 2
% ,
%  
% −
% 1
% ,
%   
% 3
% )
% .
% M 
% 1
% ​
%  = 
% ​
  
% 1
% 0
% 1
% ​
  
% ​
%  ,M 
% 2
% ​
%  =(2,−1,3).
% Then

% 𝑀
% 1
%  
% 𝑀
% 2
% =
% (
% 1
% 0
% 1
% )
%  
% (
% 2
% ,
%  
% −
% 1
% ,
%  
% 3
% )
% =
% (
% 1
% ⋅
% 2
% 1
% ⋅
% (
% −
% 1
% )
% 1
% ⋅
% 3
% 0
% ⋅
% 2
% 0
% ⋅
% (
% −
% 1
% )
% 0
% ⋅
% 3
% 1
% ⋅
% 2
% 1
% ⋅
% (
% −
% 1
% )
% 1
% ⋅
% 3
% )
% =
% (
% 2
% −
% 1
% 3
% 0
% 0
% 0
% 2
% −
% 1
% 3
% )
% .
% M 
% 1
% ​
%  M 
% 2
% ​
%  = 
% ​
  
% 1
% 0
% 1
% ​
  
% ​
%  (2,−1,3)= 
% ​
  
% 1⋅2
% 0⋅2
% 1⋅2
% ​
  
% 1⋅(−1)
% 0⋅(−1)
% 1⋅(−1)
% ​
  
% 1⋅3
% 0⋅3
% 1⋅3
% ​
  
% ​
%  = 
% ​
  
% 2
% 0
% 2
% ​
  
% −1
% 0
% −1
% ​
  
% 3
% 0
% 3
% ​
  
% ​
%  .
% 𝑀
% 2
% (
% 1
% ×
% 3
% )
% ×
% 𝑀
% 1
% (
% 3
% ×
% 1
% )
%   
% →
%   
% (
% 1
% ×
% 1
% )
% M 
% 2
% ​
%  (1×3)×M 
% 1
% ​
%  (3×1)→(1×1).

% 𝑀
% 2
%  
% 𝑀
% 1
%   
% =
%   
% (
%  
% 2
% ,
%  
% −
% 1
% ,
%  
% 3
% )
%   
% (
% 1
% 0
% 1
% )
%   
% =
%   
% 2
% ⋅
% 1
%   
% +
%   
% (
% −
% 1
% )
% ⋅
% 0
%   
% +
%   
% 3
% ⋅
% 1
%   
% =
%   
% 5
% ,
% M 
% 2
% ​
%  M 
% 1
% ​
%  =(2,−1,3) 
% ​
  
% 1
% 0
% 1
% ​
  
% ​
%  =2⋅1+(−1)⋅0+3⋅1=5,
% so 
% 𝑀
% 2
%  
% 𝑀
% 1
% =
% [
%  
% 5
%  
% ]
% M 
% 2
% ​
%  M 
% 1
% ​
%  =[5].

% 𝑀
% 2
% (
% 1
% ×
% 3
% )
% ×
% 𝑀
% 3
% (
% 3
% ×
% 2
% )
%   
% →
%   
% (
% 1
% ×
% 2
% )
% M 
% 2
% ​
%  (1×3)×M 
% 3
% ​
%  (3×2)→(1×2).

% 𝑀
% 2
% =
% (
%  
% 2
% ,
%  
% −
% 1
% ,
%  
% 3
% )
% ,
% 𝑀
% 3
% =
% (
% 4
% 2
% 1
% 3
% −
% 1
% 1
% )
% .
% M 
% 2
% ​
%  =(2,−1,3),M 
% 3
% ​
%  = 
% ​
  
% 4
% 1
% −1
% ​
  
% 2
% 3
% 1
% ​
  
% ​
%  .
% Compute

% 𝑀
% 2
%  
% 𝑀
% 3
% =
% (
% 2
% ,
%  
% −
% 1
% ,
%  
% 3
% )
%  
% (
% 4
% 2
% 1
% 3
% −
% 1
% 1
% )
% =
% (
%  
% 2
% ⋅
% 4
% +
% (
% −
% 1
% )
% ⋅
% 1
% +
% 3
% ⋅
% (
% −
% 1
% )
% ,
% 2
% ⋅
% 2
% +
% (
% −
% 1
% )
% ⋅
% 3
% +
% 3
% ⋅
% 1
% )
% =
% (
%  
% 8
% −
% 1
% −
% 3
% ,
%   
% 4
% −
% 3
% +
% 3
% )
% =
% (
%  
% 4
% ,
%   
% 4
%  
% )
% .
% M 
% 2
% ​
%  M 
% 3
% ​
%  =(2,−1,3) 
% ​
  
% 4
% 1
% −1
% ​
  
% 2
% 3
% 1
% ​
  
% ​
%  =(2⋅4+(−1)⋅1+3⋅(−1),2⋅2+(−1)⋅3+3⋅1)=(8−1−3,4−3+3)=(4,4).
% 𝑀
% 2
% (
% 1
% ×
% 3
% )
% ×
% 𝑀
% 6
% (
% 3
% ×
% 3
% )
%   
% →
%   
% (
% 1
% ×
% 3
% )
% M 
% 2
% ​
%  (1×3)×M 
% 6
% ​
%  (3×3)→(1×3).

% 𝑀
% 6
% =
% (
% 1
% 2
% 1
% 2
% 3
% 2
% 3
% 4
% 3
% )
% .
% M 
% 6
% ​
%  = 
% ​
  
% 1
% 2
% 3
% ​
  
% 2
% 3
% 4
% ​
  
% 1
% 2
% 3
% ​
  
% ​
%  .
% Then

% 𝑀
% 2
%  
% 𝑀
% 6
% =
% (
%  
% 2
% ,
%  
% −
% 1
% ,
%  
% 3
% )
%  
% (
% 1
% 2
% 1
% 2
% 3
% 2
% 3
% 4
% 3
% )
% =
% (
%  
% 2
% ⋅
% 1
% +
% (
% −
% 1
% )
% ⋅
% 2
% +
% 3
% ⋅
% 3
% ,
%   
% 2
% ⋅
% 2
% +
% (
% −
% 1
% )
% ⋅
% 3
% +
% 3
% ⋅
% 4
% ,
%   
% 2
% ⋅
% 1
% +
% (
% −
% 1
% )
% ⋅
% 2
% +
% 3
% ⋅
% 3
% )
% .
% M 
% 2
% ​
%  M 
% 6
% ​
%  =(2,−1,3) 
% ​
  
% 1
% 2
% 3
% ​
  
% 2
% 3
% 4
% ​
  
% 1
% 2
% 3
% ​
  
% ​
%  =(2⋅1+(−1)⋅2+3⋅3,2⋅2+(−1)⋅3+3⋅4,2⋅1+(−1)⋅2+3⋅3).
% Compute entries:

% 2
% −
% 2
% +
% 9
% =
% 9
% ,
% 4
% −
% 3
% +
% 12
% =
% 13
% ,
% 2
% −
% 2
% +
% 9
% =
% 9.
% 2−2+9=9,4−3+12=13,2−2+9=9.
% So 
% 𝑀
% 2
%  
% 𝑀
% 6
% =
% (
%  
% 9
% ,
%   
% 13
% ,
%   
% 9
%  
% )
% M 
% 2
% ​
%  M 
% 6
% ​
%  =(9,13,9).

% 𝑀
% 3
% (
% 3
% ×
% 2
% )
% ×
% 𝑀
% 4
% (
% 2
% ×
% 2
% )
%   
% →
%   
% (
% 3
% ×
% 2
% )
% M 
% 3
% ​
%  (3×2)×M 
% 4
% ​
%  (2×2)→(3×2).

% 𝑀
% 3
% =
% (
% 4
% 2
% 1
% 3
% −
% 1
% 1
% )
% ,
% 𝑀
% 4
% =
% (
% 4
% 1
% 2
% 3
% )
% .
% M 
% 3
% ​
%  = 
% ​
  
% 4
% 1
% −1
% ​
  
% 2
% 3
% 1
% ​
  
% ​
%  ,M 
% 4
% ​
%  =( 
% 4
% 2
% ​
  
% 1
% 3
% ​
%  ).
% Then

% 𝑀
% 3
%  
% 𝑀
% 4
% =
% (
% 4
% ⋅
% 4
% +
% 2
% ⋅
% 2
% 4
% ⋅
% 1
% +
% 2
% ⋅
% 3
% 1
% ⋅
% 4
% +
% 3
% ⋅
% 2
% 1
% ⋅
% 1
% +
% 3
% ⋅
% 3
% −
% 1
% ⋅
% 4
% +
% 1
% ⋅
% 2
% −
% 1
% ⋅
% 1
% +
% 1
% ⋅
% 3
% )
% =
% (
% 16
% +
% 4
% 4
% +
% 6
% 4
% +
% 6
% 1
% +
% 9
% −
% 4
% +
% 2
% −
% 1
% +
% 3
% )
% =
% (
% 20
% 10
% 10
% 10
% −
% 2
% 2
% )
% .
% M 
% 3
% ​
%  M 
% 4
% ​
%  = 
% ​
  
% 4⋅4+2⋅2
% 1⋅4+3⋅2
% −1⋅4+1⋅2
% ​
  
% 4⋅1+2⋅3
% 1⋅1+3⋅3
% −1⋅1+1⋅3
% ​
  
% ​
%  = 
% ​
  
% 16+4
% 4+6
% −4+2
% ​
  
% 4+6
% 1+9
% −1+3
% ​
  
% ​
%  = 
% ​
  
% 20
% 10
% −2
% ​
  
% 10
% 10
% 2
% ​
  
% ​
%  .
% 𝑀
% 3
% (
% 3
% ×
% 2
% )
% ×
% 𝑀
% 5
% (
% 2
% ×
% 3
% )
%   
% →
%   
% (
% 3
% ×
% 3
% )
% M 
% 3
% ​
%  (3×2)×M 
% 5
% ​
%  (2×3)→(3×3).

% 𝑀
% 5
% =
% (
% 2
% −
% 1
% 1
% 3
% 1
% −
% 1
% )
% .
% M 
% 5
% ​
%  =( 
% 2
% 3
% ​
  
% −1
% 1
% ​
  
% 1
% −1
% ​
%  ).
% 𝑀
% 3
%  
% 𝑀
% 5
% =
% (
% 4
% ⋅
% 2
% +
% 2
% ⋅
% 3
% 4
% ⋅
% (
% −
% 1
% )
% +
% 2
% ⋅
% 1
% 4
% ⋅
% 1
% +
% 2
% ⋅
% (
% −
% 1
% )
% 1
% ⋅
% 2
% +
% 3
% ⋅
% 3
% 1
% ⋅
% (
% −
% 1
% )
% +
% 3
% ⋅
% 1
% 1
% ⋅
% 1
% +
% 3
% ⋅
% (
% −
% 1
% )
% −
% 1
% ⋅
% 2
% +
% 1
% ⋅
% 3
% −
% 1
% ⋅
% (
% −
% 1
% )
% +
% 1
% ⋅
% 1
% −
% 1
% ⋅
% 1
% +
% 1
% ⋅
% (
% −
% 1
% )
% )
% =
% (
% 8
% +
% 6
% −
% 4
% +
% 2
% 4
% −
% 2
% 2
% +
% 9
% −
% 1
% +
% 3
% 1
% −
% 3
% −
% 2
% +
% 3
% 1
% +
% 1
% −
% 1
% −
% 1
% )
% =
% (
% 14
% −
% 2
% 2
% 11
% 2
% −
% 2
% 1
% 2
% −
% 2
% )
% .
% M 
% 3
% ​
%  M 
% 5
% ​
%  = 
% ​
  
% 4⋅2+2⋅3
% 1⋅2+3⋅3
% −1⋅2+1⋅3
% ​
  
% 4⋅(−1)+2⋅1
% 1⋅(−1)+3⋅1
% −1⋅(−1)+1⋅1
% ​
  
% 4⋅1+2⋅(−1)
% 1⋅1+3⋅(−1)
% −1⋅1+1⋅(−1)
% ​
  
% ​
%  = 
% ​
  
% 8+6
% 2+9
% −2+3
% ​
  
% −4+2
% −1+3
% 1+1
% ​
  
% 4−2
% 1−3
% −1−1
% ​
  
% ​
%  = 
% ​
  
% 14
% 11
% 1
% ​
  
% −2
% 2
% 2
% ​
  
% 2
% −2
% −2
% ​
  
% ​
%  .
% 𝑀
% 4
% (
% 2
% ×
% 2
% )
% ×
% 𝑀
% 5
% (
% 2
% ×
% 3
% )
%   
% →
%   
% (
% 2
% ×
% 3
% )
% M 
% 4
% ​
%  (2×2)×M 
% 5
% ​
%  (2×3)→(2×3).

% 𝑀
% 4
% =
% (
% 4
% 1
% 2
% 3
% )
% ,
% 𝑀
% 5
% =
% (
% 2
% −
% 1
% 1
% 3
% 1
% −
% 1
% )
% .
% M 
% 4
% ​
%  =( 
% 4
% 2
% ​
  
% 1
% 3
% ​
%  ),M 
% 5
% ​
%  =( 
% 2
% 3
% ​
  
% −1
% 1
% ​
  
% 1
% −1
% ​
%  ).
% 𝑀
% 4
%  
% 𝑀
% 5
% =
% (
% 4
% ⋅
% 2
% +
% 1
% ⋅
% 3
% 4
% ⋅
% (
% −
% 1
% )
% +
% 1
% ⋅
% 1
% 4
% ⋅
% 1
% +
% 1
% ⋅
% (
% −
% 1
% )
% 2
% ⋅
% 2
% +
% 3
% ⋅
% 3
% 2
% ⋅
% (
% −
% 1
% )
% +
% 3
% ⋅
% 1
% 2
% ⋅
% 1
% +
% 3
% ⋅
% (
% −
% 1
% )
% )
% =
% (
% 8
% +
% 3
% −
% 4
% +
% 1
% 4
% −
% 1
% 4
% +
% 9
% −
% 2
% +
% 3
% 2
% −
% 3
% )
% =
% (
% 11
% −
% 3
% 3
% 13
% 1
% −
% 1
% )
% .
% M 
% 4
% ​
%  M 
% 5
% ​
%  =( 
% 4⋅2+1⋅3
% 2⋅2+3⋅3
% ​
  
% 4⋅(−1)+1⋅1
% 2⋅(−1)+3⋅1
% ​
  
% 4⋅1+1⋅(−1)
% 2⋅1+3⋅(−1)
% ​
%  )=( 
% 8+3
% 4+9
% ​
  
% −4+1
% −2+3
% ​
  
% 4−1
% 2−3
% ​
%  )=( 
% 11
% 13
% ​
  
% −3
% 1
% ​
  
% 3
% −1
% ​
%  ).
% 𝑀
% 5
% (
% 2
% ×
% 3
% )
% ×
% 𝑀
% 1
% (
% 3
% ×
% 1
% )
%   
% →
%   
% (
% 2
% ×
% 1
% )
% M 
% 5
% ​
%  (2×3)×M 
% 1
% ​
%  (3×1)→(2×1).

% 𝑀
% 5
% =
% (
% 2
% −
% 1
% 1
% 3
% 1
% −
% 1
% )
% ,
% 𝑀
% 1
% =
% (
% 1
% 0
% 1
% )
% .
% M 
% 5
% ​
%  =( 
% 2
% 3
% ​
  
% −1
% 1
% ​
  
% 1
% −1
% ​
%  ),M 
% 1
% ​
%  = 
% ​
  
% 1
% 0
% 1
% ​
  
% ​
%  .
% 𝑀
% 5
%  
% 𝑀
% 1
% =
% (
% 2
% ⋅
% 1
% +
% (
% −
% 1
% )
% ⋅
% 0
% +
% 1
% ⋅
% 1
% 3
% ⋅
% 1
% +
% 1
% ⋅
% 0
% +
% (
% −
% 1
% )
% ⋅
% 1
% )
% =
% (
% 2
% +
% 0
% +
% 1
% 3
% +
% 0
% −
% 1
% )
% =
% (
% 3
% 2
% )
% .
% M 
% 5
% ​
%  M 
% 1
% ​
%  =( 
% 2⋅1+(−1)⋅0+1⋅1
% 3⋅1+1⋅0+(−1)⋅1
% ​
%  )=( 
% 2+0+1
% 3+0−1
% ​
%  )=( 
% 3
% 2
% ​
%  ).
% 𝑀
% 5
% (
% 2
% ×
% 3
% )
% ×
% 𝑀
% 3
% (
% 3
% ×
% 2
% )
%   
% →
%   
% (
% 2
% ×
% 2
% )
% M 
% 5
% ​
%  (2×3)×M 
% 3
% ​
%  (3×2)→(2×2).

% 𝑀
% 3
% =
% (
% 4
% 2
% 1
% 3
% −
% 1
% 1
% )
% .
% M 
% 3
% ​
%  = 
% ​
  
% 4
% 1
% −1
% ​
  
% 2
% 3
% 1
% ​
  
% ​
%  .
% 𝑀
% 5
%  
% 𝑀
% 3
% =
% (
% 2
% ⋅
% 4
% +
% (
% −
% 1
% )
% ⋅
% 1
% +
% 1
% ⋅
% (
% −
% 1
% )
% 2
% ⋅
% 2
% +
% (
% −
% 1
% )
% ⋅
% 3
% +
% 1
% ⋅
% 1
% 3
% ⋅
% 4
% +
% 1
% ⋅
% 1
% +
% (
% −
% 1
% )
% ⋅
% (
% −
% 1
% )
% 3
% ⋅
% 2
% +
% 1
% ⋅
% 3
% +
% (
% −
% 1
% )
% ⋅
% 1
% )
% =
% (
% 8
% −
% 1
% −
% 1
% 4
% −
% 3
% +
% 1
% 12
% +
% 1
% +
% 1
% 6
% +
% 3
% −
% 1
% )
% =
% (
% 6
% 2
% 14
% 8
% )
% .
% M 
% 5
% ​
%  M 
% 3
% ​
%  =( 
% 2⋅4+(−1)⋅1+1⋅(−1)
% 3⋅4+1⋅1+(−1)⋅(−1)
% ​
  
% 2⋅2+(−1)⋅3+1⋅1
% 3⋅2+1⋅3+(−1)⋅1
% ​
%  )=( 
% 8−1−1
% 12+1+1
% ​
  
% 4−3+1
% 6+3−1
% ​
%  )=( 
% 6
% 14
% ​
  
% 2
% 8
% ​
%  ).
% 𝑀
% 5
% (
% 2
% ×
% 3
% )
% ×
% 𝑀
% 6
% (
% 3
% ×
% 3
% )
%   
% →
%   
% (
% 2
% ×
% 3
% )
% M 
% 5
% ​
%  (2×3)×M 
% 6
% ​
%  (3×3)→(2×3).

% 𝑀
% 6
% =
% (
% 1
% 2
% 1
% 2
% 3
% 2
% 3
% 4
% 3
% )
% .
% M 
% 6
% ​
%  = 
% ​
  
% 1
% 2
% 3
% ​
  
% 2
% 3
% 4
% ​
  
% 1
% 2
% 3
% ​
  
% ​
%  .
% 𝑀
% 5
%  
% 𝑀
% 6
% =
% (
% 2
% ⋅
% 1
% +
% (
% −
% 1
% )
% ⋅
% 2
% +
% 1
% ⋅
% 3
% 2
% ⋅
% 2
% +
% (
% −
% 1
% )
% ⋅
% 3
% +
% 1
% ⋅
% 4
% 2
% ⋅
% 1
% +
% (
% −
% 1
% )
% ⋅
% 2
% +
% 1
% ⋅
% 3
% 3
% ⋅
% 1
% +
% 1
% ⋅
% 2
% +
% (
% −
% 1
% )
% ⋅
% 3
% 3
% ⋅
% 2
% +
% 1
% ⋅
% 3
% +
% (
% −
% 1
% )
% ⋅
% 4
% 3
% ⋅
% 1
% +
% 1
% ⋅
% 2
% +
% (
% −
% 1
% )
% ⋅
% 3
% )
% =
% (
% 2
% −
% 2
% +
% 3
% 4
% −
% 3
% +
% 4
% 2
% −
% 2
% +
% 3
% 3
% +
% 2
% −
% 3
% 6
% +
% 3
% −
% 4
% 3
% +
% 2
% −
% 3
% )
% =
% (
% 3
% 5
% 3
% 2
% 5
% 2
% )
% .
% M 
% 5
% ​
%  M 
% 6
% ​
%  =( 
% 2⋅1+(−1)⋅2+1⋅3
% 3⋅1+1⋅2+(−1)⋅3
% ​
  
% 2⋅2+(−1)⋅3+1⋅4
% 3⋅2+1⋅3+(−1)⋅4
% ​
  
% 2⋅1+(−1)⋅2+1⋅3
% 3⋅1+1⋅2+(−1)⋅3
% ​
%  )=( 
% 2−2+3
% 3+2−3
% ​
  
% 4−3+4
% 6+3−4
% ​
  
% 2−2+3
% 3+2−3
% ​
%  )=( 
% 3
% 2
% ​
  
% 5
% 5
% ​
  
% 3
% 2
% ​
%  ).
% 𝑀
% 6
% (
% 3
% ×
% 3
% )
% ×
% 𝑀
% 1
% (
% 3
% ×
% 1
% )
%   
% →
%   
% (
% 3
% ×
% 1
% )
% M 
% 6
% ​
%  (3×3)×M 
% 1
% ​
%  (3×1)→(3×1).

% 𝑀
% 6
% =
% (
% 1
% 2
% 1
% 2
% 3
% 2
% 3
% 4
% 3
% )
% ,
% 𝑀
% 1
% =
% (
% 1
% 0
% 1
% )
% .
% M 
% 6
% ​
%  = 
% ​
  
% 1
% 2
% 3
% ​
  
% 2
% 3
% 4
% ​
  
% 1
% 2
% 3
% ​
  
% ​
%  ,M 
% 1
% ​
%  = 
% ​
  
% 1
% 0
% 1
% ​
  
% ​
%  .
% 𝑀
% 6
%  
% 𝑀
% 1
% =
% (
% 1
% ⋅
% 1
% +
% 2
% ⋅
% 0
% +
% 1
% ⋅
% 1
% 2
% ⋅
% 1
% +
% 3
% ⋅
% 0
% +
% 2
% ⋅
% 1
% 3
% ⋅
% 1
% +
% 4
% ⋅
% 0
% +
% 3
% ⋅
% 1
% )
% =
% (
% 2
% 4
% 6
% )
% .
% M 
% 6
% ​
%  M 
% 1
% ​
%  = 
% ​
  
% 1⋅1+2⋅0+1⋅1
% 2⋅1+3⋅0+2⋅1
% 3⋅1+4⋅0+3⋅1
% ​
  
% ​
%  = 
% ​
  
% 2
% 4
% 6
% ​
  
% ​
%  .
% 𝑀
% 6
% (
% 3
% ×
% 3
% )
% ×
% 𝑀
% 3
% (
% 3
% ×
% 2
% )
%   
% →
%   
% (
% 3
% ×
% 2
% )
% M 
% 6
% ​
%  (3×3)×M 
% 3
% ​
%  (3×2)→(3×2).

% 𝑀
% 3
% =
% (
% 4
% 2
% 1
% 3
% −
% 1
% 1
% )
% .
% M 
% 3
% ​
%  = 
% ​
  
% 4
% 1
% −1
% ​
  
% 2
% 3
% 1
% ​
  
% ​
%  .
% 𝑀
% 6
%  
% 𝑀
% 3
% =
% (
% 1
% ⋅
% 4
% +
% 2
% ⋅
% 1
% +
% 1
% ⋅
% (
% −
% 1
% )
% 1
% ⋅
% 2
% +
% 2
% ⋅
% 3
% +
% 1
% ⋅
% 1
% 2
% ⋅
% 4
% +
% 3
% ⋅
% 1
% +
% 2
% ⋅
% (
% −
% 1
% )
% 2
% ⋅
% 2
% +
% 3
% ⋅
% 3
% +
% 2
% ⋅
% 1
% 3
% ⋅
% 4
% +
% 4
% ⋅
% 1
% +
% 3
% ⋅
% (
% −
% 1
% )
% 3
% ⋅
% 2
% +
% 4
% ⋅
% 3
% +
% 3
% ⋅
% 1
% )
% =
% (
% 4
% +
% 2
% −
% 1
% 2
% +
% 6
% +
% 1
% 8
% +
% 3
% −
% 2
% 4
% +
% 9
% +
% 2
% 12
% +
% 4
% −
% 3
% 6
% +
% 12
% +
% 3
% )
% =
% (
% 5
% 9
% 9
% 15
% 13
% 21
% )
% .
% M 
% 6
% ​
%  M 
% 3
% ​
%  = 
% ​
  
% 1⋅4+2⋅1+1⋅(−1)
% 2⋅4+3⋅1+2⋅(−1)
% 3⋅4+4⋅1+3⋅(−1)
% ​
  
% 1⋅2+2⋅3+1⋅1
% 2⋅2+3⋅3+2⋅1
% 3⋅2+4⋅3+3⋅1
% ​
  
% ​
%  = 
% ​
  
% 4+2−1
% 8+3−2
% 12+4−3
% ​
  
% 2+6+1
% 4+9+2
% 6+12+3
% ​
  
% ​
%  = 
% ​
  
% 5
% 9
% 13
% ​
  
% 9
% 15
% 21
% ​
  
% ​
%  .
% All other products 
% 𝑀
% 𝑖
% 𝑀
% 𝑗
% M 
% i
% ​
%  M 
% j
% ​
%   are undefined (inner dimensions mismatch).

\[
M_1 = \begin{pmatrix} 1 \\ 0 \\ 1 \end{pmatrix} \quad (3 \times 1), \qquad
M_2 = \begin{pmatrix} 2 & -1 & 3 \end{pmatrix} \quad (1 \times 3), \qquad
M_3 = \begin{pmatrix} 4 & 1 \\ 1 & 3 \\ -1 & 1 \end{pmatrix} \quad (3 \times 2),
\]
\[
M_4 = \begin{pmatrix} 4 & 1 \\ 2 & 3 \end{pmatrix} \quad (2 \times 2), \qquad
M_5 = \begin{pmatrix} 2 & -1 & 1 \\ 3 & 1 & -1 \end{pmatrix} \quad (2 \times 3), \qquad
M_6 = \begin{pmatrix} 1 & 2 & 1 \\ 2 & 3 & 2 \\ 3 & 4 & 3 \end{pmatrix} \quad (3 \times 3).
\]

We can compute \( M_i M_j \) only when the number of columns of \( M_i \) equals the number of rows of \( M_j \).

\begin{itemize}
    \item \( M_1 (3 \times 1) \times M_2 (1 \times 3) \to (3 \times 3) \):
    \[
    M_1 M_2 = \begin{pmatrix} 1 \\ 0 \\ 1 \end{pmatrix} \begin{pmatrix} 2 & -1 & 3 \end{pmatrix}
    = \begin{pmatrix}
        2 & -1 & 3 \\
        0 & 0 & 0 \\
        2 & -1 & 3
    \end{pmatrix}
    \]
    \item \( M_2 (1 \times 3) \times M_1 (3 \times 1) \to (1 \times 1) \):
    \[
    M_2 M_1 = 2 \cdot 1 + (-1) \cdot 0 + 3 \cdot 1 = 5
    \]
    \item \( M_2 (1 \times 3) \times M_3 (3 \times 2) \to (1 \times 2) \):
    \[
    M_2 M_3 = \begin{pmatrix} 2 & -1 & 3 \end{pmatrix} \begin{pmatrix} 4 & 1 \\ 1 & 3 \\ -1 & 1 \end{pmatrix}
    = \begin{pmatrix}
        2 \cdot 4 + (-1) \cdot 1 + 3 \cdot (-1) & 2 \cdot 1 + (-1) \cdot 3 + 3 \cdot 1
    \end{pmatrix}
    = \begin{pmatrix}
        8 - 1 - 3 & 2 - 3 + 3
    \end{pmatrix}
    = \begin{pmatrix}
        4 & 2
    \end{pmatrix}
    \]
    \item \( M_2 (1 \times 3) \times M_6 (3 \times 3) \to (1 \times 3) \):
    \[
    M_2 M_6 = \begin{pmatrix} 2 & -1 & 3 \end{pmatrix}
    \begin{pmatrix}
        1 & 2 & 1 \\
        2 & 3 & 2 \\
        3 & 4 & 3
    \end{pmatrix}
    = \begin{pmatrix}
        2 \cdot 1 + (-1) \cdot 2 + 3 \cdot 3, &
        2 \cdot 2 + (-1) \cdot 3 + 3 \cdot 4, &
        2 \cdot 1 + (-1) \cdot 2 + 3 \cdot 3
    \end{pmatrix}
    = \begin{pmatrix}
        2 - 2 + 9, & 4 - 3 + 12, & 2 - 2 + 9
    \end{pmatrix}
    = \begin{pmatrix}
        9 & 13 & 9
    \end{pmatrix}
    \]
    \item \( M_3 (3 \times 2) \times M_4 (2 \times 2) \to (3 \times 2) \):
    \[
    M_3 M_4 = \begin{pmatrix} 4 & 1 \\ 1 & 3 \\ -1 & 1 \end{pmatrix} \begin{pmatrix} 4 & 1 \\ 2 & 3 \end{pmatrix}
    = \begin{pmatrix}
        4 \cdot 4 + 1 \cdot 2 & 4 \cdot 1 + 1 \cdot 3 \\
        1 \cdot 4 + 3 \cdot 2 & 1 \cdot 1 + 3 \cdot 3 \\
        -1 \cdot 4 + 1 \cdot 2 & -1 \cdot 1 + 1 \cdot 3
    \end{pmatrix}
    = \begin{pmatrix}
        16 + 2 & 4 + 3 \\
        4 + 6 & 1 + 9 \\
        -4 + 2 & -1 + 3
    \end{pmatrix}
    = \begin{pmatrix}
        18 & 7 \\
        10 & 10 \\
        -2 & 2
    \end{pmatrix}
    \]
    \item \( M_3 (3 \times 2) \times M_5 (2 \times 3) \to (3 \times 3) \):
    \[
    M_3 M_5 = \begin{pmatrix} 4 & 1 \\ 1 & 3 \\ -1 & 1 \end{pmatrix} \begin{pmatrix} 2 & -1 & 1 \\ 3 & 1 & -1 \end{pmatrix}
    \]
    Compute each entry:
    \[
    \begin{pmatrix}
        4 \cdot 2 + 1 \cdot 3 & 4 \cdot (-1) + 1 \cdot 1 & 4 \cdot 1 + 1 \cdot (-1) \\
        1 \cdot 2 + 3 \cdot 3 & 1 \cdot (-1) + 3 \cdot 1 & 1 \cdot 1 + 3 \cdot (-1) \\
        -1 \cdot 2 + 1 \cdot 3 & -1 \cdot (-1) + 1 \cdot 1 & -1 \cdot 1 + 1 \cdot (-1)
    \end{pmatrix}
    = \begin{pmatrix}
        8 + 3 & -4 + 1 & 4 - 1 \\
        2 + 9 & -1 + 3 & 1 - 3 \\
        -2 + 3 & 1 + 1 & -1 - 1
    \end{pmatrix}
    = \begin{pmatrix}
        11 & -3 & 3 \\
        11 & 2 & -2 \\
        1 & 2 & -2
    \end{pmatrix}
    \]
    \item \( M_4 (2 \times 2) \times M_5 (2 \times 3) \to (2 \times 3) \):
    \[
    M_4 M_5 = \begin{pmatrix} 4 & 1 \\ 2 & 3 \end{pmatrix} \begin{pmatrix} 2 & -1 & 1 \\ 3 & 1 & -1 \end{pmatrix}
    \]
    Compute each entry:
    \[
    \begin{pmatrix}
        4 \cdot 2 + 1 \cdot 3 & 4 \cdot (-1) + 1 \cdot 1 & 4 \cdot 1 + 1 \cdot (-1) \\
        2 \cdot 2 + 3 \cdot 3 & 2 \cdot (-1) + 3 \cdot 1 & 2 \cdot 1 + 3 \cdot (-1)
    \end{pmatrix}
    = \begin{pmatrix}
        8 + 3 & -4 + 1 & 4 - 1 \\
        4 + 9 & -2 + 3 & 2 - 3
    \end{pmatrix}
    = \begin{pmatrix}
        11 & -3 & 3 \\
        13 & 1 & -1
    \end{pmatrix}
    \]
    \item \( M_5 (2 \times 3) \times M_1 (3 \times 1) \to (2 \times 1) \):
    \[
    M_5 M_1 = \begin{pmatrix} 2 & -1 & 1 \\ 3 & 1 & -1 \end{pmatrix} \begin{pmatrix} 1 \\ 0 \\ 1 \end{pmatrix}
    = \begin{pmatrix}
        2 \cdot 1 + (-1) \cdot 0 + 1 \cdot 1 \\
        3 \cdot 1 + 1 \cdot 0 + (-1) \cdot 1
    \end{pmatrix}
    = \begin{pmatrix}
        2 + 0 + 1 \\
        3 + 0 - 1
    \end{pmatrix}
    = \begin{pmatrix}
        3 \\
        2
    \end{pmatrix}
    \]
    \item \( M_5 (2 \times 3) \times M_3 (3 \times 2) \to (2 \times 2) \):
    \[
    M_5 M_3 = \begin{pmatrix} 2 & -1 & 1 \\ 3 & 1 & -1 \end{pmatrix} \begin{pmatrix} 4 & 1 \\ 1 & 3 \\ -1 & 1 \end{pmatrix}
    \]
    Compute each entry:
    \[
    \begin{pmatrix}
        2 \cdot 4 + (-1) \cdot 1 + 1 \cdot (-1) & 2 \cdot 1 + (-1) \cdot 3 + 1 \cdot 1 \\
        3 \cdot 4 + 1 \cdot 1 + (-1) \cdot (-1) & 3 \cdot 1 + 1 \cdot 3 + (-1) \cdot 1
    \end{pmatrix}
    = \begin{pmatrix}
        8 - 1 - 1 & 2 - 3 + 1 \\
        12 + 1 + 1 & 3 + 3 - 1
    \end{pmatrix}
    = \begin{pmatrix}
        6 & 0 \\
        14 & 5
    \end{pmatrix}
    \]
    \item \( M_5 (2 \times 3) \times M_6 (3 \times 3) \to (2 \times 3) \):
    \[
    M_5 M_6 = \begin{pmatrix} 2 & -1 & 1 \\ 3 & 1 & -1 \end{pmatrix}
    \begin{pmatrix}
        1 & 2 & 1 \\
        2 & 3 & 2 \\
        3 & 4 & 3
    \end{pmatrix}
    \]
    Compute each entry:
    \[
    \begin{pmatrix}
        2 \cdot 1 + (-1) \cdot 2 + 1 \cdot 3 & 2 \cdot 2 + (-1) \cdot 3 + 1 \cdot 4 & 2 \cdot 1 + (-1) \cdot 2 + 1 \cdot 3 \\
        3 \cdot 1 + 1 \cdot 2 + (-1) \cdot 3 & 3 \cdot 2 + 1 \cdot 3 + (-1) \cdot 4 & 3 \cdot 1 + 1 \cdot 2 + (-1) \cdot 3
    \end{pmatrix}
    = \begin{pmatrix}
        2 - 2 + 3 & 4 - 3 + 4 & 2 - 2 + 3 \\
        3 + 2 - 3 & 6 + 3 - 4 & 3 + 2 - 3
    \end{pmatrix}
    = \begin{pmatrix}
        3 & 5 & 3 \\
        2 & 5 & 2
    \end{pmatrix}
    \]
    \item \( M_6 (3 \times 3) \times M_1 (3 \times 1) \to (3 \times 1) \):
    \[
    M_6 M_1 = \begin{pmatrix} 1 & 2 & 1 \\ 2 & 3 & 2 \\ 3 & 4 & 3 \end{pmatrix} \begin{pmatrix} 1 \\ 0 \\ 1 \end{pmatrix}
    = \begin{pmatrix}
        1 \cdot 1 + 2 \cdot 0 + 1 \cdot 1 \\
        2 \cdot 1 + 3 \cdot 0 + 2 \cdot 1 \\
        3 \cdot 1 + 4 \cdot 0 + 3 \cdot 1
    \end{pmatrix}
    = \begin{pmatrix}
        2 \\
        4 \\
        6
    \end{pmatrix}
    \]
    \item \( M_6 (3 \times 3) \times M_3 (3 \times 2) \to (3 \times 2) \):
    \[
    M_6 M_3 = \begin{pmatrix} 1 & 2 & 1 \\ 2 & 3 & 2 \\ 3 & 4 & 3 \end{pmatrix} \begin{pmatrix} 4 & 1 \\ 1 & 3 \\ -1 & 1 \end{pmatrix}
    \]
    Compute each entry:
    \[
    \begin{pmatrix}
        1 \cdot 4 + 2 \cdot 1 + 1 \cdot (-1) & 1 \cdot 1 + 2 \cdot 3 + 1 \cdot 1 \\
        2 \cdot 4 + 3 \cdot 1 + 2 \cdot (-1) & 2 \cdot 1 + 3 \cdot 3 + 2 \cdot 1 \\
        3 \cdot 4 + 4 \cdot 1 + 3 \cdot (-1) & 3 \cdot 1 + 4 \cdot 3 + 3 \cdot 1
    \end{pmatrix}
    = \begin{pmatrix}
        4 + 2 - 1 & 1 + 6 + 1 \\
        8 + 3 - 2 & 2 + 9 + 2 \\
        12 + 4 - 3 & 3 + 12 + 3
    \end{pmatrix}
    = \begin{pmatrix}
        5 & 8 \\
        9 & 13 \\
        13 & 18
    \end{pmatrix}
    \]
\end{itemize}

All other products \( M_i M_j \) are undefined (inner dimensions mismatch).



\end{document}