\documentclass{article}
\usepackage{amsmath, amsthm, amssymb}
\usepackage{tikz}
\usepackage{array}
\usepackage{graphicx} % Added for including the PDF file as an image
\usepackage{float}
\usepackage{mathpazo}
\usetikzlibrary{angles, quotes}

\begin{document}
\section*{\huge Mathematics Homework Sheet 8}
\begin{flushright}
   \textbf{Authors: Abdullah Oguz Topcuoglu \& Ahmed Waleed Ahmed Badawy Shora}
\end{flushright}

% 1. Let K be a field, A ∈ Km×n and B ∈ Kn×p
% .
% (a) Prove that (AB)
% T = BTAT.
% [Hint: Using the remark on page 86 of the lecture notes, one can write down formulae
% for c
% AB
% j
% and r
% BTAT
% j
% .]
% (b) Show that
% column rank AB ≤ column rank A row rank AB ≤ row rank B
% and deduce that
% rank AB ≤ min(rank A,rank B).
% [Hint: Read the remark on page 86 of the lecture notes.]
% 2. Determine for which values of λ ∈ R the real matrix
% Aλ =
% 
% 
% 1 λ 0 0
% λ 1 0 0
% 0 λ 1 0
% 0 0 λ 1
% 
% 
% is invertible, and compute the inverse matrix A
% −1
% λ
% for these values of λ.
% 3. Let p be a prime number. Determine whether the matrix
% C =
% 
% 
% 13 7 6
% −7 1 1
% 3 8 7
% 
%  ∈ Z
% 3×3
% p
% is invertible in the cases p = 2, p = 3 and p = 5, and compute C
% −1
% if it exists.
% 4. Let
% B =
% 
% 
% 2 1 1 1 2
% 3 2 1 1 2
% 4 2 2 3 5
% 2 1 1 2 3
% 
% 
% ∈ R
% 4×5
% and r = Rang B. Find matrices T ∈ GL(4, R) and S ∈ GL(5, R) such that
% T
% −1BS =
% 
% Ir 0
% 0 0
% . (?)
% [Hint: First convert B into echelon form using elementary row operations, then convert
% the resulting matrix into the form (?) using elemenary column operations. The matrix S is
% obtained by applying the column operations to I5 in the same order, while the matrix T is
% obtained by applying the inverse row operations to I4 in reverse order.]
% 5. For which values of λ and µ do the following real systems of linear equations have no
% solution, a unique solution, infinitely many solutions?
% (a) 2x + 3y + z = 5,
% 3x − y + λz = 2,
% x + 7y − 6z = µ
% (b) x + y − 4z = 0,
% 2x + 3y + z = 1,
% 4x + 7y + λz = µ

\section*{Problem 2}
Let's do elementary row operations.

% the way we are gonna write it is like this: (A_lambda | identity) and to the right of the matrix write the operations we are doing
\begin{align*}
    A_\lambda &= \begin{pmatrix}
    1 & \lambda & 0 & 0 \\
    \lambda & 1 & 0 & 0 \\
    0 & \lambda & 1 & 0 \\
    0 & 0 & \lambda & 1
\end{pmatrix} \\
&= \left(
    \begin{array}{cccc|cccc}
        1 & \lambda & 0 & 0 & 1 & 0 & 0 & 0 \\
        \lambda & 1 & 0 & 0 & 0 & 1 & 0 & 0 \\
        0 & \lambda & 1 & 0 & 0 & 0 & 1 & 0 \\
        0 & 0 & \lambda & 1 & 0 & 0 & 0 & 1
    \end{array}
\right) \\
&\left(
    \begin{array}{cccc|cccc}
        1 & \lambda & 0 & 0 & 1 & 0 & 0 & 0 \\
        0 & 1 - \lambda^2 & 0 & 0 & -\lambda & 1 & 0 & 0 \\
        0 & \lambda & 1 & 0 & 0 & 0 & 1 & 0 \\
        0 & 0 & \lambda & 1 & 0 & 0 & 0 & 1
    \end{array}
\right) & r_2 = r_2 - \lambda r_1 \\
\end{align*}

if $\lambda^2 \neq 1$, then we can divide by $1 - \lambda^2$ and get the following:
\begin{align*}
    A_\lambda &= \left(
    \begin{array}{cccc|cccc}
        1 & \lambda & 0 & 0 & 1 & 0 & 0 & 0 \\
        0 & 1 & 0 & 0 & -\frac{\lambda}{1 - \lambda^2} & \frac{1}{1 - \lambda^2} & 0 & 0 \\
        0 & \lambda & 1 & 0 & 0 & 0 & 1 & 0 \\
        0 & 0 & \lambda & 1 & 0 & 0 & 0 & 1
    \end{array}
\right) & r_2 = r_2 / (1 - \lambda^2)\\
&\left(
    \begin{array}{cccc|cccc}
        1 & \lambda & 0       & 0 & 1                                & 0                             & 0 & 0 \\
        0 & 1       & 0       & 0 & -\frac{\lambda}{1 - \lambda^2}   & \frac{1}{1 - \lambda^2}       & 0 & 0 \\
        0 & 0       & 1       & 0 & \frac{\lambda^2}{1 - \lambda^2} & -\frac{\lambda}{1 - \lambda^2} & 0 & 0 \\
        0 & 0       & \lambda & 1 & 0                                & 0                             & 0 & 1
    \end{array}
    \right) & r_3 = r_3 - \lambda r_2 \\
&\left(
    \begin{array}{cccc|cccc}
        1 & \lambda & 0       & 0                                   & 1                                & 0                             & 0 & 0 \\
        0 & 1       & 0       & 0                                   & -\frac{\lambda}{1 - \lambda^2}   & \frac{1}{1 - \lambda^2}       & 0 & 0 \\
        0 & 0       & 1       & 0                                   & \frac{\lambda^2}{1 - \lambda^2} & -\frac{\lambda}{1 - \lambda^2} & 0 & 0 \\
        0 & 0       & 0 & 1 & -\frac{\lambda^3}{1 - \lambda^2} & \frac{\lambda^2}{1 - \lambda^2}                             & 0 & 1
    \end{array}
    \right) & r_4 = r_4 - \lambda r_3 \\
&\left(
    \begin{array}{cccc|cccc}
        1 & 0 & 0       & 0                                   & 1 + \frac{\lambda^2}{1 - \lambda^2}                               & -\frac{\lambda}{1 - \lambda^2}                             & 0 & 0 \\
        0 & 1       & 0       & 0                                   & -\frac{\lambda}{1 - \lambda^2}   & \frac{1}{1 - \lambda^2}       & 0 & 0 \\
        0 & 0       & 1       & 0                                   & \frac{\lambda^2}{1 - \lambda^2} & -\frac{\lambda}{1 - \lambda^2} & 0 & 0 \\
        0 & 0       & 0 & 1 & -\frac{\lambda^3}{1 - \lambda^2} & \frac{\lambda^2}{1 - \lambda^2}                             & 0 & 1
    \end{array}
    \right) & r_1 = r_1 - \lambda r_2 \\
\end{align*}

So if $\lambda^2 \neq 1$, then we can write the inverse matrix as follows:
\begin{align*}
    A_\lambda^{-1} &= \left(
    \begin{array}{cccc}
        1 + \frac{\lambda^2}{1 - \lambda^2} & -\frac{\lambda}{1 - \lambda^2} & 0 & 0 \\
        -\frac{\lambda}{1 - \lambda^2} & \frac{1}{1 - \lambda^2} & 0 & 0 \\
        \frac{\lambda^2}{1 - \lambda^2} & -\frac{\lambda}{1 - \lambda^2} & 0 & 0 \\
        -\frac{\lambda^3}{1 - \lambda^2} & \frac{\lambda^2}{1 - \lambda^2} & 0 & 1
    \end{array}
    \right)
\end{align*}
If $\lambda^2 = 1$, then we have two cases:
\begin{itemize}
    \item If $\lambda = 1$, then the matrix becomes:
    \[
        A_1 = \begin{pmatrix}
            1 & 1 & 0 & 0 \\
            1 & 1 & 0 & 0 \\
            0 & 1 & 1 & 0 \\
            0 & 0 & 1 & 1
        \end{pmatrix}
    \]
    which is not invertible because the first row and the second row are the same.

    \item If $\lambda = -1$, then the matrix becomes:
    \[
        A_{-1} = \begin{pmatrix}
            1 & -1 & 0 & 0 \\
            -1 & 1 & 0 & 0 \\
            0 & -1 & 1 & 0 \\
            0 & 0 & -1 & 1
        \end{pmatrix}
    \]
    which is also not invertible because \(r_1 = -r_2\).
\end{itemize}

\end{document}