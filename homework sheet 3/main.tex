\documentclass{article}
\usepackage{amsmath, amsthm, amssymb}
\usepackage{array}

\begin{document}
\section*{\huge Mathematics Homework Sheet 3}
\begin{flushright}
   \textbf{Author: Abdullah Oguz Topcuoglu \& Yousef Farag}
\end{flushright}

\begin{flushleft}
   \textbf{\Large Problem 1}
\end{flushleft}

\begin{flushleft}
   \textbf{\Large Problem 1 (a)}
\end{flushleft}

For i=0:
\begin{align*}
   f_1(-1) &= -1 \\
   f_1(0)  &= 0 \\
   f_1(1)  &= 0 \\
\end{align*}
Meaning that:
\[
   f_1(\{-1, 0, 1\}) = \{-1, 0, 0\} = \{-1, 0\}
\]
\\
Now lets take a look at the inverse:
\begin{align*}
   f_1(-1) &= -1 \\
   f_1(0)  &= 0 \\
   f_1(1)  &= 0 \\
   f_1(2)  &= 1 \\
\end{align*}
These are the only values we can get -1, 0, 1. Meaning that:
\[
   f_1^{-1}(\{-1, 0, 1\}) = \{-1, 0, 1, 2\}
\]
\\
Now, for i=2 (the second function):
\begin{align*}
   f_2(-1) &= -4 \\
   f_2(0)  &= -3 \\
   f_2(1)  &= -2 \\
\end{align*}
Meaning that:
\[
   f_2(\{-1, 0, 1\}) = \{-4, -3, -2\}
\]
\\
Now lets take a look at the inverse:
\begin{align*}
   f_2(2) &= -1 \\
   f_2(3) &= 0 \\
   f_2(4) &= 1 \\
\end{align*}
These are the only values we can get -1, 0, 1. Meaning that:
\[
   f_2^{-1}(\{-1, 0, 1\}) = \{2, 3, 4\}
\]
\\
Now, for i=3 (the third function):
\begin{align*}
   f_3(-1) &= -2 \\
   f_3(0)  &= 0 \\
   f_3(1)  &= 2 \\
\end{align*}
Meaning that:
\[
   f_3(\{-1, 0, 1\}) = \{-2, 0, 2\}
\]
\\
Now lets take a look at the inverse:
\begin{align*}
   f_3(x) & = -1 \text{, No such $x$ exists in the domain of $f_3$} \\
          & \text{In other words, } f_3^{-1}(\{-1\}) = \emptyset    \\
   f_3(0) & = 0                                                     \\
   f_3(x) & = 1 \text{, No such $x$ exists in the domain of $f_3$}  \\
          & \text{In other words, } f_3^{-1}(\{1\}) = \emptyset     \\
\end{align*}
These are the only values we can get -1, 0, 1. Meaning that:
\[
   f_3^{-1}(\{-1, 0, 1\}) = \{0\}
\]


\begin{flushleft}
   \textbf{\Large Problem 1 (b)}
\end{flushleft}


\end{document}