\documentclass{article}
\usepackage{amsmath, amsthm, amssymb}
\usepackage{array}

\begin{document}
\section*{\huge Mathematics Homework Sheet 3}
\begin{flushright}
   \textbf{Author: Abdullah Oguz Topcuoglu \& Yousef Farag}
\end{flushright}

\begin{flushleft}
   \textbf{\Large Problem 1}
\end{flushleft}

\begin{flushleft}
   \textbf{\Large Problem 1 (a)}
\end{flushleft}

For i=0:
\begin{align*}
   f_1(-1) &= -1 \\
   f_1(0)  &= 0 \\
   f_1(1)  &= 0 \\
\end{align*}
Meaning that:
\[
   f_1(\{-1, 0, 1\}) = \{-1, 0, 0\} = \{-1, 0\}
\]
\\
Now lets take a look at the inverse:
\begin{align*}
   f_1(-1) &= -1 \\
   f_1(0)  &= 0 \\
   f_1(1)  &= 0 \\
   f_1(2)  &= 1 \\
\end{align*}
These are the only values we can get -1, 0, 1. Meaning that:
\[
   f_1^{-1}(\{-1, 0, 1\}) = \{-1, 0, 1, 2\}
\]
\\
Now, for i=2 (the second function):
\begin{align*}
   f_2(-1) &= -4 \\
   f_2(0)  &= -3 \\
   f_2(1)  &= -2 \\
\end{align*}
Meaning that:
\[
   f_2(\{-1, 0, 1\}) = \{-4, -3, -2\}
\]
\\
Now lets take a look at the inverse:
\begin{align*}
   f_2(2) &= -1 \\
   f_2(3) &= 0 \\
   f_2(4) &= 1 \\
\end{align*}
These are the only values we can get -1, 0, 1. Meaning that:
\[
   f_2^{-1}(\{-1, 0, 1\}) = \{2, 3, 4\}
\]
\\
Now, for i=3 (the third function):
\begin{align*}
   f_3(-1) &= -2 \\
   f_3(0)  &= 0 \\
   f_3(1)  &= 2 \\
\end{align*}
Meaning that:
\[
   f_3(\{-1, 0, 1\}) = \{-2, 0, 2\}
\]
\\
Now lets take a look at the inverse:
\begin{align*}
   f_3(x) & = -1 \text{, No such $x$ exists in the domain of $f_3$} \\
          & \text{In other words, } f_3^{-1}(\{-1\}) = \emptyset    \\
   f_3(0) & = 0                                                     \\
   f_3(x) & = 1 \text{, No such $x$ exists in the domain of $f_3$}  \\
          & \text{In other words, } f_3^{-1}(\{1\}) = \emptyset     \\
\end{align*}
These are the only values we can get -1, 0, 1. Meaning that:
\[
   f_3^{-1}(\{-1, 0, 1\}) = \{0\}
\]


\begin{flushleft}
   \textbf{\Large Problem 1 (b)}
\end{flushleft}

Lets start with \(f_1\): \\
\(f_1\) is not injective because it maps 0 and 1 to the same value, that is:
\[
   f_1(0) = f_1(1) = 0
\]
\\
\(f_1\) is surjective because it maps to all the values in the codomain. \\
Lets take an element from the codomain \(z \in Z\). \\
If \(z \leq 0\) then we can find an element in the domain of \(f_3\), \(x \in Z\) that maps to \(z\). Simply \(x = z\) works. \\
If \(z > 0\) then we can find an element in the domain of \(f_3\), \(x \in Z\) that maps to \(z\). Simply \(x = z + 1\) works. \\
\\
Now, lets take a look at \(f_2\): \\
\(f_2\) is injective because it maps each element in the domain to a unique element in the codomain, that is:
\begin{align*}
    & f_2(x) = f_2(y) \Rightarrow x = y \\
    & \\
    & f_2(x) = x - 3,\quad f_2(y) = y - 3    \\
    & x - 3 = y - 3 \Rightarrow x = y \\
\end{align*}
\\
\(f_2\) is surjective because it maps to all the values in the codomain. \\
Lets take an element from the codomain of \(f_2\), \(z \in Z\). \\
We cand find an element in the domain of \(f_2\), \(x \in Z\) that maps to \(z\). Simply \(x = z + 3\) works. \\
\\
Now, lets take a look at \(f_3\): \\
\(f_3\) is injective because it maps each element in the domain to a unique element in the codomain, that is:
\begin{align*}
    & f_3(x) = f_3(y) \Rightarrow x = y \\
    & \\
    & f_3(x) = 2x,\quad f_3(y) = 2y    \\
    & 2x = 2y \Rightarrow x = y \\
\end{align*}
\\
\(f_3\) is not surjective because it does not map to all the values in the codomain. \\
For example there is no element in the domain of \(f_3\) that maps to 1. Generally all the odd numbers are not in the range of \(f_3\). \\


\begin{flushleft}
   \textbf{\Large Problem 1 (c)}
\end{flushleft}

By looking at Problem 1 (b), \\
We see that \(f_1\) is not injective thus \(f_1\) is not bijective. \\
We see that \(f_2\) is injective and surjective thus \(f_2\) is bijective. \\
We see that \(f_3\) is not surjective thus \(f_3\) is not bijective. \\
\\
Inverse of \(f_2\):
\begin{align*}
   f_2: Z \rightarrow Z,& \quad x \rightsquigarrow x - 3 \\
   f_2^{-1}: Z \rightarrow Z,& \quad x \rightsquigarrow x + 3 \\
\end{align*}
\\

\begin{flushleft}
   \textbf{\Large Problem 2}
\end{flushleft}


\begin{flushleft}
   \textbf{\Large Problem 2 (b)}
\end{flushleft}

\[
   (f_i \circ g = f_i \circ h) \implies g = h
\]
\\


\begin{flushleft}
   \textbf{\Large Problem 3}
\end{flushleft}


\begin{flushleft}
   \textbf{\Large Problem 3 (a)}
\end{flushleft}

Lets assume that opposite of the statement is true, that is there are more than one identity element. We are going to try to find a contradiction. \\
Lets name them \(e_1\) and \(e_2\). \\
We know that both identity elements are in the set:
\begin{align*}
   & e_1 \in F \\
   & e_2 \in F \\
\end{align*}
\begin{align}
   e_1 & = e_1 \cdot e_2 & (\text{By definition of identity element \(e_2\)}) \\
       & = e_2 \cdot e_1 & (\text{By commutative property of multiplication}) \\
       & = e_2           & (\text{By definition of identity element \(e_1\)})
\end{align}
\\
So, this is a contradiction, we assumed there are two different identity elements but they turned out to be the same element. Thus, there can be only one identity element in a field. \\


\begin{flushleft}
   \textbf{\Large Problem 3 (b)}
\end{flushleft}

\[
   \forall a,b,c \in F \quad a + c = b + c \implies a = b
\]

Lets go step by step: \\
\begin{align}
   a &= a + 0 &\text{Identity element of addtiion} \\
     &= a + (c - c) &\text{Inverse element of addition} \\
     &= (a + c) - c &\text{Associative property of addition} \\
     &= (b + c) - c &\text{Given (a+c = b+c)} \\
     &= b + (c - c) &\text{Associative property of addition} \\
     &= b + 0 &\text{Inverse element of addition} \\
     &= b &\text{Identity element of addition}
\end{align}
\\
So, we have proven that if \(a + c = b + c\) then \(a = b\). \\

\end{document}