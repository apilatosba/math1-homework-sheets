\documentclass{article}
\usepackage{amsmath, amsthm, amssymb}
\usepackage{array}

\begin{document}
\section*{\huge Mathematics Homework Sheet 4}
\begin{flushright}
   \textbf{Author: Abdullah Oguz Topcuoglu \& Yousef Farag}
\end{flushright}

\section*{Problem 1}
We want to prove
\[
   (\forall n \in N) \land (x \in R) \land (x \geq -1) \qquad (1 + x)^n \geq 1 + nx
\]
by using mathematical induction.\\
\textbf{Base Case:} For $n = 1$, we have
\[
   (1 + x)^1 = 1 + x \geq 1 + 1 \cdot x
\]
Which is true for every \(x \in R\) so it means it is also true for \(x \in [-1, \infty]\)\\
\\
\textbf{Inductive Step:} Assume that the statement is true for \(n = k\), i.e.
\[
   (1 + x)^k \geq 1 + kx
\]
is true for every \(x \in [-1, \infty]\).\\
We want to prove that the statement is also true for \(n = k + 1\), i.e.
\[
   (1 + x)^{k + 1} \geq 1 + (k + 1)x
\]
for every \(x \in [-1, \infty]\).\\

\begin{align}
   (1 + x)^{k + 1} & = (1 + x)^k \cdot (1 + x)                                                                                  \\
                   & \geq (1 + kx) \cdot (1 + x) & (\text{Using inductive step}) \label{eg:step2}                                                 \\
                   & = 1 + kx + x + kx^2                                                                                        \\
                   & = 1 + (k + 1)x + kx^2                                                                                      \\
                   & \geq 1 + (k + 1)x            &\text{(Since \(kx^2 \geq 0\))}
\end{align}
\\
And this completes the proof.
\\
\\
\textbf{Inductive step alternative solution:} \\
We assume that the statement is true for \(n = k\), i.e.
\[
   (1 + x)^k \geq 1 + kx
\]
Multiply both sides by \(1 + x\), since \(1 + x > 0\) because \(x \in [-1, \infty]\), we have
\[
   (1 + x)^{k} (1 + x) \geq (1 + kx)(1 + x)
\]
\[
   (1 + x)^{k + 1} \geq 1 + kx + x + kx^2
\]
\[
   0 \geq -kx^2
\]
Add these two together, we get
\[
   (1 + x)^{k + 1} \geq 1 + (k + 1)x
\]
And this completes the proof.\\
\\
\\
\textbf{Give a counterexample to show that the condition x \(\geq\) -1 is necessary:}\\
Let's take \(x = -2\) and \(n = 2\). Then we have

\begin{align*}
   (1 - 2)^2 &\geq 1 + 2 \cdot (-2) \\
   (-1)^2    &\geq 1 - 4            \\
   1         &\geq -3               \\
\end{align*}
Which is not true. In fact it would be false when \(n\) is even. So the condition \(x \geq -1\) is necessary. Because that way \(1 + x\) is never negative

\end{document}