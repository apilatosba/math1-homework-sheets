\documentclass{article}
\usepackage{amsmath, amsthm, amssymb}
\usepackage{array}

\begin{document}
\section*{\huge Mathematics Homework Sheet 4}
\begin{flushright}
   \textbf{Author: Abdullah Oguz Topcuoglu \& Yousef Farag}
\end{flushright}

\section*{Problem 1}
We want to prove
\[
   (\forall n \in N) \land (x \in R) \land (x \geq -1) \qquad (1 + x)^n \geq 1 + nx
\]
by using mathematical induction.\\
\textbf{Base Case:} For $n = 1$, we have
\[
   (1 + x)^1 = 1 + x \geq 1 + 1 \cdot x
\]
Which is true for every \(x \in R\) so it means it is also true for \(x \in [-1, \infty]\)\\
\\
\textbf{Inductive step:} \\
We assume that the statement is true for \(n = k\), i.e.
\[
   (1 + x)^k \geq 1 + kx
\]
Multiply both sides by \(1 + x\), since \(1 + x > 0\) because \(x \in [-1, \infty]\), we have
\[
   (1 + x)^{k} (1 + x) \geq (1 + kx)(1 + x)
\]
\[
   (1 + x)^{k + 1} \geq 1 + kx + x + kx^2
\]
\[
   0 \geq -kx^2
\]
Add these two together, we get
\[
   (1 + x)^{k + 1} \geq 1 + (k + 1)x
\]
And this completes the proof.\\
\\
\\
\textbf{Give a counterexample to show that the condition x \(\geq\) -1 is necessary:}\\
Let's take \(x = -2\) and \(n = 2\). Then we have

\begin{align*}
   (1 - 2)^2 &\geq 1 + 2 \cdot (-2) \\
   (-1)^2    &\geq 1 - 4            \\
   1         &\geq -3               \\
\end{align*}
Which is not true. In fact it would be false when \(n\) is even. So the condition \(x \geq -1\) is necessary. Because that way \(1 + x\) is never negative

\section*{Problem 2}
\section*{Problem 2 (a)}
\[
   X_1 := \{x \in R: x^2 - 2x \leq 0\}
\]
What x values satisfy this condition? \\
\begin{align*}
   x^2 - 2x &\leq 0 \\
   x(x - 2) &\leq 0
\end{align*}
In order this inequality to be satisfied the signs of \(x\) and \(x - 2\) must be different or one of them needs to be zero, 
and this only happens when \(0 \leq x \leq 2\). \\
So this means:
\[
   X_1 = [0, 2]
\]
In this case \(X_1\) is bounded from below and above.\\
\begin{align*}
   sup X_1 &= 2 \\
   inf X_1 &= 0
\end{align*}
And \(sup X_1 \in X_1\) which means \(sup X_1\) is also the maximum value.\\
\(inf X_1 \in X_1\) which means \(inf X_1\) is also the minimum value.\\
\\
\section*{Problem 2 (b)}
\[
   X_2 := \{x \in R \setminus \{0\}: 5 - x^2 > \frac{4}{x^2}\}
\]
What x values satisfy this condition? \\
\begin{align*}
   5 - x^2 &> \frac{4}{x^2} \\
   5 - x^2 - \frac{4}{x^2} &> 0 \\
   \frac{5x^2 - x^4 - 4}{x^2} &> 0 \\
\end{align*}
Since \(x^2\) is always positive, we can multiply both sides by \(x^2\).\\
\begin{align*}
   5x^2 - x^4 - 4 &> 0 \\
   x^4 - 5x^2 + 4 &< 0 \\
   (x^2 - 4)(x^2 - 1) &< 0 \\
   (x - 2)(x + 2)(x - 1)(x + 1) &< 0
\end{align*}
So, this inequality is satisfied when \(-2 < x < -1 \quad \lor \quad 1 < x < 2\).\\
So this means:
\[
   X_2 = (-2, -1) \cup (1, 2)
\]
In this case \(X_2\) is bounded from below and above.\\
\begin{align*}
   sup X_2 &= 2 \\
   inf X_2 &= -2
\end{align*}
And \(sup X_2 \notin X_2\) which means \(sup X_2\) is not the maximum value.\\
\(inf X_2 \notin X_2\) which means \(inf X_2\) is not the minimum value.\\
Maximum and minimum values are not in the set.\\

\section*{Problem 3}
We say that \(x'\) is an supremum of \(Y\) if \(\forall x \in Y, x' > x \)\\
so given that \(sup Y\) exists for set \(Y\) we can take an element \(x \in Y\) such that we know 
that the following relation holds true for \(\forall y \in Y\) because of the existence of a supremum
\[
\forall y \in Y (supY > y)  \tag{1}
\]
Now according to the second property of ordered fields\\ \[
\forall a,b,c \in F: (a \leq b) 
\land(c \leq 0) \implies a.c \geq b.c \tag{2}
\] \\let a = y, c = -1, b = supY
from \(1\) we know that \(a < b \) and we know that \(-1 < 0 \)
thus using \(2\) we can conclude 
\[
   -1.y > -1.supY
\] from the 9th property of fields we can conclude
\[
   -y > -supY \tag{3}
\]
from 1 we know that 3 holds true for all y \(\in\) Y
and thus by the defintion of the infimum, the infimum of the set -Y exists and it is equal to -supY



\section*{Problem 4}
\section*{Problem 4 (a)}
We want to prove
\[
   \forall x,y \in R \quad |x + y| \leq |x| + |y|
\]
Let's continue with this inequality
\begin{align*}
   \forall x,y \in R \quad x &\leq |x|, \;  y \leq |y|, \; -x \leq |x|, \;  -y \leq |y| \\
   x + y &\leq |x| + |y| &(\text{Considering the first two inequalities above}) \\
   -x - y &\leq |x| + |y| &(\text{Considering the last two inequalities above}) \\
\end{align*}
(\(x+y\)) and (\(-x-y\)) is nothing but two possible outcomes of \(|x+y|\) \\
So, we have
\[
   |x + y| \leq |x| + |y|
\]
And this completes the proof.\\


\end{document}