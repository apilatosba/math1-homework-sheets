\documentclass{article}
\usepackage{amsmath, amsthm, amssymb}
\usepackage{array}

\begin{document}
\section*{\huge Mathematics Homework Sheet 6}
\begin{flushright}
   \textbf{Author: Abdullah Oguz Topcuoglu \& Yousef Farag}
\end{flushright}

\section*{Problem 1}
\section*{Problem 1 (a)}
We want to compute the following limit:
\[
   \lim_{n \rightarrow \infty} \frac{(n+1)^4(1 - 4n^3)^2}{(1+2n^2)^5}
\]
\\
The top part look somethin like this:
\[
   (n+1)^4(1 - 4n^3)^2 = 16n^{10} + \sum_{i = 0}^{i = 9} k_in^i
\]
And the bottom part look like this:
\[
   (1+2n^2)^5 = 32n^{10} + \sum_{i = 0}^{i = 9} k_in^i
\]
When we substitute these into the limit, we get:
\[
   \lim_{n \rightarrow \infty} \frac{16n^{10} + \sum_{i = 0}^{i = 9} k_in^i}{32n^{10} + \sum_{i = 0}^{i = 9} k_in^i}
\]
Divide the top and bottom by $n^{10}$:
\[
   \lim_{n \rightarrow \infty} \frac{16 + \sum_{i = 0}^{i = 9} k_in^{i-10}}{32 + \sum_{i = 0}^{i = 9} k_in^{i-10}}
\]
\[
   \frac{\lim_{n \rightarrow \infty}(16 + \sum_{i = 0}^{i = 9} k_in^{i-10})}{\lim_{n \rightarrow \infty}(32 + \sum_{i = 0}^{i = 9} k_in^{i-10})}
\]
\[
   \frac{16 + \sum_{i = 0}^{i = 9} k_i\lim_{n \rightarrow \infty}n^{i-10}}{32 + \sum_{i = 0}^{i = 9} k_i\lim_{n \rightarrow \infty}n^{i-10}}
\]
And we know that \(\lim_{n \rightarrow \infty} 1/n^i\) is zero for all \(i > 0\). Therefore, the limit is:
\[
   \frac{16}{32} = \frac{1}{2}
\]

\section*{Problem 1 (b)}
We want to compute the following limit:
\[
   \lim_{n \rightarrow \infty} \sqrt{n+1} - \sqrt{n}
\]
Multiply and divide by the conjugate:
\[
   \lim_{n \rightarrow \infty} \sqrt{n+1} - \sqrt{n} \cdot \frac{\sqrt{n+1} + \sqrt{n}}{\sqrt{n+1} + \sqrt{n}}
\]
\[
   \lim_{n \rightarrow \infty} \frac{n+1 - n}{\sqrt{n+1} + \sqrt{n}}
\]
\[
   \lim_{n \rightarrow \infty} \frac{1}{\sqrt{n+1} + \sqrt{n}}
\]
And observe that
\[
   \frac{1}{n^2} < \frac{1}{\sqrt{n+1} + \sqrt{n}} < \frac{1}{\sqrt{n}}
\]
when \(n > 10\) (10 is not a magic number, it is just a number that is big enough) 
and we only care about the tail of the sequences not the head.\\
And we know that:
\[
   \lim_{n \rightarrow \infty} \frac{1}{n^2} = 0
\]
\[
   \lim_{n \rightarrow \infty} \frac{1}{\sqrt{n}} = 0
\]
From the sandwich theorem, we can conclude that:
\[
   \lim_{n \rightarrow \infty} \frac{1}{\sqrt{n+1} + \sqrt{n}} = 0
\]

\section*{Problem 2}
\[
   a_n = (1 + \frac{1}{n})^n
\]
\[
   b_n = (1 + \frac{1}{n})^{n+1}
\]

\section*{Problem 2 (a)}
We want to prove that
\[
   \frac{a_{n + 1}}{a_n} = \left(1 - \frac{1}{(n+1)^2} \right)^{n+1} \frac{n+1}{n}
\]
Let's start
\begin{align*}
   \frac{a_{n+1}}{a_n} & = \frac{(1 + \frac{1}{n+1})^{n+1}}{(1 + \frac{1}{n})^n}                               \\
                        &= \frac{(n+2)^{n+1}n^n}{(n+1)^{2n+1}} \\
                        &= \frac{(n+2)^{n+1}n^{n+1}}{(n+1)^{2n+2}} \frac{n+1}{n} \\
                        &= \frac{(n+2)^{n+1}n^{n+1}}{((n+1)^2)^{n+1}} \frac{n+1}{n} \\
                        &= (\frac{(n+2)n}{(n+1)^2})^{n+1} \frac{n+1}{n} \\
                        &= (\frac{n^2+2n}{(n+1)^2})^{n+1} \frac{n+1}{n} \\
                        &= (\frac{n^2+2n+1 -1}{(n+1)^2})^{n+1} \frac{n+1}{n} \\
                        &= (\frac{n^2+2n+1}{(n+1)^2} - \frac{1}{((n+1)^2)})^{n+1} \frac{n+1}{n} \\
                        &= (1 - \frac{1}{(n+1)^2})^{n+1} \frac{n+1}{n} \\
\end{align*}
That's what we wanted to show. \\
\\
Now \(b_n\). We want to prove
\[
   \frac{b_n}{b_{n+1}} = \left(1 + \frac{1}{n(n+2)} \right)^{n+2} \frac{n}{n+1}
\]
Let's start
\begin{align*}
   \frac{b_n}{b_{n+1}} & = \frac{(1 + \frac{1}{n})^{n+1}}{(1 + \frac{1}{n+1})^{n+2}}                               \\
                        &= \frac{(n+1)^{2n+3}}{n^{n+1}(n+2)^{n+2}} \\
                        &= \frac{(n+1)^{2n+4}}{n^{n+2}(n+2)^{n+2}} \frac{n}{n+1} \\
                        &= \frac{((n+1)^2)^{n+2}}{n^{n+2}(n+2)^{n+2}} \frac{n}{n+1} \\
                        &= (\frac{(n+1)^2}{n(n+2)})^{n+2} \frac{n}{n+1} \\
                        &= (\frac{n^2 +2n + 1}{n(n+2)})^{n+2} \frac{n}{n+1} \\
                        &= (\frac{n(n+2) + 1}{n(n+2)})^{n+2} \frac{n}{n+1} \\
                        &= (1 + \frac{1}{n(n+2)})^{n+2} \frac{n}{n+1} \\
\end{align*}
That's what we wanted to show.

\section*{Problem 2 (b)}
We want to show that
\[
   a_{n+1} \geq a_n \quad \forall n \in N
\]
Let's start
\(a_{n+1} \geq a_n\) means that \(\frac{a_{n+1}}{a_n} \geq 1\). Because \(a_n > 0 \quad \forall n \in N\).
And we computed what \(\frac{a_{n+1}}{a_n}\) is in the previous part. It is \(\left(1 - \frac{1}{(n+1)^2} \right)^{n+1} \frac{n+1}{n}\).
So we want to show \(\left(1 - \frac{1}{(n+1)^2} \right)^{n+1} \frac{n+1}{n} \geq 1\) \\
From Bernoulli's inequality we have
\[
   (1 + x)^n \geq 1 + nx
\]
Choose \(x = - \frac{1}{n^2}\). Then we have
\[
   \left(1 - \frac{1}{n^2} \right)^{n} \geq 1 - \frac{n}{n^2} = 1 - \frac{1}{n}
\]
We can rewrite this by substituting \(n\) with \(n+1\):
\[
   \left(1 - \frac{1}{(n+1)^2} \right)^{n+1} \geq 1 - \frac{1}{n+1}
\]

\begin{align*}
   \left(1 - \frac{1}{(n+1)^2} \right)^{n+1} \frac{n+1}{n} &= \left(1 - \frac{1}{(n+1)^2} \right)^{n+1} \frac{n+1}{n} \\
      &=  \left(1 - \frac{1}{(n+1)^2} \right)^{n+1} \frac{n+1}{n} \geq (1 - \frac{1}{n+1}) \frac{n+1}{n} \\
      &= \frac{n}{n+1} \frac{n+1}{n} = 1
\end{align*}
And we showed that \(a_{n+1} \geq a_n \quad \forall n \in N\).\\
\\
We want to show that \(b_n\) is monotonically decreasing.
\[
   b_{n+1} \leq b_n \quad \forall n \in N
\]
Let's start
\(b_{n+1} \leq b_n\) means that \(\frac{b_n}{b_{n+1}} \geq 1\). Because \(b_{n+1} > 0 \quad \forall n \in N\).
And we know what \(\frac{b_n}{b_{n+1}}\) is from previous part. It is \(\left(1 + \frac{1}{n(n+2)} \right)^{n+2} \frac{n}{n+1}\).
So we want to show that
\[
   \left(1 + \frac{1}{n(n+2)} \right)^{n+2} \frac{n}{n+1} \geq 1
\]
In the Berboulli's ineqaulity choose \(x = \frac{1}{n(n-2)}\). Then we have
\[
   \left(1 + \frac{1}{n(n-2)} \right)^{n} \geq 1 + \frac{n}{n(n-2)} = 1 + \frac{1}{n-2}
\]
We can rewrite this by substituting \(n\) with \(n+2\):
\[
   \left(1 + \frac{1}{(n+2)n} \right)^{n+2} \geq 1 + \frac{1}{n}
\]

\begin{align*}
   \left(1 + \frac{1}{n(n+2)} \right)^{n+2} \frac{n}{n+1} &= \left(1 + \frac{1}{n(n+2)} \right)^{n+2} \frac{n}{n+1} \\
      &=  \left(1 + \frac{1}{n(n+2)} \right)^{n+2} \frac{n}{n+1} \geq (1 + \frac{1}{n}) \frac{n}{n+1} \\
      &= \frac{n+1}{n} \frac{n}{n+1} = 1
\end{align*}
And we showed that \(b_{n+1} \leq b_n \quad \forall n \in N\).

\section*{Problem 2 (c)}
We want to show
\[
   a_n \leq b_n \quad \forall n \in N
\]
Let's start
\[
   a_n \leq b_n \quad \forall n \in N \Rightarrow \frac{a_n}{b_n} \leq 1
\]
From definiton of \(a_n\) and \(b_n\), we have
\[
   \frac{a_n}{b_n} = \frac{(1 + \frac{1}{n})^n}{(1 + \frac{1}{n})^{n+1}} = \frac{1}{1 + \frac{1}{n}} = \frac{n}{n+1} \leq 1
\]
And we showed that \(a_n \leq b_n \quad \forall n \in N\).\\
\\
Now we want to show why \(a_n\) and \(b_n\) are convergent.\\
We know that \(b_n\) is monotonically decreasing. This means that \(\sup b_n\) exists.
And we also know that \(a_n \leq b_n \quad \forall n \in N\). This means that \(a_n \leq \sup b_n\).
And we also know that \(a_n\) is monotonically increasing. This means that \(a_n\) is convergent 
because it is monotonically increasing and bounded above.\\
We know that \(a_n\) is monotonically increasing that means that \(\inf a_n\) exists. And we also know that
\(a_n \leq b_n \quad \forall n \in N\). This means that \(b_n \geq \inf a_n\).
And we also know that \(b_n\) is monotonically decreasing. This means that \(b_n\) is convergent.\\
We consider the limit of \(\lim_{n \rightarrow \infty}\frac{a_n}{b_n}\).
\begin{align*}
   & \frac{a_n}{b_n} = \frac{(1 + \frac{1}{n})^n}{(1 + \frac{1}{n})^{n+1}} = \frac{1}{1 + \frac{1}{n}} = \frac{n}{n+1} \\
   & \lim_{n \rightarrow \infty} \frac{a_n}{b_n} = \lim_{n \rightarrow \infty} \frac{n}{n+1} = 1 \\
   &=\lim_{n \rightarrow \infty} \frac{a_n}{b_n} = \frac{\lim_{n \rightarrow \infty} a_n}{\lim_{n \rightarrow \infty} b_n} \\
   &\lim_{n \rightarrow \infty} a_n = \lim_{n \rightarrow \infty} b_n
\end{align*}
That's what we wanted to show.

\section*{Problem 3}
\section*{Problem 3(a)}
\begin{align*}
   a_1 &= 1 \\
   a_{n+1} &= \frac{1}{1 + a_n} \quad \forall n \in N \\
\end{align*}
The limit:
Whatever the limit of \(a_n\) is, it is also the limit of \(a_{n+1}\). Let's call the limit \(l\).
\[
   l = \lim_{n \rightarrow \infty} a_n = \lim_{n \rightarrow \infty} a_{n+1}
\]
Using the recursion definition, we have
\begin{align*}
   \lim_{n \rightarrow \infty} a_{n+1} &= \lim_{n \rightarrow \infty} \frac{1}{1 + a_n} \\
   \lim_{n \rightarrow \infty} a_{n+1} &= \frac{1}{1 + \lim_{n \rightarrow \infty} a_n} \\
   l &= \frac{1}{1 + l} \\
   l + l^2 &= 1 \\
   l_1 &= \frac{1+\sqrt{5}}{2}, \quad l_2 = \frac{1-\sqrt{5}}{2}
\end{align*}
And we know that \(\forall n \in N \quad a_n > 0\) so the limit is \(l = l_1 = \frac{1+\sqrt{5}}{2}\)
\section*{Problem 3(b)}
We know that \(f_{n+2} = f_{n+1} + f_{n} \) if we then divide both sides by \(f_{n+1}\) then we get
\( \frac{f_{n+2}}{f_{n+1}}  = 1 + \frac{f_{n}}{f_{n+1}} \)\\
 we let \(x_{n}\) = \(\frac{f_{n+1}}{f_{n}}\) if we substitute in the statment above then we get:
 \[
   x_{n+1} = 1 + \frac{1}{x_{n}}
 \]
 Let the limit of \(x_{n} = L\) by the properties of limits we know that the limit of \(x_(n+1)\) is also equal to \(L\)
 By substituting in the statment above we get :
 \[
   L = 1 + \frac{1}{L}
 \]
 Multiply both sides by \( L \):
\[
L^2 = L + 1
\]
\[L^2 - L - 1 = 0\]
By solving this equation we get two numbers:
\[
   \frac{1}{2} + \frac{\sqrt{5}}{2}, \frac{1}{2} - \frac{\sqrt{5}}{2}
\]
Since this sequence doesn't produce any negative numbers then we can say that 
this sequence converges to \(\frac{1}{2} + \frac{\sqrt{5}}{2}\) because it is monotonically increasing and bounded from below by 1 (\(x_{1} = 1\)) and bounded from above by
\(\frac{1}{2} + \frac{\sqrt{5}}{2}\)
\end{document}