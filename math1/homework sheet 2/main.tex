\documentclass{article}
\usepackage{amsmath, amsthm, amssymb}
\usepackage{array}

\begin{document}
\section*{\huge Mathematics Homework Sheet 2}
\begin{flushright}
   \textbf{Author: Abdullah Oguz Topcuoglu \& Yousef Farag}
\end{flushright}

\begin{flushleft}
   \textbf{\Large Problem 1}
\end{flushleft}

Inductive set \(M \subseteq R\) means that
\[
   1 \in M
\]
\[
   \forall n \in M \implies n+1 \in M
\]

We want to show that the set \( \bigcap_{i \in I} M_i \) is inductive.
We know that \(\forall i \in I, M_i\) is inductive. \\
We need to show two things: \\
\begin{itemize}
   \item \(1 \in \bigcap_{i \in I} M_i\)
   \item If \(n \in \bigcap_{i \in I} M_i\), then \(n+1 \in \bigcap_{i \in I} M_i\)
\end{itemize}

Let's start with the first one: \\
Is \(1 \in \bigcap_{i \in I} M_i\)? \\
Yes because \(\forall i \in I, 1 \in M_i\). \\
\\
Now let's move on to the second one: \\
Pick an element \(n \in \bigcap_{i \in I} M_i\). \\
This means that \(\forall i \in I, n \in M_i\). \\
By definition of inductive set, \(\forall i \in I, n+1 \in M_i\). \\
Which means that \(n+1 \in \bigcap_{i \in I} M_i\). \\
\\
Thus, \(\bigcap_{i \in I} M_i\) is inductive.

\begin{flushleft}
   \textbf{\Large Problem 2}
\end{flushleft}

\[
   \sum_{k=0}^{n} k^2 = \frac{n(n+1)(2n+1)}{6}
\]
\\
is what we want to prove using mathematical induction \\
We need to do two things: \\
\begin{itemize}
   \item Prove the base case
   \item Prove the inductive step
\end{itemize}
Let's start with the base case: \\
Insert n = 0 into the equation:
\[
   \sum_{k=0}^{0} k^2 = 0^2 = 0 = \frac{0(0+1)(2*0+1)}{6} = 0
\]
Now the inductive step: \\
Assume that the equation holds for n = m, that is:
\[
   \sum_{k=0}^{m} k^2 = \frac{m(m+1)(2m+1)}{6}
\]
Now we need to prove that the equation holds for n = m + 1:
\[
   \sum_{k=0}^{m+1} k^2 = \frac{(m+1)(m+2)(2m+3)}{6}
\]
\\
\begin{align*}
   \sum_{k=0}^{m+1} k^2 & = \sum_{k=0}^{m} k^2 + (m+1)^2     \\
                        & = \frac{m(m+1)(2m+1)}{6} + (m+1)^2 & (\text{This step uses the induction assumption}) \\
                        &= \frac{m(m+1)(2m+1) + 6 (m+1)^2}{6} \\
                        &= \frac{(m+1)(m(2m+1)+6(m+1))}{6} \\
                        &= \frac{(m+1)(2m^2+m+6m+6)}{6} \\
                        &= \frac{(m+1)(2m^2+7m+6)}{6} \\
                        &= \frac{(m+1)((m+2)(2m+3))}{6}
\end{align*}

\begin{flushleft}
   \textbf{\Large Problem 3}
\end{flushleft}


\begin{flushleft}
   \textbf{\large Problem 3 (a)}
\end{flushleft}

Symmetric relation means that
\[
   \forall x, y \in X, xRy \iff yRx
\]

$R_1$ and $R_2$ are symmetric. \\
\[
   \forall x, y \in X, xR_1y \iff yR_1x
\]
\[ 
   \forall x, y \in X, xR_2y \iff yR_2x
\]
Let's pick an element from $R_1 \cup R_2$. $(x, y) \in R_1 \cup R_2$ \\
Which means that $(x, y) \in R_1$ or $(x, y) \in R_2$ \\
If it is in $R_1$, then $(y, x) \in R_1$ \\
If it is in $R_2$, then $(y, x) \in R_2$ \\
In both cases $(y, x) \in R_1 \cup R_2$ \\
Which means that $R_1 \cup R_2$ is symmetric. \\

\begin{flushleft}
   \textbf{\large Problem 3 (b)}
\end{flushleft}

Reflexive relation means that
\[
   \forall x \in X, xRx
\]
$R_1$ is reflexive.
\[
   \forall x \in X, xR_1x
\]
$R_2$ is arbitrary. \\
Let $R_3$ be $R_1 \cup R_2$. \\
Let $x \in X$ \\
Is $xR_3x$? \\
If $xR_3x$, then $(x, x) \in R_3$ \\
Which means that $(x, x) \in R_1$ or $(x, x) \in R_2$ \\
And we know that $xR_1x$ is true because $R_1$ is reflexive. \\
Thus $xR_3x$ is true which means $R_3$ is reflexive \\


\begin{flushleft}
   \textbf{\large Problem 3 (c)}
\end{flushleft}

Antisymmetric relation means that
\[
   \forall x, y \in X, xRy \land yRx \implies x = y
\]

$R_1$ and $R_2$ is antisymmetric. \\
Let's take a look at this example: \\
\[
   X = \{1, 2\}
\]
\[
   R_1 = \{(1, 2)\}
\]
\[
   R_2 = \{(2, 1)\}
\]
$R_1$ and $R_2$ are antisymmetric. But $R_1 \cup R_2$ is not antisymmetric because it contains $(1, 2)$ and $(2, 1)$ and $1 \not= 2$. \\


\begin{flushleft}
   \textbf{\Large Problem 4}
\end{flushleft}

\[X = Z \times N\]
\[
   (a, b) \sim (c, d) \iff ad = bc
\]

\begin{flushleft}
   \textbf{\large Problem 4 (a)}
\end{flushleft}


Relation $\sim$ being an equivalence relation means that it is reflexive, symmetric and transitive. \\
Let's start with reflexivity: \\
\[
   (a, b) \sim (a, b) \implies ab = ba
\]
Above statement is true for all $a \in Z$, $b \in N$. \\
Thus $\sim$ is reflexive. \\
\\
Now let's check symmetry: \\
\[
   \forall a,c \in Z \;\; \forall b,d \in N \;\; (a, b) \sim (c, d) \iff ad = bc \iff cb = da \iff (c, d) \sim (a, b)
\]
Thus $\sim$ is symmetric. \\
\\
Now let's check transitivity: \\
\[
   \forall a,b,c,d,e,f \in Z \;\; \forall x,y,z \in N \;\; (a, b) \sim (c, d) \land (c, d) \sim (e, f) \implies (a, b) \sim (e, f)
\]
\[
   (a, b) \sim (c, d) \iff ad = bc
\]
\[
   (c, d) \sim (e, f) \iff cf = de
\]
Multiply these two equations together, we get:
\[
   adcf = bcde
\]
\[
   af = be
\]
\[
   af = be \iff (a, b) \sim (e, f)
\]
Thus $\sim$ is transitive.


\begin{flushleft}
   \textbf{\large Problem 4 (b)}
\end{flushleft}

\((a, b) \sim (a', b') \land (c, d) \sim (c', d')\) means that
\[
   ab' = a'b \land cd' = c'd
\]
We want prove \((ad+cb,bd) \sim (a'd'+c'b',b'd')\) \\
\((ad+cb,bd) \sim (a'd'+c'b',b'd')\) means that
\[
   (ad+cb)b'd' = (a'd'+c'b')bd
\]
\[
   adb'd' + cbb'd' = a'd'bd + c'b'bd
\]
\[
   adb'd' - a'd'bd = c'b'bd - cbb'd'
\]
\[
   dd'(ab' - a'b) = bb'(c'd - cd')
\]
\[
   dd'(0) = bb'(0) \quad (\text{Since} \quad ab' = a'b \; and \; cd' = c'd)
\]
\(0 = 0\) which is true for every a b c d a' b' c' d'. Thus \((ad+cb,bd) \sim (a'd'+c'b',b'd')\)

\end{document}