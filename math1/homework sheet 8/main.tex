\documentclass{article}
\usepackage{amsmath, amsthm, amssymb}
\usepackage{array}

\begin{document}
\section*{\huge Mathematics Homework Sheet 8}
\begin{flushright}
   \textbf{Author: Abdullah Oguz Topcuoglu \& Yousef Farag}
\end{flushright}

\section*{Problem 1}
\[
   a_n := \root n \of {n} - 1
\]
We want to prove
\[
   a_n \leq \root \of {\frac{2}{n}} \qquad \forall n \geq 2
\]
and we want to prove
\[
   \lim_{n \rightarrow \infty} \root n \of {n} = 1
\]
\\
Apply the binomial formula to \((1 + a_n)^n\)
\begin{align*}
   (1 + a_n)^n &= n \\
               &= \sum_{k=0}^{n} \binom{n}{k} a_n^k \\
               &= \sum_{k=0}^{n} \binom{n}{k} (\root n \of {n} - 1)^k \\
               &= 1 + \binom{n}{1}(\root n \of {n} - 1) + \binom{n}{2} (\root n \of {n} - 1)^2 + \ldots + (\root n \of {n} - 1)^n \\
\end{align*}
% TODO: idk what to do next
\\
\\
\\
Using the fact that \(a_n \leq \root \of {\frac{2}{n}}\), we want to show
\[
   \lim_{n \rightarrow \infty} \root n \of {n} = 1
\]
We know that
\[
   1 - \frac{1}{n} \leq \root n \of {n}
\]
because \(1 - \frac{1}{n}\) is less than 1 and \(\root n \of {n}\) is greater or equal to 1 for all \(n \geq 1\). \\
Also from the previous inequality we have
\begin{align*}
   \root n \of {n} - 1 &\leq \root \of {\frac{2}{n}} \\
   \root n \of {n} &\leq 1 + \root \of {\frac{2}{n}}
\end{align*}
When combined we have
\[
   1 - \frac{1}{n} \leq \root n \of {n} \leq 1 + \root \of {\frac{2}{n}}
\]
\[
   \lim_{n \rightarrow \infty} 1 - \frac{1}{n} = 1
\]
\[
   \lim_{n \rightarrow \infty} 1 + \root \of {\frac{2}{n}} = 1
\]
Therefore, by the sandwich theorem, we have
\[
   \lim_{n \rightarrow \infty} \root n \of {n} = 1
\]

\section*{Problem 2}
\section*{Problem 2(a)}
\[
   a_n := \frac{4 + 3n^2}{n(2n+1)^2}
\]
We want to prove
\[
   (\frac{1}{n})_{n \in N} \in O((a_n)_{n \in N})
\]
Which means \(\frac{\frac{1}{n}}{a_n}\) is bounded. \\
Let's start
\begin{align*}
   \frac{1}{na_n} &= \frac{n(2n+1)^2}{n(4 + 3n^2)} \\
                  &= \frac{(2n+1)^2}{4 + 3n^2} \\
                  &= \frac{4n^2 + 4n + 1}{4 + 3n^2} \\
                  &= \frac{4 + \frac{4}{n} + \frac{1}{n^2}}{3 + \frac{4}{n^2}} \\
   lim_{n \rightarrow \infty} & \frac{4 + \frac{4}{n} + \frac{1}{n^2}}{3 + \frac{4}{n^2}} = \frac{4}{3}
\end{align*}
Existence of limit implies that \(\frac{1}{na_n}\) is bounded. Therefore, \((\frac{1}{n})_{n \in N} \in O((a_n)_{n \in N})\) \\
The series \(\sum_{n=1}^{\infty} a_n\) is divergent. Since \(lim_{n \rightarrow \infty} \frac{\frac{1}{n}}{a_n} = \frac{4}{3}\), 
the limit of \(lim_{n \rightarrow \infty} \frac{\frac{1}{2n}}{a_n} = \frac{2}{3} \leq 1\) 
which means for almost all \(n \in N\) \(\frac{1}{2n} \leq a_n\). And from the therom 3.44(ii) in the lecture notes,
\(a_n\) is divergent. because \(\sum_{n=1}^{\infty}\frac{1}{2n}\) is divergent and \(a_n \geq \frac{1}{2n} \geq 0\) for almost all \(n \in N\).

\section*{Problem 2(b)}
\section*{Problem 2(b)(i)}
\[
   \sum_{n=1}^{\infty} \frac{(-1)^{n+1}}{n}x^n
\]
We notice that the series is alternating series. Let's check for Leibniz criterion. \\
When is \(x^n/n\) a monotonous null sequence? The answer is when x is in \([-1, 1]\). \\
So we know the series converges for \(x \in [-1, 1]\). \\
For absolute convergence we consider the series
\[
   \sum_{n=1}^{\infty} \frac{x^n}{n}
\]
Let's do ratio test
\begin{align*}
   \lim_{n \rightarrow \infty} \frac{a_{n+1}}{a_n} &= \lim_{n \rightarrow \infty} \frac{x^{n+1}}{n+1} \cdot \frac{n}{x^n} \\
               &= \lim_{n \rightarrow \infty} \frac{x n}{n+1} \\
               &= x \lim_{n \rightarrow \infty} \frac{n}{n+1} \\
               &= x
\end{align*}
By the ratio test we see that if \(x \in (-1, 1)\) then the series converges absolutely. \\
And if \(x > 1, x < -1\) then the series diverges. \\


\section*{Problem 2(b)(ii)}
\[
   \sum_{n=0}^{\infty} \frac{(-1)^n}{(2n+1)!}x^{2n+1}
\]
We notice that the series is alternating series. Let's check for Leibniz criterion. \\
When is \(x^{2n+1}/(2n+1)!\) a monotonous null sequence? The answer is for all x.
Because in the lecture we learnt that factorial grows faster than exponential. \\
For absolute convergence we consider the series
\[
   \sum_{n=0}^{\infty} \frac{x^{2n+1}}{(2n+1)!}
\]
Let's do ratio test
\[
   \lim_{n \rightarrow \infty} \frac{a_{n+1}}{a_n} = \lim_{n \rightarrow \infty} \frac{x^{2n+3}}{(2n+3)!} \cdot \frac{(2n+1)!}{x^{2n+1}} = \lim_{n \rightarrow \infty} \frac{x^2}{(2n+2)(2n+3)} = 0 \quad \forall x \in R
\]
Therefore, the series converges absolutely for all x.

\end{document}